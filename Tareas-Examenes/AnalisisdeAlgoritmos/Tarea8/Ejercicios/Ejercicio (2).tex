
\section{Utilizar BFS o DFS para determinar las componentes conexas de una gráfica
no conexa. Dada una gráfica G, su algoritmo debe ser capaz de indicar el
conjunto de vértices de cada componente conexa.}

Se puede usar el siguiente algoritmo:

\begin{enumerate}
  \item Inicializar un conjunto vacío para almacenar las componentes conexas.
  \item Inicializar una lista vacía para almacenar los nodos visitados.
  \item Para cada nodo no visitado en la gráfica, realizar lo siguiente:
  \begin{enumerate}
    \item  Inicializar una lista vacía para almacenar los nodos de la componente conexa actual.
    \item  Realizar una búsqueda en anchura (BFS) desde el nodo actual,
    marcando cada nodo visitado y agregándolo a la lista de nodos de la componente conexa actual.
    \item Agregar la lista de nodos de la componente conexa actual al conjunto de componentes conexas.
    \item Agregar los nodos visitados a la lista de nodos visitados.
  \end{enumerate}
  \item Devolver el conjunto de componentes conexas. 
\end{enumerate}

\newpage

Pseudocodigo:

\begin{lstlisting}[language=python]
  BFS_componentes_conexas(G):
    visitados = set()
    componentes_conexas = set()
    nodos_visitados = []

    for cada nodo n en G:
        if n no esta en visitados:
            componente_conexa_actual = []
            visitados.add(n)
            componente_conexa_actual.agregar(n)
            nodos_visitados.agregar(n)

            cola = Cola()
            cola.encolar(n)

            while cola no este vacia:
                nodo_actual = cola.desencolar()

                for cada vecino v de nodo_actual:
                    if v no esta en visitados:
                        visitados.add(v)
                        componente_conexa_actual.agregar(v)
                        nodos_visitados.agregar(v)
                        cola.encolar(v)

            componentes_conexas.agregar(conjunto_congelado(componente_conexa_actual))

    return componentes_conexas
\end{lstlisting}
