\section{Proporcionar una gráfica conexa $G(V, A)$ con al menos 17 vértices y al menos 35 aristas con pesos positivos en el intervalo $[1, 8] \cup \mathbb{Z}$ deberá haber al menos cuatro aristas de cada costo $c, c \in [1, 8] \cup \mathbb{Z}$. Aplicar los siguientes algoritmos a la gráfica dada $G$, ilustrando cómo van transformandose las estructuras y mostrando al final los valores de las etiquetas para cada vértice.}

V = {1, 2, 3, 4, 5, 6, 7, 8, 9, 10, 11, 12, 13, 14, 15, 16, 17}\\

A = {(1, 2), (1, 3), (1, 4), (2, 5), (2, 6), (3, 7), (3, 8), (4, 9), (4, 10), (5, 11), (5, 12), (6, 13), (6, 14), (7, 15), (7, 16), (8, 17), (9, 11), (9, 12), (10, 13), (10, 14), (11, 15), (11, 16), (12, 17), (13, 15), (13, 16), (14, 17), (15, 16), (16, 17), (1, 5), (2, 7), (3, 9), (4, 11), (5, 13), (6, 15)}\\

Para facilitar la visualización, la gráfica se representa en forma de lista de adyacencia:

\begin{lstlisting}[language = python]

1: [2, 3, 4, 5],
2: [1, 5, 6, 7],
3: [1, 7, 8, 9],
4: [1, 9, 10],
5: [2, 11, 12],
6: [2, 13, 14],
7: [3, 15, 16],
8: [3, 17],
9: [4, 11, 12],
10: [4, 13, 14],
11: [5, 9, 15, 16],
12: [5, 9, 17],
13: [6, 10, 15, 16],
14: [6, 10, 17],
15: [7, 11, 13, 16],
16: [7, 11, 13, 15, 17],
17: [8, 12, 14, 16]

\end{lstlisting}

\begin{enumerate}



\end{enumerate}
