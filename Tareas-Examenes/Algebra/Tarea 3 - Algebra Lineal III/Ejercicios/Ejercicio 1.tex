\section{Una matriz cuadrada $A$ se llama involutiva si y sólo si $A^2 = I$. Prueba que si
$\lambda$ es valor propio de una matriz involutiva, 
entonces $\lambda = 1$ ó $\lambda = -1$}
\textbf{Demostraci\'on 1:}\\
Sea $A$ una matriz involutiva, entonces por definici\'on $A^2=I$ y sea $\vec{v}\neq 0$ vector propio de $A$ asociado al valor propio $0\neq \lambda\in K$, por definici\'on tenemos que $A\vec{v}=\lambda \vec{v}$. Entonces al aplicar dos veces la matriz (y por la propiedad de sacar dos veces el escalar de una matriz) y como tenemos por hip\'otesis que $A^2=I$:
\[1\cdot \vec{v}=\vec{v}= I \vec{v}=A^2\vec{v}=A(A\vec{v})=A(\lambda \vec{v})=\lambda\cdot A \vec{v}=\lambda\cdot\lambda \vec{v}=\lambda^2\vec{v}\]
\[\therefore 1\cdot\vec{v}=\lambda^2\vec{v}\]
Pero esto pasa sii $\lambda^2=1$, por lo que $\lambda=1$ \'o $\lambda=-1$ (las ra\'ices cuadradas de $1$).\qed
