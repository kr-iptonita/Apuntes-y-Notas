\section{Encontrar el polinomio característico de las matrices siguientes y verifica que se cumple el
Teorema de Cayley-Hamilton:}
\textbf{Soluci\'on 2:}\\
Sabemos que el Teorema de Cayley-Hamilton nos dice que toda matriz cuadrada $A$ satisface  $p_\lambda(A)=\det(A-\lambda I)$es el polinomio característico de $A$ , entonces $p(A)$ es la matriz nula cudarada
\begin{itemize}

    \item $C=\begin{pmatrix}7&-3\\ 5&-2\end{pmatrix}$
    Polinomio definido como: $\det \left(C-\lambda I\right)$
    $$C-\lambda I=\begin{pmatrix}7&-3\\ 5&-2\end{pmatrix}-\lambda\begin{pmatrix}1&0\\ 0&1\end{pmatrix}=\begin{pmatrix}7&-3\\ 5&-2\end{pmatrix}-\begin{pmatrix}\lambda&0\\ 0&\lambda\end{pmatrix}=\begin{pmatrix}7-\lambda&-3\\ 5&-2-\lambda\end{pmatrix}$$
    $$\Longrightarrow \det \begin{pmatrix}7-\lambda&-3\\ 5&-2-\lambda\end{pmatrix}=\left(7-\lambda\right)\left(-2-\lambda\right)-\left(-3\right)(5)$$
    $$p(x)=\lambda^2-5\lambda+1$$
    Verificando\\
    Sustituyendo $C=\lambda$, tenemos $C^2-5C+1I$ 
    $$\begin{pmatrix}7&-3\\ \:\:\:5&-2\end{pmatrix}^2-5\begin{pmatrix}7&-3\\ \:\:\:5&-2\end{pmatrix}+1\begin{pmatrix}1&0\\0&1\end{pmatrix}$$
    $$=\begin{pmatrix}7&-3\\ \:\:\:5&-2\end{pmatrix}\begin{pmatrix}7&-3\\ \:\:\:5&-2\end{pmatrix}+\begin{pmatrix}-35&15\\ \:\:\:-25&10\end{pmatrix}+\begin{pmatrix}1&0\\0&1\end{pmatrix}$$
    $$=\begin{pmatrix}(7)(7)+(-3)(5)&(7)(-3)+(-3)(-2)\\ (5)(7)+(-2)(5)&(5)(-3)+(-2)(-2)\end{pmatrix}+\begin{pmatrix}-35+1&15\\ \:\:\:-25&10+1\end{pmatrix}$$
    $$=\begin{pmatrix}49-15&-21+6\\ \:\:\ 35-10&-15+4\end{pmatrix}+\begin{pmatrix}-34&15\\ \:\:\:-25&11\end{pmatrix}=\begin{pmatrix}34&-15\\ \:\:\ 25&-11\end{pmatrix}+\begin{pmatrix}-34&15\\ \:\:\:-25&11\end{pmatrix}=\begin{pmatrix}0&0\\ \:\:\:0&0\end{pmatrix}$$
    $$\therefore C^2-5C+1I=0$$
    Por lo que el teorema se cumple
    
    \item $D=\begin{pmatrix}1&2&3\\ 3&0&6\\ 5&6&4\end{pmatrix}$\\
    Calculamos:
    \[D-\lambda I=\begin{pmatrix}1&2&3\\ 3&0&6\\ 5&6&4\end{pmatrix}-\lambda \begin{pmatrix}1&0&0\\ 0&1&0\\ 0&0&1\end{pmatrix}=\begin{pmatrix}1-\lambda&2&3\\ 3&-\lambda&6\\ 5&6&4-\lambda\end{pmatrix}\]
    Entonces calculando su determinante por cofactores del primer rengl\'on, tenemos:
    $$\begin{vmatrix}1-\lambda&2&3\\ 3&-\lambda&6\\ 5&6&4-\lambda\end{vmatrix}$$
    $$=\left(1-\lambda\right)\begin{vmatrix}-\lambda&6\\ 6&4-\lambda\end{vmatrix}-2 \begin{vmatrix}3&6\\ 5&4-\lambda\end{vmatrix}+3 \begin{vmatrix}3&-\lambda\\ 5&6\end{vmatrix}$$
    $$=(1-\lambda)[\left(-\lambda\right)\left(4-\lambda\right)-(6)(6)]-2[3\left (4-\lambda\right)-(6)(5)]+3[(3)(6)-(5)(-\lambda)]$$
    $$=\left(1-\lambda\right)\left(-4\lambda+\lambda^2-36\right)-2\left(-3\lambda-18\right)+3\left(18+5\lambda\right)$$
    $$=-\lambda^3+5\lambda^2+32\lambda-36+6\lambda+36+54+15\lambda$$
    $$p(x)=-\lambda^3+5\lambda^2+53\lambda+54$$
    Verificando, primero calculamos $D^2$ y $D^3$:
    \[D^2 =\begin{pmatrix}1&2&3\\ 3&0&6\\ 5&6&4\end{pmatrix}\begin{pmatrix}1&2&3\\ 3&0&6\\ 5&6&4\end{pmatrix}= \begin{pmatrix}(1)(1)+(2)(3)+(3)(5)&(1)(2)+(2)(0)+(3)(6)&(1)(3)+(2)(6)+(3)(4)\\ 
    (3)(1)+(0)(3)+(6)(5)&(3)(2)+(0)(0)+(6)(6)&(3)(3)+(0)(6)+(6)(4)\\
    (5)(1)+(6)(3)+(4)(5)&(5)(2)+(6)(0)+(4)(6)&(5)(3)+(6)(6)+(4)(4)\\\end{pmatrix}\]\[= \begin{pmatrix}1+6+15&2+0+18&3+12+12\\ 
    3+0+30&6+0+36&9+0+24\\
    5+18+20&10+0+24&15+36+16\\\end{pmatrix}=\begin{pmatrix}22&20&27\\ 33&42&33\\ 43&34&67\end{pmatrix}\]
    \[D^3=D\cdot D^2=\begin{pmatrix}1&2&3\\ 3&0&6\\ 5&6&4\end{pmatrix}\begin{pmatrix}22&20&27\\ 33&42&33\\ 43&34&67\end{pmatrix}\]\[= \begin{pmatrix}(1)(22)+(2)(33)+(3)(43)&(1)(20)+(2)(42)+(3)(34)&(1)(27)+(2)(33)+(3)(67)\\ 
    (3)(22)+(0)(33)+(6)(43)&(3)(20)+(0)(42)+(6)(34)&(3)(27)+(0)(33)+(6)(67)\\
    (5)(22)+(6)(33)+(4)(43)&(5)(20)+(6)(42)+(4)(34)&(5)(27)+(6)(33)+(4)(67)\\\end{pmatrix}\]\[=\begin{pmatrix}22+66+129&20+84+102&27+66+201\\66+0+258&60+0+204&81+0+402\\110+198+172&100+252+136&135+198+268\end{pmatrix}=\begin{pmatrix}217&206&294\\ 324&264&483\\ 480&488&601\end{pmatrix}\]
    Sustituyendo $D=\lambda$ en $-D^3+5D^2+53D+54I$
    $$=-\begin{pmatrix}217&206&294\\ 324&264&483\\ 480&488&601\end{pmatrix}+5\begin{pmatrix}1&2&3\\ 3&0&6\\ 5&6&4\end{pmatrix}^2+53\begin{pmatrix}1&2&3\\ 
    3&0&6\\ 5&6&4\end{pmatrix}+54I$$
    $$=-\begin{pmatrix}217&206&294\\ 324&264&483\\ 480&488&601\end{pmatrix}+5\begin{pmatrix}22&20&27\\ 33&42&33\\ 43&34&67\end{pmatrix}+53\begin{pmatrix}1&2&3\\ 3&0&6\\ 5&6&4\end{pmatrix}+54I$$
    $$=\begin{pmatrix}217&206&294\\ 324&264&483\\ 480&488&601\end{pmatrix}+\begin{pmatrix}110&100&135\\ 165&210&165\\ 215&170&335\end{pmatrix}+\begin{pmatrix}53&106&159\\ 159&0&318\\ 265&318&212\end{pmatrix}+54I$$
    $$=\begin{pmatrix}-107&-106&-159\\ -159&-54&-318\\ -265&-318&-266\end{pmatrix}+\begin{pmatrix}53&106&159\\ 159&0&318\\ 265&318&212\end{pmatrix}+54I$$
    $$=\begin{pmatrix}-54&0&0\\ 0&-54&0\\ 0&0&-54\end{pmatrix}+54I$$
    $$-54I+54I=0$$
    Por tanto, se cumple el teorema
    
    \item $E=\begin{pmatrix}1&6&-2\\ -3&2&0\\
    0&3&-4\end{pmatrix}$
    Calculando epor cofactores el determinante de $E-\lambda I$ sobre el primer rengl\'on tenemos:
    $$\begin{vmatrix}1-\lambda&6&-2\\ -3&2-\lambda&0\\ 0&3&-4-\lambda\end{vmatrix}$$
    $$=\left(1-\lambda\right)\begin{vmatrix}2-\lambda&0\\ 3&-4-\lambda\end{vmatrix}-6\begin{vmatrix}-3&0\\ 0&-4-\lambda\end{vmatrix}-2\begin{vmatrix}-3&2-\lambda\\ 0&3\end{vmatrix}$$
    $$=\left(1-\lambda\right)\left(2-\lambda\right)\left(-4-\lambda\right)-0\cdot \:3 -6(\left(-3\right)\left(-4-\lambda\right)-0\cdot \:0)-2(\left(-3\right)\cdot \:3-\left(2-\lambda\right)\cdot \:0)$$
    $$=\left(1-\lambda\right)\left(\lambda^2+2\lambda-8\right)-6\left(-3\left(-\lambda-4\right)\right)-2\left(-9\right)$$
    $$=-\lambda^3-\lambda^2-8\lambda-62$$
    Verificando, primero calculamos $E^2$ y $E^3$:
    \[E^2 =\begin{pmatrix}1&6&-2\\ -3&2&0\\ 0&3&-4\end{pmatrix}\begin{pmatrix}1&6&-2\\ -3&2&0\\ 0&3&-4\end{pmatrix}\]\[= \begin{pmatrix}(1)(1)+(6)(-3)+(-2)(0)&(1)(6)+(6)(2)+(-2)(3)&(1)(-2)+(6)(0)+(-2)(-4)\\ 
    (-3)(1)+(2)(-3)+(0)(0)&(-3)(6)+(2)(2)+(0)(3)&(-3)(-2)+(2)(0)+(0)(-4)\\
    (0)(1)+(3)(-3)+(-4)(0)&(0)(6)+(3)(2)+(-4)(3)&(0)(-2)+(3)(0)+(-4)(-4)\\ \end{pmatrix}\]\[= \begin{pmatrix}1-18+0&6+12-6&-2+0+8\\ 
    -3-6+0&-18+4+0&6+0+0\\
    0-9+0&0+6-12&0+0-16\\\end{pmatrix}=\begin{pmatrix}-17&12&6\\ -9&-14&6\\ -9&-6&16\end{pmatrix}\]
    \[E^3=E\cdot E^2=\begin{pmatrix}1&6&-2\\ -3&2&0\\ 0&3&-4\end{pmatrix}\begin{pmatrix}-17&12&6\\ -9&-14&6\\ -9&-6&16\end{pmatrix}\]\[= \begin{pmatrix}(1)(-17)+(6)(-9)+(-2)(-9)&(1)(12)+(6)(-14)+(-2)(-6)&(1)(6)+(6)(6)+(-2)(16)\\ 
    (-3)(-17)+(2)(-9)+(0)(-9)&(-3)(12)+(2)(-14)+(0)(-6)&(-3)(6)+(2)(6)+(0)(16)\\
    (0)(-17)+(3)(-9)+(-4)(-9)&(0)(12)+(3)(-14)+(-4)(-6)&(0)(6)+(3)(6)+(-4)(16)\\ \end{pmatrix}\]\[= \begin{pmatrix}-17-54+18&12-84+12&6+36-32\\ 
    51-18+0&-36-28+0&-18+12+0\\
    0-27+36&0-42+24&0+18-64\\\end{pmatrix}=\begin{pmatrix}-53&-60&10\\ 33&-64&-6\\ 9&-18&-46\end{pmatrix}\]
    
    Sustituyendo $E=\lambda$ en $-E^3-E^2-8E-62I$
    $$=-\begin{pmatrix}1&6&-2\\ \:\:\:-3&2&0\\ \:\:\:0&3&-4\end{pmatrix}^3-\begin{pmatrix}1&6&-2\\ \:\:\:-3&2&0\\ \:\:\:0&3&-4\end{pmatrix}^2-8\begin{pmatrix}1&6&-2\\ \:\:\:-3&2&0\\ \:\:\:0&3&-4\end{pmatrix}-62I$$
    $$=-\begin{pmatrix}-53&-60&10\\ 33&-64&-6\\ 9&-18&-46\end{pmatrix}-\begin{pmatrix}1&6&-2\\ -3&2&0\\ 0&3&-4\end{pmatrix}^2-8\begin{pmatrix}1&6&-2\\ -3&2&0\\ 0&3&-4\end{pmatrix}-62I$$
    $$=\begin{pmatrix}-53&-60&10\\ 33&-64&-6\\ 9&-18&-46\end{pmatrix}-\begin{pmatrix}-17&12&6\\ -9&-14&6\\ -9&-6&16\end{pmatrix}-8\begin{pmatrix}1&6&-2\\ -3&2&0\\ 0&3&-4\end{pmatrix}-62I$$
    $$=-\begin{pmatrix}-53&-60&10\\ 33&-64&-6\\ 9&-18&-46\end{pmatrix}-\begin{pmatrix}-17&12&6\\ -9&-14&6\\ -9&-6&16\end{pmatrix}-\begin{pmatrix}8&48&-16\\ -24&16&0\\ 0&24&-32\end{pmatrix}-62I$$
    $$=\begin{pmatrix}53&60&-10\\ -33&64&6\\ -9&18&46\end{pmatrix}-\begin{pmatrix}-17&12&6\\ -9&-14&6\\ -9&-6&16\end{pmatrix}-\begin{pmatrix}8&48&-16\\ -24&16&0\\ 0&24&-32\end{pmatrix}-62I$$
    $$=\begin{pmatrix}70&48&-16\\ -24&78&0\\ 0&24&30\end{pmatrix}-\begin{pmatrix}8&48&-16\\ -24&16&0\\ 0&24&-32\end{pmatrix}-62I$$
    $$=\begin{pmatrix}62&0&0\\ 0&62&0\\ 0&0&62\end{pmatrix}-62I$$
    $$=62I-62I=0$$
    Por tanto, se cumple el teorema 

\end{itemize}{}