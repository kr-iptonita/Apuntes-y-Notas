\section{Determina los valores y vectores propios de las siguientes transformaciones (operadores):}
\textbf{Soluci\'on 9:}\\
\begin{itemize}
\item[$a)$] $T: \mathbb{R}^2 \rightarrow \mathbb{R}^2$ definida por $T(x, y) = (4x + 3y, 3x - 4y)$.\\\\
Sabemos que para que para obtener los eigenvalore y eigenvectores de una transformaci\'on $T$ es necesario primero obtener su matriz, en nuestro caso, construiremos la matriz $A$ de $2\times 2$ asociada a la transformaci\'on $T$ respecto a la base can\'onica de $\mathbb{R}^2$, entonces primero debemos calcular la transformaci\'on para cada vector de la base can\'onica, es decir:
\[T(\vec{e_1})=T(1,0)=(4(1)+3(0),3(1)-4(0))=(4,3)\]
\[T(\vec{e_2})=T(0,1)=(4(0)+3(1),3(0)-4(1))=(3,-4)\]
Entonces definimos a $A$ en el que cada columna es el vector resultante de la transfromaci\'on al vector dentro de la base:
\[A=\begin{pmatrix}T(\vec{e_1})&T(\vec{e_2})\end{pmatrix}=\begin{pmatrix}4&3\\
3&-4\end{pmatrix}\]
Y como est\'a construida respecto de la base can\'onica, no es necesario calcular la combinaci\'on lineal respecto de la base can\'onica por cada vector (pues son los coeficientes del vector mismo).\\
Sabemos que para encontrar un valor propio $\lambda_n$ de una matriz $A\in\mathcal{M}_{n\times n }$ que satisfaga $T(\vec{v})=A\vec{v}=\lambda_n\vec{v}$, debemos tener que $\text{det}(A-\lambda_nI_{n\times n })=0$, de esta manera obtenemos su polinomio caracter\'istico, cuyas $n$ ra\'ices son sus valores propios.
Lo primero que se deber\'a hacer es encontrar su polinomio caracter\'istico:
\[0=\text{det}(A-\lambda_nI_{n\times n })=\left|\begin{pmatrix}
4&3\\
3&-4
\end{pmatrix}-\lambda\begin{pmatrix}
1 &0\\0 &1
\end{pmatrix}\right|=\begin{vmatrix}4-\lambda &3\\
3 &-4-\lambda\end{vmatrix}=(4-\lambda)(-4-\lambda)-(3)(3)\]\[=-16+\lambda^2-9=\lambda^2-25\]
Por lo que su polinomio caracter\'istico es $\lambda^2-25=(\lambda+5)(\lambda-5)=0$, de modo que sus ra\'ices (valores propios) son $\lambda_1=-5$ y $\lambda_2=5$. Ahora lo que haremos ser\'a calcular los vectores propios de cada uno, para lo cual debemos resolver la ecuaci\'on $(A-\lambda_nI_{n\times n })\vec{v}=\vec{0}$ (con $\vec{v}=(x,y)$):
\begin{itemize}
    \item Para $\lambda_1=-5$, sustituyendo, tenemos que:
    \[\begin{pmatrix}0\\
0\end{pmatrix}=\vec{0}=(A-\lambda_1I_{n\times n })\vec{v}=\begin{pmatrix}4-(-5) &3\\
3&-4-(-5)\end{pmatrix}\begin{pmatrix}x\\
y\end{pmatrix}=\begin{pmatrix}9 &3\\
3 &1\end{pmatrix}\begin{pmatrix}x\\
y\end{pmatrix}=\begin{pmatrix}9x+3y\\
3x+y\end{pmatrix}\]
\[\therefore \begin{pmatrix}0\\
0\end{pmatrix}=\begin{pmatrix}9x+3y\\
3x+y\end{pmatrix}\]
Por lo que tenemos el siguiente sistema de ecuaciones:
\begin{eqnarray*}
9x+3y&=&0\\
3x+y&=&0
\end{eqnarray*}
Pero si nos damos cuenta la primera ecuaci\'on es m\'ultiplo de la segunda, pues si dividimos los coeficientes de las $x$ entre los coeficientes de las $y$ nos queda $\displaystyle\frac{9}{3}=\frac{3}{1}=3$.\\
Si despejamos de la segunda ecuaci\'on tenemos que $y=-3x$, por lo que si damos a $x\in\mathbb{R}$ como valor fijo tenemos un vector $(x,y)=(x,-3x)=x(1,-3)$.\\
S.P.G., damos el valor de $x=1$, por lo que el valor propio $\lambda_1=-5$ tiene asociado el vector propio $\vec{v_1}=(1,-3)$.


\item Para $\lambda_2=5$, sustituyendo, tenemos que:
    \[\begin{pmatrix}0\\
0\end{pmatrix}=\vec{0}=(A-\lambda_1I_{n\times n })\vec{v}=\begin{pmatrix}4-(5) &3\\
3&-4-(5)\end{pmatrix}\begin{pmatrix}x\\
y\end{pmatrix}=\begin{pmatrix}-1 &3\\
3 &-9\end{pmatrix}\begin{pmatrix}x\\
y\end{pmatrix}=\begin{pmatrix}-x+3y\\
3x-9y\end{pmatrix}\]
\[\therefore \begin{pmatrix}0\\
0\end{pmatrix}=\begin{pmatrix}-x+3y\\
3x-9y\end{pmatrix}\]
Por lo que tenemos el siguiente sistema de ecuaciones:
\begin{eqnarray*}
-x+3y&=&0\\
3x-9y&=&0
\end{eqnarray*}
Pero si nos damos cuenta la primera ecuaci\'on es m\'ultiplo de la segunda, pues si dividimos los coeficientes de las $x$ entre los coeficientes de las $y$ nos queda $\displaystyle\frac{-1}{3}=\frac{3}{-9}=-\frac{1}{3}$.\\
Si despejamos de la segunda ecuaci\'on tenemos que $3y=x$, por lo que si damos a $y\in\mathbb{R}$ como valor fijo tenemos un vector $(x,y)=(3y,y)=y(3,1)$.\\
S.P.G., damos el valor de $x=1$, por lo que el valor propio $\lambda_2=5$ tiene asociado el vector propio $\vec{v_2}=(3,1)$.

\end{itemize}
\item[$b)$] $T: \mathbb{R}^3 \rightarrow \mathbb{R}^3$ dada por $T(x, y, z) = (2y - z, 2x - z, 2x -y)$.\\\\
Sabemos que para que para obtener los eigenvalore y eigenvectores de una transformaci\'on $T$ es necesario primero obtener su matriz, en nuestro caso, construiremos la matriz $B$ de $3\times 3$ asociada a la transformaci\'on $T$ respecto a la base can\'onica de $\mathbb{R}^3$, entonces primero debemos calcular la transformaci\'on para cada vector de la base can\'onica, es decir:
\[T(\vec{e_1})=T(1,0,0)=(2(0) - (0), 2(1) - (0), 2(1) -(0))=(0,2,2)\]
\[T(\vec{e_2})=T(0,1,0)=(2(1) - (0), 2(0) - (0), 2(0) -(1))=(2,0,-1)\]
\[T(\vec{e_3})=T(0,0,1)=(2(0) - (1), 2(0) - (1), 2(0) -(0))=(-1-1,0)\]
Entonces definimos a $A$ en el que cada columna es el vector resultante de la transfromaci\'on al vector dentro de la base:
\[B=\begin{pmatrix}T(\vec{e_1})&T(\vec{e_2})&T(\vec{e_3})\end{pmatrix}=\begin{pmatrix}0&2&-1\\2&0&-1\\2&-1&0\end{pmatrix}\]
Y como est\'a construida respecto de la base can\'onica, no es necesario calcular la combinaci\'on lineal respecto de la base can\'onica por cada vector (pues son los coeficientes del vector mismo).\\
Sabemos que para encontrar un valor propio $\lambda_n$ de una matriz $B\in\mathcal{M}_{n\times n }$ que satisfaga $T(\vec{v})=B\vec{v}=\lambda_n\vec{v}$, debemos tener que $\text{det}(B-\lambda_nI_{n\times n })=0$, de esta manera obtenemos su polinomio caracter\'istico, cuyas $n$ ra\'ices son sus valores propios.
Lo primero que se deber\'a hacer es encontrar su polinomio caracter\'istico, lo cual haremos por cofactores sobre la primera columna:
\[0=\text{det}(B-\lambda_nI_{n\times n })=\left|\begin{pmatrix}
0&2&-1\\
2&0&-1\\
2&-1&0
\end{pmatrix}-\lambda\begin{pmatrix}
1 &0&0\\0 &1&0\\0&0&1
\end{pmatrix}\right|=\begin{vmatrix}-\lambda&2&-1\\
2&-\lambda&-1\\
2&-1&-\lambda\end{vmatrix}=-\lambda\begin{vmatrix}
-\lambda&-1\\
-1&-\lambda\end{vmatrix}-2\begin{vmatrix}2&-1\\
-1&-\lambda\end{vmatrix}+2\begin{vmatrix}2&-1\\
-\lambda&-1\end{vmatrix}\]\[=-\lambda[(-\lambda)(-\lambda)-(-1)(-1)]-2[(2)(-\lambda)-(-1)(-1)]+2[(2)(-1)-(-\lambda)(-1)]=-\lambda[\lambda^2-1]-2[-2\lambda-1]+2[-2-\lambda]\]\[=-\lambda^3+\lambda+4\lambda+2-4-2\lambda=-\lambda^3+3\lambda-2\]
Por lo que su polinomio caracter\'istico es $\lambda^3-3\lambda+2=0$, del cual si jos damos cuenta 1 es ra\'iz, pues $1-3+2=0$, de modo que sus ra\'ices (valores propios) son:
\[\lambda^3-3\lambda+2=(\lambda-1)(\lambda^2+\lambda-2)=(\lambda-1)(\lambda-1)(\lambda+2)=0\]
$\lambda_1=\lambda_2=1$ (tiene multiplicidad 2) y $\lambda_3=-2$. Ahora lo que haremos ser\'a calcular los vectores propios de cada uno, para lo cual debemos resolver la ecuaci\'on $(B-\lambda_nI_{n\times n })\vec{v}=\vec{0}$ (con $\vec{v}=(x,y,z)$):
\begin{itemize}
    \item Para $\lambda_1=\lambda_2=1$, sustituyendo, tenemos que:
    \[\begin{pmatrix}0\\
0\\0\end{pmatrix}=\vec{0}=(B-\lambda_1I_{n\times n })\vec{v}=\begin{pmatrix}-1&2&-1\\
2&-1&-1\\
2&-1&-1\end{pmatrix}\begin{pmatrix}x\\
y\\z\end{pmatrix}=\begin{pmatrix}-x+2y-z\\
2x-y-z\\2x-y-z\end{pmatrix}\]
\[\therefore \begin{pmatrix}0\\
0\\0\end{pmatrix}=\begin{pmatrix}-x+2y-z\\
2x-y-z\\2x-y-z\end{pmatrix}\]
Por lo que tenemos el siguiente sistema de ecuaciones:
\begin{eqnarray*}
-x+2y-z&=&0\\
2x-y-z&=&0\\2x-y-z&=&0
\end{eqnarray*}
Pero si nos damos cuenta la segunda ecuaci\'on es igual a la tercera,
, por lo que si restamos la primera ecuaci\'on a la segunda, tenemos:
\[2x-y-z-(-x+2y-z)=0-0~~\Longrightarrow~3x-3y=0~~\Longrightarrow~x=y\]
De modo que al sustituir en la tercera:
\[2x-y-z=2y-y-z=y-z=0~~\Longrightarrow~y=z\]Por lo que si damos a $x\in\mathbb{R}$ como valor fijo tenemos un vector $(x,y,z)=(x,x,x)=x(1,1,1)$.\\
S.P.G., damos el valor de $x=1$, por lo que el valor propio $\lambda_1=\lambda_2=1$ tiene asociado el vector propio $\vec{v_1}=\vec{v_2}=(1,1,1)$.


\item Para $\lambda_2=-2$, sustituyendo, tenemos que:
    \[\begin{pmatrix}0\\
0\\0\end{pmatrix}=\vec{0}=(B-\lambda_1I_{n\times n })\vec{v}=\begin{pmatrix}2&2&-1\\
2&2&-1\\
2&-1&2\end{pmatrix}\begin{pmatrix}x\\
y\\z\end{pmatrix}=\begin{pmatrix}2x+2y-z\\
2x+2y-z\\2x-y+2z\end{pmatrix}\]
\[\therefore \begin{pmatrix}0\\
0\\0\\0\end{pmatrix}=\begin{pmatrix}2x+2y-z\\
2x+2y-z\\2x-y+2z\end{pmatrix}\]
Por lo que tenemos el siguiente sistema de ecuaciones:
\begin{eqnarray*}
2x+2y-z&=&0\\
2x+2y-z&=&0\\2x-y+2z&=&0
\end{eqnarray*}
Pero si nos damos cuenta la primer ecuaci\'on es igual a la segunda,
, por lo que si restamos la tercera ecuaci\'on a la primera, tenemos:
\[2x+2y-z-(2x-y+2z)=0-0~~\Longrightarrow~3y-3z=0~~\Longrightarrow~y=z\]
De modo que al sustituir en la tercera:
\[2x+2y-z=2x+2y-y=2x+y=0~~\Longrightarrow~y=-2x\]Por lo que si damos a $x\in\mathbb{R}$ como valor fijo tenemos un vector $(x,y,z)=(x,-2x,-2x)=x(1,-2,-2)$.\\
S.P.G., damos el valor de $x=1$, por lo que el valor propio $\lambda_3=-2$ tiene asociado el vector propio $\vec{v_3}=(1,-2,-2)$.


\end{itemize}
\end{itemize}