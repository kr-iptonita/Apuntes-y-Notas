\documentclass{article}
\usepackage[utf8]{inputenc}
\usepackage{amsmath,amssymb,amsthm}
\usepackage[spanish, mexico]{babel}
\usepackage{graphicx}
\usepackage[table,xcdraw]{xcolor}
\usepackage{float} 
\usepackage{wrapfig}
\usepackage{multirow, array} 
\usepackage{polynom}
\usepackage{parskip}
\usepackage[left=2.00cm, right=2.00cm, top=2.00cm, bottom=2.00cm]{geometry}
\usepackage{float} % para usar [H]

\graphicspath{.JPG}
\renewcommand{\qed}{$\blacksquare$}
\newcommand\tab[1][1cm]{\hspace*{#1}}
\newcommand{\ie}{\textit{i}.\textit{e}.}
\newcommand{\eg}{\textit{e}.\textit{g}.}
\providecommand{\norm}[1]{\lVert#1\rVert}


\begin{document}

\begin{titlepage}
	\centering
	
	{\scshape\LARGE Universidad Nacional Autónoma de México\par}
	\vspace{1cm}
	{\scshape\Large Facultad de Ciencias \par}
	{\huge\bfseries Tarea 2 \par}
	{\Large\itshape Análisis de Algoritmos \par}
	
	\begin{table}[ht]
	\centering
	\rowcolors{1}{pastelgreen}{pastelpink}
	\begin{tabular}{|l|l|}
	\hline
	\multicolumn{2}{|c|}{\cellcolor{pastelgreen}\textbf{Información del curso}} \\ \hline
	Profesor & María de Luz Gasca Soto \\
	Ayudante & Brenda Margarita Becerra Ruíz \\
	Ayudante & Enrique Ehecatl Hernández Ferreiro \\
	\hline
	\end{tabular}
	\end{table}

	{\large\itshape Autor: Juárez Torres Carlos Alberto \par}

	\begin{center}
		\begin{verbatim}
			______________1¶¶¶____¶¶¶1___¶¶¶1_________________
			_____________¶¶¶1___1¶¶1___1¶¶1___________________
			____________1¶¶1___1¶¶1___1¶¶1____________________
			____________1¶¶1___1¶¶1___1¶¶¶____________________
			_____________¶¶¶____¶¶¶1___¶¶¶1___________________
			______________¶¶¶¶___1¶¶¶___1¶¶¶__________________
			_______________1¶¶¶1___¶¶¶1___¶¶¶¶________________
			_________________1¶¶1____¶¶¶____¶¶¶_______________
			___________________¶¶1____¶¶1____¶¶1______________
			___________________¶¶¶____¶¶¶____¶¶¶______________
			__________________1¶¶1___1¶¶1____¶¶1______________
			_________________¶¶¶____¶¶¶1___1¶¶1_______________
			________________11_____111_____11_________________		Toma un café mañanero UwU
			__________¶¶¶¶¶¶¶¶¶¶¶¶¶¶¶¶¶¶¶¶¶¶¶¶¶¶¶¶¶¶¶¶________
			1¶¶¶¶¶¶¶¶¶¶¶__¶¶¶¶¶¶¶¶¶¶¶¶¶¶¶¶¶¶¶¶¶¶¶¶¶¶¶¶________
			1¶¶¶¶¶¶¶¶¶¶¶__1¶¶¶¶¶¶¶¶¶¶¶¶¶¶¶¶¶¶¶¶¶¶¶¶¶¶¶________
			1¶¶_______¶¶__1¶¶¶¶¶¶¶¶¶¶¶¶¶¶¶¶¶¶¶¶¶¶¶¶¶¶¶________
			1¶¶_______¶¶__1¶¶¶¶¶¶¶¶¶¶¶¶¶¶¶¶¶¶¶¶¶¶¶¶¶¶¶________
			1¶¶_______¶¶__¶¶¶¶¶¶¶¶¶¶¶¶¶¶¶¶¶¶¶¶¶¶¶¶¶¶¶¶________
			1¶¶_______¶¶__1¶¶¶¶¶¶¶¶¶¶¶¶¶¶¶¶¶¶¶¶¶¶¶¶¶¶¶________
			_¶¶¶¶¶¶¶¶¶¶¶__¶¶¶¶¶¶¶¶¶¶¶¶¶¶¶¶¶¶¶¶¶¶¶¶¶¶¶¶________
			_¶¶¶¶¶¶¶¶¶¶¶__¶¶¶¶¶¶¶¶¶¶¶¶¶¶¶¶¶¶¶¶¶¶¶¶¶¶¶¶________
			__________¶¶___1¶¶¶¶¶¶¶¶¶¶¶¶¶¶¶¶¶¶¶¶¶¶¶¶¶1________
			__________1¶¶___¶¶¶¶¶¶¶¶¶¶¶¶¶¶¶¶¶¶¶¶¶¶¶¶¶_________
			____________¶¶¶¶¶¶¶¶¶¶¶¶¶¶¶¶¶¶¶¶¶¶¶¶¶¶11__________
			11_____________________________________________111
			1¶¶¶¶¶¶¶¶¶¶¶¶¶¶¶¶¶¶¶¶¶¶¶¶¶¶¶¶¶¶¶¶¶¶¶¶¶¶¶¶¶¶¶¶¶¶¶¶1
			__¶¶111111111¶¶¶¶¶¶¶¶¶¶¶¶¶¶¶¶¶¶¶¶¶¶¶¶¶111111111¶__
			
		\end{verbatim}
	\end{center}
	
	{\large Fecha de entrega: \today\par}
	
\end{titlepage}

%Ejercicios%


\section{Una matriz cuadrada $A$ se llama involutiva si y sólo si $A^2 = I$. Prueba que si
$\lambda$ es valor propio de una matriz involutiva, 
entonces $\lambda = 1$ ó $\lambda = -1$}
\textbf{Demostraci\'on 1:}\\
Sea $A$ una matriz involutiva, entonces por definici\'on $A^2=I$ y sea $\vec{v}\neq 0$ vector propio de $A$ asociado al valor propio $0\neq \lambda\in K$, por definici\'on tenemos que $A\vec{v}=\lambda \vec{v}$. Entonces al aplicar dos veces la matriz (y por la propiedad de sacar dos veces el escalar de una matriz) y como tenemos por hip\'otesis que $A^2=I$:
\[1\cdot \vec{v}=\vec{v}= I \vec{v}=A^2\vec{v}=A(A\vec{v})=A(\lambda \vec{v})=\lambda\cdot A \vec{v}=\lambda\cdot\lambda \vec{v}=\lambda^2\vec{v}\]
\[\therefore 1\cdot\vec{v}=\lambda^2\vec{v}\]
Pero esto pasa sii $\lambda^2=1$, por lo que $\lambda=1$ \'o $\lambda=-1$ (las ra\'ices cuadradas de $1$).\qed

\section{Encontrar el polinomio característico de las matrices siguientes y verifica que se cumple el
Teorema de Cayley-Hamilton:}
\textbf{Soluci\'on 2:}\\
Sabemos que el Teorema de Cayley-Hamilton nos dice que toda matriz cuadrada $A$ satisface  $p_\lambda(A)=\det(A-\lambda I)$es el polinomio característico de $A$ , entonces $p(A)$ es la matriz nula cudarada
\begin{itemize}

    \item $C=\begin{pmatrix}7&-3\\ 5&-2\end{pmatrix}$
    Polinomio definido como: $\det \left(C-\lambda I\right)$
    $$C-\lambda I=\begin{pmatrix}7&-3\\ 5&-2\end{pmatrix}-\lambda\begin{pmatrix}1&0\\ 0&1\end{pmatrix}=\begin{pmatrix}7&-3\\ 5&-2\end{pmatrix}-\begin{pmatrix}\lambda&0\\ 0&\lambda\end{pmatrix}=\begin{pmatrix}7-\lambda&-3\\ 5&-2-\lambda\end{pmatrix}$$
    $$\Longrightarrow \det \begin{pmatrix}7-\lambda&-3\\ 5&-2-\lambda\end{pmatrix}=\left(7-\lambda\right)\left(-2-\lambda\right)-\left(-3\right)(5)$$
    $$p(x)=\lambda^2-5\lambda+1$$
    Verificando\\
    Sustituyendo $C=\lambda$, tenemos $C^2-5C+1I$ 
    $$\begin{pmatrix}7&-3\\ \:\:\:5&-2\end{pmatrix}^2-5\begin{pmatrix}7&-3\\ \:\:\:5&-2\end{pmatrix}+1\begin{pmatrix}1&0\\0&1\end{pmatrix}$$
    $$=\begin{pmatrix}7&-3\\ \:\:\:5&-2\end{pmatrix}\begin{pmatrix}7&-3\\ \:\:\:5&-2\end{pmatrix}+\begin{pmatrix}-35&15\\ \:\:\:-25&10\end{pmatrix}+\begin{pmatrix}1&0\\0&1\end{pmatrix}$$
    $$=\begin{pmatrix}(7)(7)+(-3)(5)&(7)(-3)+(-3)(-2)\\ (5)(7)+(-2)(5)&(5)(-3)+(-2)(-2)\end{pmatrix}+\begin{pmatrix}-35+1&15\\ \:\:\:-25&10+1\end{pmatrix}$$
    $$=\begin{pmatrix}49-15&-21+6\\ \:\:\ 35-10&-15+4\end{pmatrix}+\begin{pmatrix}-34&15\\ \:\:\:-25&11\end{pmatrix}=\begin{pmatrix}34&-15\\ \:\:\ 25&-11\end{pmatrix}+\begin{pmatrix}-34&15\\ \:\:\:-25&11\end{pmatrix}=\begin{pmatrix}0&0\\ \:\:\:0&0\end{pmatrix}$$
    $$\therefore C^2-5C+1I=0$$
    Por lo que el teorema se cumple
    
    \item $D=\begin{pmatrix}1&2&3\\ 3&0&6\\ 5&6&4\end{pmatrix}$\\
    Calculamos:
    \[D-\lambda I=\begin{pmatrix}1&2&3\\ 3&0&6\\ 5&6&4\end{pmatrix}-\lambda \begin{pmatrix}1&0&0\\ 0&1&0\\ 0&0&1\end{pmatrix}=\begin{pmatrix}1-\lambda&2&3\\ 3&-\lambda&6\\ 5&6&4-\lambda\end{pmatrix}\]
    Entonces calculando su determinante por cofactores del primer rengl\'on, tenemos:
    $$\begin{vmatrix}1-\lambda&2&3\\ 3&-\lambda&6\\ 5&6&4-\lambda\end{vmatrix}$$
    $$=\left(1-\lambda\right)\begin{vmatrix}-\lambda&6\\ 6&4-\lambda\end{vmatrix}-2 \begin{vmatrix}3&6\\ 5&4-\lambda\end{vmatrix}+3 \begin{vmatrix}3&-\lambda\\ 5&6\end{vmatrix}$$
    $$=(1-\lambda)[\left(-\lambda\right)\left(4-\lambda\right)-(6)(6)]-2[3\left (4-\lambda\right)-(6)(5)]+3[(3)(6)-(5)(-\lambda)]$$
    $$=\left(1-\lambda\right)\left(-4\lambda+\lambda^2-36\right)-2\left(-3\lambda-18\right)+3\left(18+5\lambda\right)$$
    $$=-\lambda^3+5\lambda^2+32\lambda-36+6\lambda+36+54+15\lambda$$
    $$p(x)=-\lambda^3+5\lambda^2+53\lambda+54$$
    Verificando, primero calculamos $D^2$ y $D^3$:
    \[D^2 =\begin{pmatrix}1&2&3\\ 3&0&6\\ 5&6&4\end{pmatrix}\begin{pmatrix}1&2&3\\ 3&0&6\\ 5&6&4\end{pmatrix}= \begin{pmatrix}(1)(1)+(2)(3)+(3)(5)&(1)(2)+(2)(0)+(3)(6)&(1)(3)+(2)(6)+(3)(4)\\ 
    (3)(1)+(0)(3)+(6)(5)&(3)(2)+(0)(0)+(6)(6)&(3)(3)+(0)(6)+(6)(4)\\
    (5)(1)+(6)(3)+(4)(5)&(5)(2)+(6)(0)+(4)(6)&(5)(3)+(6)(6)+(4)(4)\\\end{pmatrix}\]\[= \begin{pmatrix}1+6+15&2+0+18&3+12+12\\ 
    3+0+30&6+0+36&9+0+24\\
    5+18+20&10+0+24&15+36+16\\\end{pmatrix}=\begin{pmatrix}22&20&27\\ 33&42&33\\ 43&34&67\end{pmatrix}\]
    \[D^3=D\cdot D^2=\begin{pmatrix}1&2&3\\ 3&0&6\\ 5&6&4\end{pmatrix}\begin{pmatrix}22&20&27\\ 33&42&33\\ 43&34&67\end{pmatrix}\]\[= \begin{pmatrix}(1)(22)+(2)(33)+(3)(43)&(1)(20)+(2)(42)+(3)(34)&(1)(27)+(2)(33)+(3)(67)\\ 
    (3)(22)+(0)(33)+(6)(43)&(3)(20)+(0)(42)+(6)(34)&(3)(27)+(0)(33)+(6)(67)\\
    (5)(22)+(6)(33)+(4)(43)&(5)(20)+(6)(42)+(4)(34)&(5)(27)+(6)(33)+(4)(67)\\\end{pmatrix}\]\[=\begin{pmatrix}22+66+129&20+84+102&27+66+201\\66+0+258&60+0+204&81+0+402\\110+198+172&100+252+136&135+198+268\end{pmatrix}=\begin{pmatrix}217&206&294\\ 324&264&483\\ 480&488&601\end{pmatrix}\]
    Sustituyendo $D=\lambda$ en $-D^3+5D^2+53D+54I$
    $$=-\begin{pmatrix}217&206&294\\ 324&264&483\\ 480&488&601\end{pmatrix}+5\begin{pmatrix}1&2&3\\ 3&0&6\\ 5&6&4\end{pmatrix}^2+53\begin{pmatrix}1&2&3\\ 
    3&0&6\\ 5&6&4\end{pmatrix}+54I$$
    $$=-\begin{pmatrix}217&206&294\\ 324&264&483\\ 480&488&601\end{pmatrix}+5\begin{pmatrix}22&20&27\\ 33&42&33\\ 43&34&67\end{pmatrix}+53\begin{pmatrix}1&2&3\\ 3&0&6\\ 5&6&4\end{pmatrix}+54I$$
    $$=\begin{pmatrix}217&206&294\\ 324&264&483\\ 480&488&601\end{pmatrix}+\begin{pmatrix}110&100&135\\ 165&210&165\\ 215&170&335\end{pmatrix}+\begin{pmatrix}53&106&159\\ 159&0&318\\ 265&318&212\end{pmatrix}+54I$$
    $$=\begin{pmatrix}-107&-106&-159\\ -159&-54&-318\\ -265&-318&-266\end{pmatrix}+\begin{pmatrix}53&106&159\\ 159&0&318\\ 265&318&212\end{pmatrix}+54I$$
    $$=\begin{pmatrix}-54&0&0\\ 0&-54&0\\ 0&0&-54\end{pmatrix}+54I$$
    $$-54I+54I=0$$
    Por tanto, se cumple el teorema
    
    \item $E=\begin{pmatrix}1&6&-2\\ -3&2&0\\
    0&3&-4\end{pmatrix}$
    Calculando epor cofactores el determinante de $E-\lambda I$ sobre el primer rengl\'on tenemos:
    $$\begin{vmatrix}1-\lambda&6&-2\\ -3&2-\lambda&0\\ 0&3&-4-\lambda\end{vmatrix}$$
    $$=\left(1-\lambda\right)\begin{vmatrix}2-\lambda&0\\ 3&-4-\lambda\end{vmatrix}-6\begin{vmatrix}-3&0\\ 0&-4-\lambda\end{vmatrix}-2\begin{vmatrix}-3&2-\lambda\\ 0&3\end{vmatrix}$$
    $$=\left(1-\lambda\right)\left(2-\lambda\right)\left(-4-\lambda\right)-0\cdot \:3 -6(\left(-3\right)\left(-4-\lambda\right)-0\cdot \:0)-2(\left(-3\right)\cdot \:3-\left(2-\lambda\right)\cdot \:0)$$
    $$=\left(1-\lambda\right)\left(\lambda^2+2\lambda-8\right)-6\left(-3\left(-\lambda-4\right)\right)-2\left(-9\right)$$
    $$=-\lambda^3-\lambda^2-8\lambda-62$$
    Verificando, primero calculamos $E^2$ y $E^3$:
    \[E^2 =\begin{pmatrix}1&6&-2\\ -3&2&0\\ 0&3&-4\end{pmatrix}\begin{pmatrix}1&6&-2\\ -3&2&0\\ 0&3&-4\end{pmatrix}\]\[= \begin{pmatrix}(1)(1)+(6)(-3)+(-2)(0)&(1)(6)+(6)(2)+(-2)(3)&(1)(-2)+(6)(0)+(-2)(-4)\\ 
    (-3)(1)+(2)(-3)+(0)(0)&(-3)(6)+(2)(2)+(0)(3)&(-3)(-2)+(2)(0)+(0)(-4)\\
    (0)(1)+(3)(-3)+(-4)(0)&(0)(6)+(3)(2)+(-4)(3)&(0)(-2)+(3)(0)+(-4)(-4)\\ \end{pmatrix}\]\[= \begin{pmatrix}1-18+0&6+12-6&-2+0+8\\ 
    -3-6+0&-18+4+0&6+0+0\\
    0-9+0&0+6-12&0+0-16\\\end{pmatrix}=\begin{pmatrix}-17&12&6\\ -9&-14&6\\ -9&-6&16\end{pmatrix}\]
    \[E^3=E\cdot E^2=\begin{pmatrix}1&6&-2\\ -3&2&0\\ 0&3&-4\end{pmatrix}\begin{pmatrix}-17&12&6\\ -9&-14&6\\ -9&-6&16\end{pmatrix}\]\[= \begin{pmatrix}(1)(-17)+(6)(-9)+(-2)(-9)&(1)(12)+(6)(-14)+(-2)(-6)&(1)(6)+(6)(6)+(-2)(16)\\ 
    (-3)(-17)+(2)(-9)+(0)(-9)&(-3)(12)+(2)(-14)+(0)(-6)&(-3)(6)+(2)(6)+(0)(16)\\
    (0)(-17)+(3)(-9)+(-4)(-9)&(0)(12)+(3)(-14)+(-4)(-6)&(0)(6)+(3)(6)+(-4)(16)\\ \end{pmatrix}\]\[= \begin{pmatrix}-17-54+18&12-84+12&6+36-32\\ 
    51-18+0&-36-28+0&-18+12+0\\
    0-27+36&0-42+24&0+18-64\\\end{pmatrix}=\begin{pmatrix}-53&-60&10\\ 33&-64&-6\\ 9&-18&-46\end{pmatrix}\]
    
    Sustituyendo $E=\lambda$ en $-E^3-E^2-8E-62I$
    $$=-\begin{pmatrix}1&6&-2\\ \:\:\:-3&2&0\\ \:\:\:0&3&-4\end{pmatrix}^3-\begin{pmatrix}1&6&-2\\ \:\:\:-3&2&0\\ \:\:\:0&3&-4\end{pmatrix}^2-8\begin{pmatrix}1&6&-2\\ \:\:\:-3&2&0\\ \:\:\:0&3&-4\end{pmatrix}-62I$$
    $$=-\begin{pmatrix}-53&-60&10\\ 33&-64&-6\\ 9&-18&-46\end{pmatrix}-\begin{pmatrix}1&6&-2\\ -3&2&0\\ 0&3&-4\end{pmatrix}^2-8\begin{pmatrix}1&6&-2\\ -3&2&0\\ 0&3&-4\end{pmatrix}-62I$$
    $$=\begin{pmatrix}-53&-60&10\\ 33&-64&-6\\ 9&-18&-46\end{pmatrix}-\begin{pmatrix}-17&12&6\\ -9&-14&6\\ -9&-6&16\end{pmatrix}-8\begin{pmatrix}1&6&-2\\ -3&2&0\\ 0&3&-4\end{pmatrix}-62I$$
    $$=-\begin{pmatrix}-53&-60&10\\ 33&-64&-6\\ 9&-18&-46\end{pmatrix}-\begin{pmatrix}-17&12&6\\ -9&-14&6\\ -9&-6&16\end{pmatrix}-\begin{pmatrix}8&48&-16\\ -24&16&0\\ 0&24&-32\end{pmatrix}-62I$$
    $$=\begin{pmatrix}53&60&-10\\ -33&64&6\\ -9&18&46\end{pmatrix}-\begin{pmatrix}-17&12&6\\ -9&-14&6\\ -9&-6&16\end{pmatrix}-\begin{pmatrix}8&48&-16\\ -24&16&0\\ 0&24&-32\end{pmatrix}-62I$$
    $$=\begin{pmatrix}70&48&-16\\ -24&78&0\\ 0&24&30\end{pmatrix}-\begin{pmatrix}8&48&-16\\ -24&16&0\\ 0&24&-32\end{pmatrix}-62I$$
    $$=\begin{pmatrix}62&0&0\\ 0&62&0\\ 0&0&62\end{pmatrix}-62I$$
    $$=62I-62I=0$$
    Por tanto, se cumple el teorema 

\end{itemize}{}
\section{Sea $T:V\longrightarrow W$ lineal. Demuestra que si $U,W \leq V$
son invariantes bajo $T$, entonces también lo son $U+W$, $U\cap W$.}

\begin{enumerate}

\item \textbf{ Demostración 3.a}\\ Sea $U+W =Z$. Por definición, el conjunto $Z=\{z\in V : z= u+w$   con $u\in U$ y $w\in W\}$\\ P.D. $T(z)\in Z$, $\forall z\in Z$.\\
Sea $z\in Z$ arbitario, por definición tenemos que $z= u+w$ (con $u\in U$ y $w\in W$). Aplicando $T$ sobre $z$ y tomando en cuenta que $T$ es lineal, es decir "separa la suma", tenemos que:
\[T(z) = T(u+w) = T(u)+ T(w)\] como $U$ y $W$ son subespacios invariantes bajo $T$ entonces, tenemos que por definici\'on de contenci\'on: \[T(u) \in U~~~\text{y}~~~  T(w)\in W\]así por definici\'on, tenemos que: \[ T(z)=T(u)+T(w) \in U + W = Z\] \[\therefore T(z) \in Z\]  Y como $z$ era arbitrario, se cumple para toda $z\in Z$, por lo tanto $Z=U+W$ es un subespacio invariante bajo $T$.\qed

\item\textbf{ Demostración 3.b}
\\ Sea $U\cap W =X$. Por definición, el conjunto $X=\{x\in V :$   con $x\in U$ y $x\in W\}$\\ 

P.D. $T(x)\in X$, $\forall x\in X$.\\

Sea $x\in X$ arbitario, por definición tenemos que $x\in U$ y $x\in W$. Aplicando $T$ sobre $x$ y por definici\'on de $U$ y $W$ invariantes, tenemos que: \[T(x) \in U~~~\text{y}~~~  T(x)\in W\]
Es decir $T(x)\in U\cap W =X $
\[\therefore T(x)\in X\] Y como $x$ era arbitrario, se cumple para toda $x\in X$, por lo tanto $X=U+W$ es un subespacio invariante bajo $T$.\qed
\end{enumerate}
\section{Para las siguientes matrices:}
\[a)~ A = \begin{pmatrix}
3 &-4\\
2 &-6
\end{pmatrix}~~ , ~b)~B =\begin{pmatrix}
2&2\\
1 &3\end{pmatrix}
\]\textbf{Soluci\'on 4:}\\
\begin{itemize}
    \item[$i)$] Encontrar todos los valores y vectores propios correspondientes.
    Sabemos que para encontrar un valor propio $\lambda_n$ de una matriz $A\in\mathcal{M}_{n\times n }$ que satisfaga $A\vec{v}=\lambda_n\vec{v}$, debemos tener que $\text{det}(A-\lambda_nI_{n\times n })=0$, de esta manera obtenemos su polinomio caracter\'istico, cuyas $n$ ra\'ices son sus valores propios.
    \begin{enumerate}
        \item[$a)$] Lo primero que se deber\'a hacer es encontrar su polinomio caracter\'istico:
        \[0=\text{det}(A-\lambda_nI_{n\times n })=\left|\begin{pmatrix}
3 &-4\\
2 &-6
\end{pmatrix}-\lambda\begin{pmatrix}
1 &0\\0 &1
\end{pmatrix}\right|=\begin{vmatrix}3-\lambda &-4\\
2 &-6-\lambda\end{vmatrix}=(3-\lambda)(-6-\lambda)-(2)(-4)\]\[=-18-3\lambda+\lambda^2+8=\lambda^2+3\lambda-10\]
Por lo que su polinomio caracter\'istico es $\lambda^2+3\lambda-10=(\lambda+5)(\lambda-2)=0$, de modo que sus ra\'ices (valores propios) son $\lambda_1=-5$ y $\lambda_2=2$. Ahora lo que haremos ser\'a calcular los vectores propios de cada uno, para lo cual debemos resolver la ecuaci\'on $(A-\lambda_nI_{n\times n })\vec{v}=\vec{0}$ (con $\vec{v}=(x,y)$):
\begin{itemize}
    \item Para $\lambda_1=-5$, sustituyendo, tenemos que:
    \[\begin{pmatrix}0\\
0\end{pmatrix}=\vec{0}=(A-\lambda_1I_{n\times n })\vec{v}=\begin{pmatrix}3-(-5) &-4\\
2 &-6-(-5)\end{pmatrix}\begin{pmatrix}x\\
y\end{pmatrix}=\begin{pmatrix}8 &-4\\
2 &-1\end{pmatrix}\begin{pmatrix}x\\
y\end{pmatrix}=\begin{pmatrix}8x-4y\\
2x-y\end{pmatrix}\]
\[\therefore \begin{pmatrix}0\\
0\end{pmatrix}=\begin{pmatrix}8x-4y\\
2x-y\end{pmatrix}\]
Por lo que tenemos el siguiente sistema de ecuaciones:
\begin{eqnarray*}
8x-4y&=&0\\
2x-y&=&0
\end{eqnarray*}
Pero si nos damos cuenta la primera ecuaci\'on es m\'ultiplo de la segunda, pues si dividimos los coeficientes de las $x$ entre los coeficientes de las $y$ nos queda $\displaystyle\frac{8}{-4}=\frac{2}{-1}=-2$.\\
Si despejamos de la segunda ecuaci\'on tenemos que $y=2x$, por lo que si damos a $x\in\mathbb{R}$ como valor fijo tenemos un vector $(x,y)=(x,2x)=x(1,2)$.\\
S.P.G., damos el valor de $x=1$, por lo que el valor propio $\lambda_1=-5$ tiene asociado el vector propio $\vec{v_1}=(1,2)$.

\item Para $\lambda_2=2$, sustituyendo, tenemos que:
    \[\begin{pmatrix}0\\
0\end{pmatrix}=\vec{0}=(A-\lambda_2I_{n\times n })\vec{v}=\begin{pmatrix}3-2 &-4\\
2 &-6-2\end{pmatrix}\begin{pmatrix}x\\
y\end{pmatrix}=\begin{pmatrix}1 &-4\\
2 &-8\end{pmatrix}\begin{pmatrix}x\\
y\end{pmatrix}=\begin{pmatrix}x-4y\\
2x-8y\end{pmatrix}\]
\[\therefore \begin{pmatrix}0\\
0\end{pmatrix}=\begin{pmatrix}x-4y\\
2x-8y\end{pmatrix}\]
Por lo que tenemos el siguiente sistema de ecuaciones:
\begin{eqnarray*}
x-4y&=&0\\
2x-8y&=&0
\end{eqnarray*}
Pero si nos damos cuenta la primera ecuaci\'on es m\'ultiplo de la segunda, pues si dividimos los coeficientes de las $x$ entre los coeficientes de las $y$ nos queda $\displaystyle\frac{1}{-4}=\frac{2}{-8}$.\\
Si despejamos de la primera ecuaci\'on tenemos que $x=4y$, por lo que si damos a $y\in\mathbb{R}$ como valor fijo tenemos un vector $(x,y)=(4y,y)=y(4,1)$.\\
S.P.G., damos el valor de $y=1$, por lo que el valor propio $\lambda_2=2$ tiene asociado el vector propio $\vec{v_2}=(4,1)$.

    
\end{itemize}
        \item[$b)$] Calculando su polinomio caracter\'istico:
        \[0=\text{det}(B-\lambda_nI_{n\times n })=\left|\begin{pmatrix}
2&2\\
1 &3
\end{pmatrix}-\lambda\begin{pmatrix}
1 &0\\0 &1
\end{pmatrix}\right|=\begin{vmatrix}2-\lambda &2\\
1 &3-\lambda\end{vmatrix}=(2-\lambda)(3-\lambda)-(1)(2)\]\[=6-5\lambda+\lambda^2-2=\lambda^2-5\lambda+4\]
Por lo que su polinomio caracter\'istico es $\lambda^2-5\lambda+4=(\lambda-4)(\lambda-1)=0$, de modo que sus ra\'ices (valores propios) son $\lambda_1=4$ y $\lambda_2=1$. Ahora lo que haremos ser\'a calcular los vectores propios de cada uno, para lo cual debemos resolver la ecuaci\'on $(B-\lambda_nI_{n\times n })\vec{v}=\vec{0}$ (con $\vec{v}=(x,y)$):
\begin{itemize}
    \item Para $\lambda_1=4$, sustituyendo, tenemos que:
    \[\begin{pmatrix}0\\
0\end{pmatrix}=\vec{0}=(B-\lambda_1I_{n\times n })\vec{v}=\begin{pmatrix}2-4 &2\\
1 &3-4\end{pmatrix}\begin{pmatrix}x\\
y\end{pmatrix}=\begin{pmatrix}-2 &2\\
1 &-1\end{pmatrix}\begin{pmatrix}x\\
y\end{pmatrix}=\begin{pmatrix}-2x+2y\\
x-y\end{pmatrix}\]
\[\therefore \begin{pmatrix}0\\
0\end{pmatrix}=\begin{pmatrix}-2x+2y\\
x-y\end{pmatrix}\]
Por lo que tenemos el siguiente sistema de ecuaciones:
\begin{eqnarray*}
-2x+2y&=&0\\
x-y&=&0
\end{eqnarray*}
Pero si nos damos cuenta la primera ecuaci\'on es m\'ultiplo de la segunda, pues si dividimos los coeficientes de las $x$ entre los coeficientes de las $y$ nos queda $\displaystyle\frac{-2}{2}=\frac{1}{-1}=-1$.\\
Si despejamos de la segunda ecuaci\'on tenemos que $x=y$, por lo que si damos a $x\in\mathbb{R}$ como valor fijo tenemos un vector $(x,y)=(x,x)=x(1,1)$.\\
S.P.G., damos el valor de $x=1$, por lo que el valor propio $\lambda_1=4$ tiene asociado el vector propio $\vec{v_1}=(1,1)$.

\item Para $\lambda_2=1$, sustituyendo, tenemos que:
    \[\begin{pmatrix}0\\
0\end{pmatrix}=\vec{0}=(B-\lambda_1I_{n\times n })\vec{v}=\begin{pmatrix}2-1 &2\\
1 &3-1\end{pmatrix}\begin{pmatrix}x\\
y\end{pmatrix}=\begin{pmatrix}1 &2\\
1 &2\end{pmatrix}\begin{pmatrix}x\\
y\end{pmatrix}=\begin{pmatrix}x+2y\\
x+2y\end{pmatrix}\]
\[\therefore \begin{pmatrix}0\\
0\end{pmatrix}=\begin{pmatrix}x+2y\\
x+2y\end{pmatrix}\]
Por lo que tenemos el siguiente sistema de ecuaciones:
\begin{eqnarray*}
x+2y&=&0\\
x+2y&=&0
\end{eqnarray*}
Pero si nos damos cuenta la primera ecuaci\'on es igual a segunda, por lo que si despejamos de la primera ecuaci\'on tenemos que $x=-2y$, por lo que si damos a $y\in\mathbb{R}$ como valor fijo tenemos un vector $(x,y)=(-2y,y)=y(-2,1)$.\\
S.P.G., damos el valor de $y=1$, por lo que el valor propio $\lambda_2=1$ tiene asociado el vector propio $\vec{v_2}=(-2,1)$.

    
\end{itemize}
    \end{enumerate}
    \item[$ii)$] Encontrar matrices $P,D$ con $P$ no singular y de manera que $D = P^{-1}AP$ es diagonal.\\
    Para encontrar una matriz diagonal $D$ y $P$ que cumpla $D = P^{-1}AP$, usamos el \textbf{Teorema 14} visto en clase en la cual una base $\beta$ est\'a formada por vectores propios de una transformaci\'on $T$, la cual est\'a asociada a los valores propios $\lambda_1\,dots,\lambda_n$, entonces la matriz de $T$ respecto de la base $\beta$ es la matriz diagonal $D$, mientras que $P$ y $P^{-1}$ son las matrices de cambio de base $\beta$ a la base can\'onica y viceversa, de modo que sabemos que los valores propios son las columnas de $P$, mientras que los valores propios son los elementos en la diagonal de $D$:
    \[P=\begin{pmatrix}\vec{v_1}&\cdots&\vec{v_n}\end{pmatrix}~~~~~\text{y}~~~~~~~~D=\begin{pmatrix}\lambda_1&\cdots&0\\ \vdots&\ddots&\vdots\\0&\cdots&\lambda_n\end{pmatrix}\]
    Entonces teniendo lo del inciso anterior, tenemos que:
    \begin{itemize}
        \item Sabemos que para $\displaystyle A= \begin{pmatrix}
3 &-4\\
2 &-6
\end{pmatrix}$ se tienen los valores propios $\lambda_1=-5$ y $\lambda_2=2$ y los vectores propios $\vec{v_1}=(1,2)$ y $\vec{v_2}=(4,1)$, respectivamente, por lo que nuestras matrices quedan:
\[P=\begin{pmatrix}\vec{v_1}&\vec{v_2}\end{pmatrix}=\begin{pmatrix}1&4\\2&1\end{pmatrix}~~~~~\text{y}~~~~~~~~D=\begin{pmatrix}\lambda_1&0\\0&\lambda_2\end{pmatrix}=\begin{pmatrix}-5&0\\0&2\end{pmatrix}\]
Entonces usando la f\'ormula $\displaystyle P=\begin{pmatrix}a&b\\c&d\end{pmatrix}~~\Longrightarrow~~P^{-1}=\frac{1}{ad-bc}\begin{pmatrix}d&-b\\-c&a\end{pmatrix}$, por tanto:
\[P^{-1}=\frac{1}{(1)(1)-(2)(4)}\begin{pmatrix}1&-4\\-2&1\end{pmatrix}=\begin{pmatrix}-\frac{1}{7}&\frac{4}{7}\\\frac{2}{7}&-\frac{1}{7}\end{pmatrix}\]
Por lo que solo falta verificar que $D = P^{-1}AP$, entonces:
\[P^{-1}AP=\begin{pmatrix}-\frac{1}{7}&\frac{4}{7}\\\frac{2}{7}&-\frac{1}{7}\end{pmatrix}\begin{pmatrix}
3 &-4\\
2 &-6
\end{pmatrix}\begin{pmatrix}1&4\\2&1\end{pmatrix}=\begin{pmatrix}-\frac{1}{7}&\frac{4}{7}\\\frac{2}{7}&-\frac{1}{7}\end{pmatrix}\begin{pmatrix}
(3)(1)+(-4)(2) &(3)(4)+(-4)(1)\\
(2)(1)+(-6)(2) &(2)(4)+(-6)(1)\\
\end{pmatrix}\]\[\begin{pmatrix}-\frac{1}{7}&\frac{4}{7}\\\frac{2}{7}&-\frac{1}{7}\end{pmatrix}\begin{pmatrix}
-5 &8\\
-10 &2\\
\end{pmatrix}=\begin{pmatrix}\left(-\frac{1}{7}\right)(-5)+\left(\frac{4}{7}\right)(-10)&\left(-\frac{1}{7}\right)(8)+\left(\frac{4}{7}\right)(2)\\\left(\frac{2}{7}\right)(-5)+\left(-\frac{1}{7}\right)(-10)&\left(\frac{2}{7}\right)(8)+\left(-\frac{1}{7}\right)(2)\end{pmatrix}=\begin{pmatrix}\left(\frac{1}{7}\right)(5-40)&\left(\frac{1}{7}\right)(-8+8)\\\left(\frac{1}{7}\right)(-10+10)&\left(\frac{1}{7}\right)(16-2)\end{pmatrix}\]\[=\begin{pmatrix}\frac{-35}{7}&\frac{0}{7}\\\frac{0}{7}&\frac{14}{7}\end{pmatrix}=\begin{pmatrix}-5&0\\0&2\end{pmatrix}=D\]
\[\therefore P^{-1}AP=D\]


\item Sabemos que para $\displaystyle B= \begin{pmatrix}
2 &2\\
1 &3
\end{pmatrix}$ se tienen los valores propios $\lambda_1=4$ y $\lambda_2=1$ y los vectores propios $\vec{v_1}=(1,1)$ y $\vec{v_2}=(-2,1)$, respectivamente, por lo que nuestras matrices quedan:
\[P=\begin{pmatrix}\vec{v_1}&\vec{v_2}\end{pmatrix}=\begin{pmatrix}1&-2\\1&1\end{pmatrix}~~~~~\text{y}~~~~~~~~D=\begin{pmatrix}\lambda_1&0\\0&\lambda_2\end{pmatrix}=\begin{pmatrix}4&0\\0&1\end{pmatrix}\]
Entonces usando la f\'ormula $\displaystyle P=\begin{pmatrix}a&b\\c&d\end{pmatrix}~~\Longrightarrow~~P^{-1}=\frac{1}{ad-bc}\begin{pmatrix}d&-b\\-c&a\end{pmatrix}$, por tanto:
\[P^{-1}=\frac{1}{(1)(1)-(-2)(1)}\begin{pmatrix}1&2\\-1&1\end{pmatrix}=\begin{pmatrix}\frac{1}{3}&\frac{2}{3}\\-\frac{1}{3}&\frac{1}{3}\end{pmatrix}\]
Por lo que solo falta verificar que $D = P^{-1}AP$, entonces:
\[P^{-1}BP=\begin{pmatrix}\frac{1}{3}&-\frac{2}{3}\\-\frac{1}{3}&\frac{1}{3}\end{pmatrix}\begin{pmatrix}
2 &2\\
1 &3
\end{pmatrix}\begin{pmatrix}1&-2\\1&1\end{pmatrix}=\begin{pmatrix}\frac{1}{3}&\frac{2}{3}\\-\frac{1}{3}&\frac{1}{3}\end{pmatrix}\begin{pmatrix}
(2)(1)+(2)(1) &(2)(-2)+(2)(1)\\
(1)(1)+(3)(1) &(1)(-2)+(3)(1)
\end{pmatrix}\]\[=\begin{pmatrix}\frac{1}{3}&\frac{2}{3}\\-\frac{1}{3}&\frac{1}{3}\end{pmatrix}\begin{pmatrix}
4 &-2\\
4 &1
\end{pmatrix}=\begin{pmatrix}\left(\frac{1}{3}\right)(4)+\left(\frac{2}{3}\right)(4)&\left(\frac{1}{3}\right)(-2)+\left(\frac{2}{3}\right)(1)\\\left(-\frac{1}{3}\right)(4)+\left(\frac{1}{3}\right)(4)&\left(-\frac{1}{3}\right)(-2)+\left(\frac{1}{3}\right)(1)\end{pmatrix}=\begin{pmatrix}\left(\frac{1}{3}\right)(4+8)&\left(\frac{1}{3}\right)(-2+2)\\\left(\frac{1}{3}\right)(-4+4)&\left(\frac{1}{3}\right)(2+1)\end{pmatrix}\]\[=\begin{pmatrix}
4 &0\\
0 &1
\end{pmatrix}=D\]
\[\therefore P^{-1}BP=D\]

    \end{itemize}
    
    
\end{itemize}
\section{Sean $V = \mathcal{M}_2(\mathbb{R})$, $T:V\longrightarrow V$, dado por $T(X) = X^T$, su transpuesta.}
\begin{itemize}
    \item [$a)$] Prueba que $T$ es lineal.\\\\
    \textbf{Demostraci\'on 5.a:}\\
    Como $\mathcal{M}_2(\mathbb{R})$ ambos es un $\mathbb{R}$-espacio vectorial, para demostrar que $T$ es una transformaci\'on lineal, debemos probar que:
\begin{itemize}
    \item $T(A+B) = T(A) + T(B)$ para cada $A,B \in \mathcal{M}_2(\mathbb{R})$ . ($T$ es aditiva).\\
    Sean $A$ y $B$ matrices de $\mathcal{M}_2(\mathbb{R})$, primero calculamos \[T(A)=A^T~~~~\text{y}~~~~T(B)=B^T\]
    Por tanto:
    \[T(A)+T(B)=A^T+B^T\]
    Ahora calculamos:
    \[T(A+B)=(A+B)^T\]
    Por lo que por las mismas propiedades de la transpuesta, sabemos que se abre a sumas, por tanto:
    \[T(A)+T(B)=A^T+B^T=(A+B)^T=T(A+B)\]
    \[\therefore T(A+B) = T(A) + T(B)\]
    Por lo que $T$ es aditiva.
    
 \item  $T(\alpha A) = \alpha T(A)$ para cada $A \in \mathcal{M}_2(\mathbb{R})$ y $\alpha \in \mathbb{R}$. ($T$ es $\mathbb{R}$-lineal).\\
 Sean $A\in \mathcal{M}_2(\mathbb{R})$ una matriz arbitaria y $\alpha\in \mathbb{R}$ un escalar arbitrario, primero calculamos \[\alpha T(A)=\alpha (A)^T\]
    Ahora calculamos:
    \[T(\alpha A)=(\alpha A)^T\]
    Pero por la propiedad de la transpuesta, sabemos que podemos factorizar el escalar de una tranpuesta, de modo que:
    \[T(\alpha A) =(\alpha A)^T=\alpha(A)^T=\alpha T(A)\]
    \[\therefore T(\alpha A) =\alpha T(A)\]
    Por lo que $T$ es $\mathbb{R}-$lineal.

\end{itemize}
Como de ambos puntos, $T: \mathcal{M}_2(\mathbb{R}) \rightarrow \mathcal{M}_2(\mathbb{R})$ definida como $T(A)=A^T$ cumple ambos puntos, entonces $T$ es una transformaci\'on lineal, que adem\'as es un endomorfismo.\qed
    
    \item [$b)$] Calcula la matriz $A$ de $T$ relativa a la base canónica de $V$.\\\\
    \textbf{Soluci\'on 5.b:}\\
    Sabemos que para calcular la matriz $A$ asociada a $T$, lo primero que debemos hacer es plantear una base can\'onica para nuestro espacio vectorial $\mathcal{M}_2(\mathbb{R})$, la cual podemos definir de acuerdo a las matrices can\'onicas como:
    \[\mathcal{B}=\left\{E_{1\hspace{0.5mm}1}=e_1=\begin{pmatrix}
    1 & 0\\
    0 & 0
    \end{pmatrix}, E_{1\hspace{0.5mm}2}=e_2=\begin{pmatrix}
    0 & 1\\
    0 & 0
    \end{pmatrix}, E_{2\hspace{0.5mm}1}=e_3=\begin{pmatrix}
    0 & 0\\
    1 & 0
    \end{pmatrix}, E_{2\hspace{0.5mm}2}=e_4=\begin{pmatrix}
    0 & 0\\
    0 & 1
    \end{pmatrix}\right\}\]
    Ahora, sabemos que para determinar $A$ debemos definir cada columna como el vector coordenada de la matriz aplicada a cada elemento de la base (con la numeraci\'on correspodiente), respecto a la base $\mathcal{B}$, es decir:
    \[A=\begin{pmatrix} [T(e_1)]_{\mathcal{B}} & [T(e_2)]_{\mathcal{B}} &[T(e_3)]_{\mathcal{B}} &[T(e_4)]_{\mathcal{B}} \end{pmatrix}\]
    Por lo tanto, encontrando cada vector coordenada, tenemos:
    \begin{itemize}
        \item \[T(e_1)=\begin{pmatrix}
    1 & 0\\
    0 & 0
    \end{pmatrix}^T=\begin{pmatrix}
    1 & 0\\
    0 & 0
    \end{pmatrix}=e_1=1\cdot e_1+0\cdot e_2+0\cdot e_3+0\cdot e_4\]
    \[\therefore [T(e_1)]_{\mathcal{B}}=\begin{pmatrix} 1 & 0 & 0 & 0\end{pmatrix}\]
    
    \item \[T(e_2)=\begin{pmatrix}
    0 & 1\\
    0 & 0
    \end{pmatrix}^T=\begin{pmatrix}
    0 & 0\\
    1 & 0
    \end{pmatrix}=e_3=0\cdot e_1+0\cdot e_2+1\cdot e_3+0\cdot e_4\]
    \[\therefore [T(e_2)]_{\mathcal{B}}=\begin{pmatrix} 0 & 0 & 1 & 0\end{pmatrix}\]
    
    \item \[T(e_3)=\begin{pmatrix}
    0 & 0\\
    1 & 0
    \end{pmatrix}^T=\begin{pmatrix}
    0 & 1\\
    0 & 0
    \end{pmatrix}=e_2=0\cdot e_1+1\cdot e_2+0\cdot e_3+0\cdot e_4\]
    \[\therefore [T(e_3)]_{\mathcal{B}}=\begin{pmatrix} 0 & 1 & 0 & 0\end{pmatrix}\]
    
    \item \[T(e_4)=\begin{pmatrix}
    0 & 0\\
    0 & 1
    \end{pmatrix}^T=\begin{pmatrix}
    0 & 0\\
    0 & 1
    \end{pmatrix}=e_4=0\cdot e_1+0\cdot e_2+0\cdot e_3+1\cdot e_4\]
    \[\therefore [T(e_1)]_{\mathcal{B}}=\begin{pmatrix} 0 & 0 & 0 & 1\end{pmatrix}\]
    \end{itemize}
    De modo que finalmente coloc\'andolos como columna en la matriz obtenemos:
    \[A= \begin{pmatrix} [T(e_1)]_{\mathcal{B}} & [T(e_2)]_{\mathcal{B}} &[T(e_3)]_{\mathcal{B}} &[T(e_4)]_{\mathcal{B}} \end{pmatrix}=\begin{pmatrix}
    1 & 0 & 0 & 0\\
    0 & 0 & 1 & 0\\
    0 & 1 & 0 & 0\\
    0 & 0 & 0 & 1\end{pmatrix}\]
    
    \item [$c)$] Calcula los valores propios de $T$, los espacios propios, sus dimensiones y una base de cada uno.\\\\
    \textbf{Soluci\'on 5.c:}\\
    Lo primero que se deber\'a hacer es encontrar su polinomio caracter\'istico:
        \[0=\text{det}(A-\lambda_nI_{n\times n })=\left|\begin{pmatrix}
    1 & 0 & 0 & 0\\
    0 & 0 & 1 & 0\\
    0 & 1 & 0 & 0\\
    0 & 0 & 0 & 1\end{pmatrix}-\lambda\begin{pmatrix}
    1 & 0 & 0 & 0\\
    0 & 1 & 0 & 0\\
    0 & 0 & 1 & 0\\
    0 & 0 & 0 & 1\end{pmatrix}\right|=\begin{vmatrix}
    1-\lambda & 0 & 0 & 0\\
    0 & -\lambda & 1 & 0\\
    0 & 1 & -\lambda & 0\\
    0 & 0 & 0 & 1-\lambda\end{vmatrix}\] Como sabemos que una matriz equivalente por filas tiene el mismo determinante, si asumimos que $\lambda \neq 0$, podemos sumar $\displaystyle \frac{1}{\lambda}$ veces el segundo rengl\'on al tercero, de modo que tengamos una matriz escalonada y asi multiplicar la diagonal:
    \[=\begin{vmatrix}
    1-\lambda & 0 & 0 & 0\\
    0 & -\lambda & 1 & 0\\
    0 & 1+\frac{1}{\lambda}(-\lambda) & -\lambda+\frac{1}{\lambda}(1) & 0\\
    0 & 0 & 0 & 1-\lambda\end{vmatrix}=\begin{vmatrix}
    1-\lambda & 0 & 0 & 0\\
    0 & -\lambda & 1 & 0\\
    0 & 0 & -\lambda+\frac{1}{\lambda} & 0\\
    0 & 0 & 0 & 1-\lambda\end{vmatrix}=(1-\lambda)(-\lambda)\left(-\lambda+\frac{1}{\lambda}\right)(1-\lambda)\]\[=(1-\lambda)^2\left(+\lambda^2+\frac{-\lambda}{\lambda}\right)=(\lambda-1)^2(\lambda^2-1)=(\lambda-1)(\lambda-1)(\lambda-1)(\lambda+1)\]
    \[\therefore (\lambda-1)(\lambda-1)(\lambda-1)(\lambda+1)=0\]
    
    Por lo que tenemos que $\lambda_1=-1$ y $\lambda_2=\lambda_3=\lambda_4=1$ (multiplicidad de 3), por lo que tenemos que $\text{ma}(-1)=1$ y $\text{ma}(1)=3$.\\
    Ahora, usando la \textbf{Definici\'on 20} vista en clase, como $T$ es un endomorfismo y tenemos su matriz asociada $A$ respecto de alguna base can\'onica, calculamos cada uno de los subespacios propios correspondientes a $\lambda_i$ (subconjunto que consta de los vectores
propios), con $i=1,2,3,4$:
\begin{itemize}
    \item Para $\lambda_1=-1$, sabemos que $E(1)=\text{Nuc}(A-(-1)I)$, por lo que sustituyendo en la ecuaci\'on $(A-(-1)I)\vec{x}=(0)$, con $x=(a,b,c,d)$, tenemos:
    \[(A-(-1)I)\vec{x}=\begin{pmatrix}
    1-(-1) & 0 & 0 & 0\\
    0 & -(-1) & 1 & 0\\
    0 & 1 & -(-1) & 0\\
    0 & 0 & 0 & 1-(-1)\end{pmatrix}\begin{pmatrix}a\\b\\c\\d\end{pmatrix}=\begin{pmatrix}
    2 & 0 & 0 & 0\\
    0 & 1 & 1 & 0\\
    0 & 1 & 1 & 0\\
    0 & 0 & 0 & 2\end{pmatrix}\begin{pmatrix}a\\b\\c\\d\end{pmatrix}\]\[=\begin{pmatrix}2(a)+0(b)+0(c)+0(d)\\0(a)+1(b)+1(c)+0(d)\\0(a)+1(b)+1(c)+0(d)\\0(a)+0(b)+0(c)+2(d)\end{pmatrix}=\begin{pmatrix}2a\\b+c\\b+c\\2d\end{pmatrix}=\begin{pmatrix}0\\0\\0\\0\end{pmatrix}\]
    Por lo que tenemos el siguiente sistema de ecuaciones:
\begin{eqnarray*}
2a&=&0\\
b+c&=&0\\
b+c&=&0\\
2d&=&0
\end{eqnarray*}
De la primera y cuarta ecuaci\'on tenemos que $a=d=0$, mientras que despejando de la segunda y tercera, tenemos que $c=-b$, por lo que tenemos que $b$ es una variable libre, por lo que tenemos que $\begin{pmatrix}0&b&-b&0\end{pmatrix}\in E(-1)$, de tal forma que $E(-1)=\{(0,\alpha,-\alpha,0)~|~\alpha\in\mathbb{R}\}$.
Por tanto, si fijamos $(0,\alpha,-\alpha,0)=\alpha(0,1,-1,0)$, tenemos que $\langle(0,1,-1,0)\rangle=E(-1)$ y como es un \'unico vector es linealmente independiente, por lo que $\text{dim}(E(-1))=1$.


\item Para $\lambda_2=\lambda_3=\lambda_4=1$ (con multiplicidad algebraica 3), sabemos que $E(1)=\text{Nuc}(A-1I)$, por lo que sustituyendo en la ecuaci\'on $(A-1I)\vec{x}=(0)$, con $x=(a,b,c,d)$, tenemos:
    \[(A-1I)\vec{x}=\begin{pmatrix}
    1-1 & 0 & 0 & 0\\
    0 & -1 & 1 & 0\\
    0 & 1 & -1 & 0\\
    0 & 0 & 0 & 1-1\end{pmatrix}\begin{pmatrix}a\\b\\c\\d\end{pmatrix}=\begin{pmatrix}
    0 & 0 & 0 & 0\\
    0 & -1 & 1 & 0\\
    0 & 1 & -1 & 0\\
    0 & 0 & 0 & 0\end{pmatrix}\begin{pmatrix}a\\b\\c\\d\end{pmatrix}\]\[=\begin{pmatrix}0(a)+0(b)+0(c)+0(d)\\0(a)+(-1)(b)+1(c)+0(d)\\0(a)+1(b)+(-1)(c)+0(d)\\0(a)+0(b)+0(c)+0(d)\end{pmatrix}=\begin{pmatrix}0\\-b+c\\b-c\\0\end{pmatrix}=\begin{pmatrix}0\\0\\0\\0\end{pmatrix}\]
    Por lo que tenemos el siguiente sistema de ecuaciones:
\begin{eqnarray*}
0&=&0\\
-b+c&=&0\\
b-c&=&0\\
0&=&0
\end{eqnarray*}
Despejando de la segunda y tercera, tenemos que $c=b$, por lo que tenemos que $b$ es una variable libre, pero al tampoco tener restricciones para $a$ y $d$, tam bien las podemos tratar como variables libres, por lo que tenemos que $\begin{pmatrix}a&b&b&d\end{pmatrix}\in E(1)$, de tal forma que $E(1)=\{(\beta,\alpha,\alpha,\gamma)~|~\alpha,\beta,\gamma\in\mathbb{R}\}$.\\
Es f\'acil ver que los vectores $(1,0,0,0)$, $(0,1,1,0)$ y $(0,0,0,1)$ son base de $E(1)$, pues si tomamos un vector arbitrario $(\beta,\alpha,\alpha,\gamma)\in E(1)$, tenemos que:
\[(\beta,\alpha,\alpha,\gamma)=\beta(1,0,0,0)+\alpha(0,1,1,0)+\gamma(0,0,0,1)\]
y adem\'as son linealmente independientes, pues:
\[(\beta,\alpha,\alpha,\gamma)=\beta(1,0,0,0)+\alpha(0,1,1,0)+\gamma(0,0,0,1)=(0,0,0,0)\]
Esto pasa si y solo si $\alpha=\beta=\gamma=0$, entonces es obvio que los tres vectores son base de $E(1)$, por lo que $\text{dim}(E(1))=3$.
\end{itemize}
    
    \item [$d)$] ¿Es $T$ es diagonalizable?\\\\
    \textbf{Respuesta 5.d:}\\
    Usando el \textbf{Teorema 23} visto en clase, tenemos que comprobar los siguientes puntos:
    \begin{itemize}
        \item El polinomio característico de $T$ tiene todas sus raíces en $K=\mathbb{R}$.\\
        Esto es verdad, pues encontramos que $\lambda_1=-1$ y $\lambda_2=1$, por lo que ambas pertenecen a $\mathbb{R}$
\item Para cada valor propio $\lambda$ se verifica $\text{dim}(E(\lambda))=\text{ma}(\lambda)$.\\
Esto tambi\'en es verdad, pues encontramos que:
\[\text{dim}(E(-1))=1=\text{ma}(-1)\]Y\[\text{dim}(E(1))=3=\text{ma}(1)\]

    \end{itemize}
    Por lo que como ambos puntos se cumplen, tenemos que efectivamente $T$ es diagonizable.
    
\end{itemize}

\section{ Si $V$ es un espacio vectorial real de dimensión finita, y si $P_1,P_2:V \rightarrow V $ son proyecciones,
demuestra que son equivalentes:}
\begin{itemize}
    \item [$a)$] $P_1+P_2$ es un proyección.
    \item [$b)$] $P_1 \circ P_2=P_2 \circ P_1=0$
\end{itemize}
\textbf{Demostraci\'on 6:}\\
Por ser una equivalencia (\ie $a)\Longleftrightarrow b)$ ), demostraremos ambos condicionales.
\begin{itemize}
\item $a)\Longrightarrow b)$\\
    Sea $P_1+P_2$ una proyección. P.D.$P_1 \circ P_2=P_2 \circ P_1=0$ \\
   Como $P=P_1+P_2$ una proyección cumple que es una transformaci´on lineal $P : V \rightarrow V$ que verifica $P^2=P$ (siendo $P^2 = P\circ P$).\\
   Como sabemos que $P^2=P$, entonces $P^2(v)=P(v)$, $\forall v\in V$, por lo que si desarrollamos ambos lados con un $v\neq0$:
   \[P^2(v)=(P_1+P_2)[(P_1+P_2)(v)]=(P_1+P_2)(P_1(v)+P_2(v))\]\[=P_1(P_1(v))+P_1(P_2(v))+P_2(P_1(v))+P_2(P_2(v))=P_1^2(v)+P_1\circ P_2(v)+P_2\circ P_1(v)+P_2^2(v)\]
   Si $P_1$ y $P_2$ son proyecciones se sabe que $P_1^2=P_1$ y $P_2^2=P_2$ por lo tanto:
   \[\therefore P^2(v)=P_1(v)+P_1\circ P_2(v)+P_2\circ P_1(v)+P_2(v)\]
   Por el otro lado:
   \[\therefore P(v)=(P_1+P_2)(v)=P_1(v)+P_2(v)\]
   De este modo si $P^2(v)=P(v)$, tenemos:
   \[P_1(v)+P_2(v)=P_1(v)+P_1\circ P_2(v)+P_2\circ P_1(v)+P_2(v)\]
   Entonces como $P_1,P_2$ son transfromaciones lineales por definici\'on, sabemos que existen sus inversos aditivos, por lo que despejando, tenemos:
   \[P_1\circ P_2(v)+P_2\circ P_1(v)=0~~~\Longrightarrow~~~(P_1\circ P_2+P_2\circ P_1)(v)=0\]Como $v\neq 0$, tenemos:
   \[P_1\circ P_2+P_2\circ P_1=0~~~\Longrightarrow~~~-P_1\circ P_2=P_2\circ P_1\]
   Es importante recalcar que $P_1\neq P_2$, pues de lo contrario, tendr\'iamos que $P=2P_1$ y por lo tanto para que $P^2=4P_1=2P_1=P$, tendr\'iamos que $P=P_1=P_2=0$  y directamente tendr\'iamos que $P_1 \circ P_2=P_2 \circ P_1=0$, de este modo tenemos que $P_1-P_2\neq 0$, por lo tanto, recordando que $P_1^2=P_1$ y $P_2^2=P_2$ se tiene:
   \[P_1\circ P_2(v)+P_2\circ P_1(v)=0~~~\Longrightarrow~~~P_1^2\circ P_2(v)+P_2^2\circ P_1(v)=0~~~\Longrightarrow~~~P_1((P_1\circ P_2))(v)+P_2((P_2\circ P_1))(v)=0\]
   Recordando a que llegamos a que $-P_1\circ P_2=P_2\circ P_1$, sis sustituimos, tenemos que usando las porpiedades de transformaciones lineales:
   \[P_1((P_1\circ P_2))(v)+P_2((-P_1\circ P_2))(v)=0~~~\Longrightarrow~~~P_1((P_1\circ P_2))(v)-P_2((P_1\circ P_2))(v)=0~~~\Longrightarrow~~~(P_1-P_2)(P_1\circ P_2)(v)=0\]
   Como sabemos que $v\neq0$, $P_1-P_2\neq 0$, entonces $P_1\circ P_2=0$ y $P_1\circ P_2=-P_2\circ P_1=0$.\qed
        Por tanto Si $P_1, P_2$ son proyecciones y  $P_1+P_2$ tambien lo es $\Longrightarrow$ $P_1 \circ P_2=P_2 \circ P_1=0$ \\
    \item $b)\Longrightarrow a)$\\
    Sea $P_1 \circ P_2=P_2 \circ P_1=0$. P.D. $P_1+P_2$ es un proyección.\\
    Para demostrar que $P_1+P_2$ es un proyección, debemos demostrar que cumple las condiciones de proyecci\'on, es decir que es una transformaci´on lineal $P : V \rightarrow V$ que verifica $P^2=P$ (siendo $P^2 = P\circ P$).:
    \begin{enumerate}
        \item Sea $P=P_1+P_2$, tenemos que $P^2=(P_1+P_2)^2=(P_1+P_2)\circ (P_1+P_2)$, si calculamos $P(v)$, con $v\in V$ arbitrario, entonces tenemos: 
        $$P^2(v)=(P_1+P_2)[(P_1+P_2)(v)]=(P_1+P_2)(P_1(v)+P_2(v))$$
        $$=P_1(P_1(v))+P_1(P_2(v))+P_2(P_1(v))+P_2(P_2(v))=P_1^2(v)+P_1\circ P_2(v)+P_2\circ P_1(v)+P_2^2(v)$$
        por hip\'otesis tenemos que $P_1 \circ P_2=P_2 \circ P_1=0$, por lo tanto:
        $$P^2(v)=P_1^2(v)+P_2^2(v)$$
        pero si tanto $P_1$ como $P_2$ son proyecciones se sabe que $P_1^2=P_1$ y $P_2^2=P_2$ por lo tanto:
        $$P^2(v)=P_1^2(v)+P_2^2(v)=P_1(v)+P_2(v)=(P_1+P_2)(v)=P(v)$$
        $$\therefore P^2=P$$
        \item Dado que al ser $P_1$ y $P_2$ proyecciones, entonces son por definici\'on transformaciones lineales de $V$ a $V$, por lo tanto $P=P_1+P_2$ es lineal y la demostraci\'on es directa y $P:V\rightarrow V$ es una transformaci\'on lineal.
        %\item De igual forma si $P_1$ y $P_2$ son proyecciones, son continuas, entonces: $P_1+P_2$ es continua        $\Rightarrow P$ es continua        \item Consideremos $P^*=(P_1+P_2)^*$        $$P^*=P_1^*+P_2^*$$        Por la condición de que $P_1$, $P_2$ son proyecciones tenemos:

    \end{enumerate} 
        %$$P^*=P_1+P_2\Rightarrow P^*=P$$
        Por tanto Si $P_1, P_2$ son proyecciones y  $P_1 \circ P_2=P_2 \circ P_1=0$ $\Longrightarrow$ $P_1+P_2$ tambien lo es\\
    
    Por ambos puntos probados podemos afirmar que $a)$ y $b)$ son equivalentes.
    %\item $P_1 \circ P_2=P_2 \circ P_1=0$    Pd. el espacio columna de $P_1=M_1$ y $P_2=M_2$ son ortogonales    Consideremos $x\in M_1$ y $y\in M_2$, y además $M_1=\{ x\in V: X=P_1x \}; \: M_2=\{y\in V: y=P_2y\}$\\    entonces. $P_1x=x$, $P_2y=y$\\    \textcolor{blue}{    Entonces $\norm{x}^2=\norm{P_1x}^2\leq \norm{(P_1+P_2)x}^2$    $$=<(P_1+P_2)x,(P_1+P_2)^*x>$$    $$=<(P_1+P_2)(P_1+P_2)X,X>$$    $$=<(P_+P_2)^2x,x>$$    $$=<(P_1+P_2)x,x>$$    $$=<P_1x,x>+<P_2x,x>$$    $$=<Px,x>=<P^*x,x>=<Px,P^*x>=\norm{Px}^2\leq \norm{x}^2$$    $$\therefore \norm{x}^2=\norm{Px+(I-P)x}^2=\norm{Px}^2+\norm{(I-P)x}^2\geq \norm{Px}^2$$    Ahora bien    $$\norm{x}^2=\norm{P_1x}^2\leq \norm{(P_1+P_2)x}^2\leq \norm{x}^2$$    $$\Rightarrow \norm{x}^2=\norm{P_1x}^2= \norm{(P_1+P_2)x}^2$$    $\Rightarrow \norm{P_1x}^2+\norm{P_2x}^2=\norm{P_1x}^2$    $$\Rightarrow \norm{P_2x}^2=0$$    $$\Rightarrow P_2x=0$$    $$\Rightarrow x\in \mbox{espacio nulo de } P_2=M_2^{\perp}$$    $$\Rightarrow X\in M_2^{\perp} $$    $$ M_1\subseteq M_2$$    $$\therefore M_1\perp M_2 \Rightarrow P_1\circ P_2=0$$}    y como  \footnote{Cabe notar que tambien es posible repetir los pasos en azul intercambiando a \textcolor{red}{$y$} por \textcolor{red}{$x$} } $$a\perp b \Leftrightarrow b\perp a$$     $$M_2 \perp M_1 \Rightarrow P_2 \circ P_2 =0$$    $\therefore$    $$P_1\circ P_2=P_2 \circ P_1 =0$$
    
\end{itemize}
\section{Sea $T:V \longrightarrow V$ lineal. Prueba que:}


\begin{itemize}
\item[$a)$]Si $\langle T(u),v\rangle=0$, para cada $u,v \in V $, entonces es $T=0$\\\\
    \textbf{Demostración 7.a:}\\
    Dados $T:V\longrightarrow V$ y $\langle T(u),v\rangle=0$, $\forall u,v \in V $. P.D. que $T=0$.\\
    Por hip\'otesis tomamos $u,v \in V $ tales que que:
    \[\langle T(u),v\rangle= 0 = \langle u,T^* (v) \rangle\]
    \[\Longrightarrow u \perp T^*(v)\]
    \[\Longrightarrow u \in (\text{Im}(T^*))^{\perp}, \forall u \in V \]

    Y en la ayudant\'ia vimos que sucede que $(\text{Im}(T^*))^{\perp} =\text{Nuc}(T)$

    \[\Longrightarrow u \in \text{Nuc}(T)\]Lo que por definici\'on nos indica que
    \[T(u)=0\]
    Pero como $u$ es arbitraria, entonces se cumple para todo $u\in V$ y de esta forma tenemos que $T=0$

$\therefore \langle T(u),v\rangle=0$, para cada $u,v \in V \Longrightarrow T=0. $\qed


\item[$b)$] Si $V$ es un espacio complejo y si $\langle T(u), u\rangle=0$, para cada $u \in V$, entonces $T=0$ \\\\
\textbf{Demostraci\'on 7.b:}\\
Dados $T:V\longrightarrow V$ donde V es un espacio complejo pues por hip\'otesis nos dan que est\'a definido sobre $\mathbb{C}$) y $\langle T(u), u\rangle=0$, $\forall u\in V$, P.D. $T=0$ \\ 
Sea $u=ra+b$, $\forall a, b \in V$, $r \in \mathbb{C}$ De manera que por propiedades del producto interno en $\mathbb{C}$, tenemos:
\[0=\langle T(ra+b), ra+b\rangle=\langle T(ra)+T(b), ra+b\rangle=\langle rT(a), ra+b\rangle+\langle T(b), ra+b\rangle=r\overline{\langle ra+b,T(a)\rangle}+\overline{\langle ra+b,T(b)\rangle}\]\[=r\overline{r}\cdot\overline{\langle a,T(a)\rangle}+r\overline{\langle b,T(a)\rangle}+\overline{r}\cdot\overline{\langle a,T(b)\rangle}+\overline{\langle b,T(b)\rangle}\]
\[=|r|^2 \langle T(a),a \rangle + r\langle T(a),b\rangle + \overline{r}\langle  T(b), a\rangle+ \langle T(b), b\rangle\]
Pero recordando que $\langle T(u), u\rangle=0$, $\forall u\in V$, tenemos:
\[=|r|^2 (0) + r\langle T(a),b\rangle + \overline{r}\langle  T(b), a\rangle+0=r\langle T(a),b\rangle + \overline{r}\langle  T(b), a\rangle\]
\[\therefore 0=r\langle T(a),b\rangle + \overline{r}\langle  T(b),a\rangle\]

Si le damos el valor de $r=1=\overline{r}$, entonces tenemos que: \[\langle T(a), b \rangle + \langle T(b), a \rangle = 0\]
Si le damos el valor de $r=i$, entonces $\overline{r}=-i$ tenemos que: \[i\langle T(a), b \rangle - i\langle T(b), a \rangle = 0~~~\Longrightarrow~~~\langle T(a), b \rangle - \langle T(b), a \rangle = 0\]
Si analizamos estas dos ecuaciones (y las sumamos o restamos para encobtrar valores) entonces nos podemos percatar de que $\langle T(a), b \rangle=\langle T(b), a \rangle=0$, $\forall a, b \in V$ (para cada par de vectores), entonces por la \textbf{Demostraci\'on 7.a}, tenemos que $T=0$. \qed



\item[$c)$] Si $T$ es autoadjunto y si $\langle T(u), u =0\rangle$, para cada $u \in V$, entonces $T=0$\\\\
\textbf{Demostraci\'on 7.c:}\\
Dados $T:V\longrightarrow V$, $T$ autoadjunto y $\langle T(u),u\rangle=0$. $\forall u\in v$. P.D $T=0$.\\
Por hip\'otesis tenemos que $T=T^*$ (por definici\'on de autoadjunto):
\[\langle T(u),u\rangle= 0 = \langle u,T^*(u) \rangle= \langle u,T(u) \rangle\]
\[\Longrightarrow u \perp T(u)\]
\[\Longrightarrow u \in (\text{Im}(T^*))^{\perp}\]
ya que $T(u)\in \text{Im}(T)$ y por lo visto en la ayudant\'ia y ya mencionado anteriormente $(\text{Im}(T^*))^{\perp} = \text{Nuc}(T)$. As\'i mismo,como $T=T^*$, entonces por definici\'on vamos a tener que:
\[\Longrightarrow u \in \text{Nuc}(T^*)\]
\[\text{Nuc}(T^*)=\text{Nuc}(T)\]
\[u \in \text{Nuc}(T)\]
\[T(u)=0\]
Y como se vale para todo $u\in V$ (en especial si 
$u\neq0$), tenemos finalmente que:
\[T=0\]
$\therefore $ Si $T$ es autoadjunto y si $\langle T(u), u =0\rangle$, para cada $u \in V, \Longrightarrow T=0$. $\qed$

\end{itemize}






\section{Calcula el polinomio característico, valores y vectores propios de las siguientes matrices
complejas:}
\[a) A= \begin{pmatrix}0& -i\\
-2i &-2\end{pmatrix}\hspace{1cm}B= \begin{pmatrix}
0&1&0&0\\
0& 0 &1& 0\\
0&0&0&1\\
1&0&0&0\end{pmatrix}\]
\textbf{Soluci\'on 8:}\\
Encontrar todos los valores y vectores propios correspondientes.
    Sabemos que para encontrar un valor propio $\lambda_n$ de una matriz $A\in\mathcal{M}_{n\times n }$ que satisfaga $A\vec{v}=\lambda_n\vec{v}$, debemos tener que $\text{det}(A-\lambda_nI_{n\times n })=0$, de esta manera obtenemos su polinomio caracter\'istico, cuyas $n$ ra\'ices son sus valores propios.
    \begin{enumerate}
        \item[$a)$] Lo primero que se deber\'a hacer es encontrar su polinomio caracter\'istico:
        \[0=\text{det}(A-\lambda_nI_{n\times n })=\left|\begin{pmatrix}0& -i\\
-2i &-2\end{pmatrix}-\lambda\begin{pmatrix}
1 &0\\0 &1
\end{pmatrix}\right|=\begin{vmatrix}-\lambda &-i\\
-2i &-2-\lambda\end{vmatrix}=(-\lambda)(-2-\lambda)-(-2i)(-i)\]\[=2\lambda+\lambda^2+2=\lambda^2+2\lambda+2\]
Por lo que su polinomio caracter\'istico es $\lambda^2+2\lambda+2=0$, de modo que sus ra\'ices (valores propios) son:
\[\lambda= \frac{-(2)\pm\sqrt{(2)^2-4(1)(2)}}{2(1)}= \frac{-2\pm\sqrt{4-8}}{2}= \frac{-2\pm\sqrt{-4}}{2}= \frac{-2\pm\sqrt{4}\sqrt{-1}}{2}= \frac{-2\pm2i}{2}=-1\pm i\]$\lambda_1=-1-i$ y $\lambda_2=-1+i$. Ahora lo que haremos ser\'a calcular los vectores propios de cada uno, para lo cual debemos resolver la ecuaci\'on $(A-\lambda_nI_{n\times n })\vec{v}=\vec{0}$ (con $\vec{v}=(x,y)\in\mathbb{C}^2$):
\begin{itemize}
    \item Para $\lambda_1=-1-i$, sustituyendo, tenemos que:
    \[\begin{pmatrix}0\\
0\end{pmatrix}=\vec{0}=(A-\lambda_1I_{n\times n })\vec{v}=\begin{pmatrix}0-(-1-i) &-i\\
-2i &-2-(-1-i)\end{pmatrix}\begin{pmatrix}x\\
y\end{pmatrix}=\begin{pmatrix}1+i &-i\\
-2i &-1+i\end{pmatrix}\begin{pmatrix}x\\
y\end{pmatrix}=\begin{pmatrix}(1+i)x-iy\\
-2ix+(-1+i)y\end{pmatrix}\]
\[\therefore \begin{pmatrix}0\\
0\end{pmatrix}=\begin{pmatrix}(1+i)x-iy\\
-2ix+(-1+i)y\end{pmatrix}\]
Por lo que tenemos el siguiente sistema de ecuaciones:
\begin{eqnarray*}
(1+i)x-iy&=&0\\
-2ix+(-1+i)y&=&0
\end{eqnarray*}
Pero si nos damos cuenta la primera ecuaci\'on es m\'ultiplo de la segunda, pues si dividimos los coeficientes de las $x$ entre los coeficientes de las $y$ nos queda $\displaystyle\frac{1+i}{-i}=i(1+i)=-1+i=\frac{-2i}{-1+i}=\frac{2i}{1-i}=\frac{2i(1+i)}{2}$.\\
Si despejamos de la segunda ecuaci\'on tenemos que $iy=(1+i)x~\Longrightarrow~~y=-i(1+i)x=(1-i)x$, por lo que si damos a $x\in\mathbb{R}$ como valor fijo tenemos un vector $(x,y)=(x,(1-i)x)=x(1,1-i)$.\\
S.P.G., damos el valor de $x=1$, por lo que el valor propio $\lambda_1=-1-i$ tiene asociado el vector propio $\vec{v_1}=(1,1-i)$.

\item Para $\lambda_2=-1+i$, sustituyendo, tenemos que:
    \[\begin{pmatrix}0\\
0\end{pmatrix}=\vec{0}=(A-\lambda_2I_{n\times n })\vec{v}=\begin{pmatrix}0-(-1+i) &-i\\
-2i &-2-(-1+i)\end{pmatrix}\begin{pmatrix}x\\
y\end{pmatrix}=\begin{pmatrix}1-i &-i\\
-2i &-1-i\end{pmatrix}\begin{pmatrix}x\\
y\end{pmatrix}=\begin{pmatrix}(1-i)x-iy\\
-2ix+(-1-i)y\end{pmatrix}\]
\[\therefore \begin{pmatrix}0\\
0\end{pmatrix}=\begin{pmatrix}(1-i)x-iy\\
-2ix+(-1-i)y\end{pmatrix}\]
Por lo que tenemos el siguiente sistema de ecuaciones:
\begin{eqnarray*}
(1-i)x-iy&=&0\\
-2ix+(-1-i)y&=&0
\end{eqnarray*}
Pero si nos damos cuenta la primera ecuaci\'on es m\'ultiplo de la segunda, pues si dividimos los coeficientes de las $x$ entre los coeficientes de las $y$ nos queda $\displaystyle\frac{1-i}{-i}=i(1-i)=1+i=\frac{-2i}{-1-i}=\frac{2i}{1+i}=\frac{2i(1-i)}{2}$.\\
Si despejamos de la segunda ecuaci\'on tenemos que $iy=(1-i)x~\Longrightarrow~~y=-i(1-i)x=(-1-i)x$, por lo que si damos a $x\in\mathbb{R}$ como valor fijo tenemos un vector $(x,y)=(x,(-1-i)x)=x(1,-1-i)$.\\
S.P.G., damos el valor de $x=1$, por lo que el valor propio $\lambda_2=-1-i$ tiene asociado el vector propio $\vec{v_2}=(1,-1-i)$.
\end{itemize}


\item[$b)$] Lo primero que se deber\'a hacer es encontrar su polinomio caracter\'istico por cofactores sobre la primera columna:
        \[0=\text{det}(B-\lambda_nI_{n\times n })=\left|\begin{pmatrix}0&1&0&0\\
0& 0 &1& 0\\
0&0&0&1\\
1&0&0&0\end{pmatrix}-\lambda\begin{pmatrix}
1 &0&0&0\\0 &1&0&0\\0 &0&1&0\\0 &0&0&1
\end{pmatrix}\right|=\begin{vmatrix}-\lambda&1&0&0\\
0& -\lambda &1& 0\\
0&0&-\lambda&1\\
1&0&0&-\lambda\end{vmatrix}\]Ahora de la primera lo haremos por cofactores de la primer columna, mientras de la segunda por los del primer rengl\'on:\[=(-\lambda)\begin{vmatrix}
 -\lambda &1& 0\\
0&-\lambda&1\\
0&0&-\lambda\end{vmatrix}-(1)\begin{vmatrix}1&0&0\\
 -\lambda &1& 0\\
0&-\lambda&1\end{vmatrix}=(-\lambda)\left[(-\lambda)\begin{vmatrix}
-\lambda&1\\
0&-\lambda\end{vmatrix}\right]-(1)\left[(1)\begin{vmatrix}
 1& 0\\
-\lambda&1\end{vmatrix}\right]\]\[=\lambda^2\begin{vmatrix}
-\lambda&1\\
0&-\lambda\end{vmatrix}-\begin{vmatrix}
 1& 0\\
-\lambda&1\end{vmatrix}=\lambda^2[(-\lambda)(-\lambda)-(0)(1)]-[(1)(1)-(-\lambda)(0)]=\lambda^4-1\]
Por lo que su polinomio caracter\'istico es $\lambda^4-1=(\lambda^2-1)(\lambda^2+1)=(\lambda-1)(\lambda+1)(\lambda-i)(\lambda+i)=0$, de modo que sus ra\'ices (valores propios) son:
\[\lambda_1=1~~~\lambda_2=-1~~~\lambda_3=i~~~\lambda_4=-i\]Ahora lo que haremos ser\'a calcular los vectores propios de cada uno, para lo cual debemos resolver la ecuaci\'on $(A-\lambda_nI_{n\times n })\vec{v}=\vec{0}$ (con $\vec{v}=(x,y,z,w)\in\mathbb{C}^4$):
\begin{itemize}
    \item Para $\lambda_1=1$, sustituyendo, tenemos que:
    \[\begin{pmatrix}0\\
0\\0\\0\end{pmatrix}=\vec{0}=(B-\lambda_1I_{n\times n })\vec{v}=\begin{pmatrix}-1&1&0&0\\
0& -1 &1& 0\\
0&0&-1&1\\
1&0&0&-1\end{pmatrix}\begin{pmatrix}x\\
y\\z\\w\end{pmatrix}=\begin{pmatrix}-x+y\\
-y+z\\-z+w\\-w+x\end{pmatrix}\]
\[\therefore \begin{pmatrix}0\\
0\\0\\0\end{pmatrix}=\begin{pmatrix}-x+y\\
-y+z\\-z+w\\-w+x\end{pmatrix}\]
Por lo que tenemos el siguiente sistema de ecuaciones:
\begin{eqnarray*}
-x+y&=&0\\
-y+z&=&0\\-z+w&=&0\\-w+x&=&0
\end{eqnarray*}
Si despejamos cada ecuaci\'on llegamos a que $x=y=z=w$, por lo que si damos a $x\in\mathbb{R}$ como valor fijo tenemos un vector $(x,y,z,w)=(x,x,x,x)=x(1,1,1,1)$.\\
S.P.G., damos el valor de $x=1$, por lo que el valor propio $\lambda_1=1$ tiene asociado el vector propio $\vec{v_1}=(1,1,1,1)$.

\item Para $\lambda_2=-1$, sustituyendo, tenemos que:
    \[\begin{pmatrix}0\\
0\\0\\0\end{pmatrix}=\vec{0}=(B-\lambda_2I_{n\times n })\vec{v}=\begin{pmatrix}1&1&0&0\\
0& 1 &1& 0\\
0&0&1&1\\
1&0&0&1\end{pmatrix}\begin{pmatrix}x\\
y\\z\\w\end{pmatrix}=\begin{pmatrix}x+y\\
y+z\\z+w\\w+x\end{pmatrix}\]
\[\therefore \begin{pmatrix}0\\
0\\0\\0\end{pmatrix}=\begin{pmatrix}x+y\\
y+z\\z+w\\w+x\end{pmatrix}\]
Por lo que tenemos el siguiente sistema de ecuaciones:
\begin{eqnarray*}
x+y&=&0\\
y+z&=&0\\z+w&=&0\\w+x&=&0
\end{eqnarray*}
Si despejamos la primera ecuaci\'on llegamos a que $x=-y$, de la segunda tenemos que $-y=z$, de la tercera que $z=-w$ y de la cuarta que $x=-w$, por lo que nos da que $x=z=-y=-w$, por lo que si damos a $x\in\mathbb{R}$ como valor fijo tenemos un vector $(x,y,z,w)=(x,-x,x,-x)=x(1,-1,1,-1)$.\\
S.P.G., damos el valor de $x=1$, por lo que el valor propio $\lambda_2=-1$ tiene asociado el vector propio $\vec{v_2}=(1,-1,1,-1)$.


\item Para $\lambda_3=i$, sustituyendo, tenemos que:
    \[\begin{pmatrix}0\\
0\\0\\0\end{pmatrix}=\vec{0}=(B-\lambda_3I_{n\times n })\vec{v}=\begin{pmatrix}-i&1&0&0\\
0& -i &1& 0\\
0&0&-i&1\\
1&0&0&-i\end{pmatrix}\begin{pmatrix}x\\
y\\z\\w\end{pmatrix}=\begin{pmatrix}-ix+y\\
-iy+z\\-iz+w\\-iw+x\end{pmatrix}\]
\[\therefore \begin{pmatrix}0\\
0\\0\\0\end{pmatrix}=\begin{pmatrix}-ix+y\\
-iy+z\\-iz+w\\-iw+x\end{pmatrix}\]
Por lo que tenemos el siguiente sistema de ecuaciones:
\begin{eqnarray*}
-ix+y&=&0\\
-iy+z&=&0\\-iz+w&=&0\\-iw+x&=&0
\end{eqnarray*}
Si despejamos la primera ecuaci\'on llegamos a que $ix=y$, de la segunda tenemos que $iy=z$, de la tercera que $iz=w$ y de la cuarta que $x=iw$, por lo que nos da que $x=-z=-iy=iw$, por lo que si damos a $x\in\mathbb{R}$ como valor fijo tenemos un vector $(x,y,z,w)=(x,ix,-x,-ix)=x(1,i,-1,-i)$.\\
S.P.G., damos el valor de $x=1$, por lo que el valor propio $\lambda_3=i$ tiene asociado el vector propio $\vec{v_3}=(1,i,-1,-i)$.


\item Para $\lambda_4=-i$, sustituyendo, tenemos que:
    \[\begin{pmatrix}0\\
0\\0\\0\end{pmatrix}=\vec{0}=(B-\lambda_4I_{n\times n })\vec{v}=\begin{pmatrix}i&1&0&0\\
0& i &1& 0\\
0&0&i&1\\
1&0&0&i\end{pmatrix}\begin{pmatrix}x\\
y\\z\\w\end{pmatrix}=\begin{pmatrix}ix+y\\
iy+z\\iz+w\\iw+x\end{pmatrix}\]
\[\therefore \begin{pmatrix}0\\
0\\0\\0\end{pmatrix}=\begin{pmatrix}ix+y\\
iy+z\\iz+w\\iw+x\end{pmatrix}\]
Por lo que tenemos el siguiente sistema de ecuaciones:
\begin{eqnarray*}
ix+y&=&0\\
iy+z&=&0\\iz+w&=&0\\iw+x&=&0
\end{eqnarray*}
Si despejamos la primera ecuaci\'on llegamos a que $-ix=y$, de la segunda tenemos que $-iy=z$, de la tercera que $-iz=w$ y de la cuarta que $x=-iw$, por lo que nos da que $x=-z=iy=iw$, por lo que si damos a $x\in\mathbb{R}$ como valor fijo tenemos un vector $(x,y,z,w)=(x,-ix,-x,ix)=x(1,-i,-1,i)$.\\
S.P.G., damos el valor de $x=1$, por lo que el valor propio $\lambda_4=-i$ tiene asociado el vector propio $\vec{v_4}=(1,-i,-1,i)$.


\end{itemize}
\end{enumerate}

\section{Determina los valores y vectores propios de las siguientes transformaciones (operadores):}
\textbf{Soluci\'on 9:}\\
\begin{itemize}
\item[$a)$] $T: \mathbb{R}^2 \rightarrow \mathbb{R}^2$ definida por $T(x, y) = (4x + 3y, 3x - 4y)$.\\\\
Sabemos que para que para obtener los eigenvalore y eigenvectores de una transformaci\'on $T$ es necesario primero obtener su matriz, en nuestro caso, construiremos la matriz $A$ de $2\times 2$ asociada a la transformaci\'on $T$ respecto a la base can\'onica de $\mathbb{R}^2$, entonces primero debemos calcular la transformaci\'on para cada vector de la base can\'onica, es decir:
\[T(\vec{e_1})=T(1,0)=(4(1)+3(0),3(1)-4(0))=(4,3)\]
\[T(\vec{e_2})=T(0,1)=(4(0)+3(1),3(0)-4(1))=(3,-4)\]
Entonces definimos a $A$ en el que cada columna es el vector resultante de la transfromaci\'on al vector dentro de la base:
\[A=\begin{pmatrix}T(\vec{e_1})&T(\vec{e_2})\end{pmatrix}=\begin{pmatrix}4&3\\
3&-4\end{pmatrix}\]
Y como est\'a construida respecto de la base can\'onica, no es necesario calcular la combinaci\'on lineal respecto de la base can\'onica por cada vector (pues son los coeficientes del vector mismo).\\
Sabemos que para encontrar un valor propio $\lambda_n$ de una matriz $A\in\mathcal{M}_{n\times n }$ que satisfaga $T(\vec{v})=A\vec{v}=\lambda_n\vec{v}$, debemos tener que $\text{det}(A-\lambda_nI_{n\times n })=0$, de esta manera obtenemos su polinomio caracter\'istico, cuyas $n$ ra\'ices son sus valores propios.
Lo primero que se deber\'a hacer es encontrar su polinomio caracter\'istico:
\[0=\text{det}(A-\lambda_nI_{n\times n })=\left|\begin{pmatrix}
4&3\\
3&-4
\end{pmatrix}-\lambda\begin{pmatrix}
1 &0\\0 &1
\end{pmatrix}\right|=\begin{vmatrix}4-\lambda &3\\
3 &-4-\lambda\end{vmatrix}=(4-\lambda)(-4-\lambda)-(3)(3)\]\[=-16+\lambda^2-9=\lambda^2-25\]
Por lo que su polinomio caracter\'istico es $\lambda^2-25=(\lambda+5)(\lambda-5)=0$, de modo que sus ra\'ices (valores propios) son $\lambda_1=-5$ y $\lambda_2=5$. Ahora lo que haremos ser\'a calcular los vectores propios de cada uno, para lo cual debemos resolver la ecuaci\'on $(A-\lambda_nI_{n\times n })\vec{v}=\vec{0}$ (con $\vec{v}=(x,y)$):
\begin{itemize}
    \item Para $\lambda_1=-5$, sustituyendo, tenemos que:
    \[\begin{pmatrix}0\\
0\end{pmatrix}=\vec{0}=(A-\lambda_1I_{n\times n })\vec{v}=\begin{pmatrix}4-(-5) &3\\
3&-4-(-5)\end{pmatrix}\begin{pmatrix}x\\
y\end{pmatrix}=\begin{pmatrix}9 &3\\
3 &1\end{pmatrix}\begin{pmatrix}x\\
y\end{pmatrix}=\begin{pmatrix}9x+3y\\
3x+y\end{pmatrix}\]
\[\therefore \begin{pmatrix}0\\
0\end{pmatrix}=\begin{pmatrix}9x+3y\\
3x+y\end{pmatrix}\]
Por lo que tenemos el siguiente sistema de ecuaciones:
\begin{eqnarray*}
9x+3y&=&0\\
3x+y&=&0
\end{eqnarray*}
Pero si nos damos cuenta la primera ecuaci\'on es m\'ultiplo de la segunda, pues si dividimos los coeficientes de las $x$ entre los coeficientes de las $y$ nos queda $\displaystyle\frac{9}{3}=\frac{3}{1}=3$.\\
Si despejamos de la segunda ecuaci\'on tenemos que $y=-3x$, por lo que si damos a $x\in\mathbb{R}$ como valor fijo tenemos un vector $(x,y)=(x,-3x)=x(1,-3)$.\\
S.P.G., damos el valor de $x=1$, por lo que el valor propio $\lambda_1=-5$ tiene asociado el vector propio $\vec{v_1}=(1,-3)$.


\item Para $\lambda_2=5$, sustituyendo, tenemos que:
    \[\begin{pmatrix}0\\
0\end{pmatrix}=\vec{0}=(A-\lambda_1I_{n\times n })\vec{v}=\begin{pmatrix}4-(5) &3\\
3&-4-(5)\end{pmatrix}\begin{pmatrix}x\\
y\end{pmatrix}=\begin{pmatrix}-1 &3\\
3 &-9\end{pmatrix}\begin{pmatrix}x\\
y\end{pmatrix}=\begin{pmatrix}-x+3y\\
3x-9y\end{pmatrix}\]
\[\therefore \begin{pmatrix}0\\
0\end{pmatrix}=\begin{pmatrix}-x+3y\\
3x-9y\end{pmatrix}\]
Por lo que tenemos el siguiente sistema de ecuaciones:
\begin{eqnarray*}
-x+3y&=&0\\
3x-9y&=&0
\end{eqnarray*}
Pero si nos damos cuenta la primera ecuaci\'on es m\'ultiplo de la segunda, pues si dividimos los coeficientes de las $x$ entre los coeficientes de las $y$ nos queda $\displaystyle\frac{-1}{3}=\frac{3}{-9}=-\frac{1}{3}$.\\
Si despejamos de la segunda ecuaci\'on tenemos que $3y=x$, por lo que si damos a $y\in\mathbb{R}$ como valor fijo tenemos un vector $(x,y)=(3y,y)=y(3,1)$.\\
S.P.G., damos el valor de $x=1$, por lo que el valor propio $\lambda_2=5$ tiene asociado el vector propio $\vec{v_2}=(3,1)$.

\end{itemize}
\item[$b)$] $T: \mathbb{R}^3 \rightarrow \mathbb{R}^3$ dada por $T(x, y, z) = (2y - z, 2x - z, 2x -y)$.\\\\
Sabemos que para que para obtener los eigenvalore y eigenvectores de una transformaci\'on $T$ es necesario primero obtener su matriz, en nuestro caso, construiremos la matriz $B$ de $3\times 3$ asociada a la transformaci\'on $T$ respecto a la base can\'onica de $\mathbb{R}^3$, entonces primero debemos calcular la transformaci\'on para cada vector de la base can\'onica, es decir:
\[T(\vec{e_1})=T(1,0,0)=(2(0) - (0), 2(1) - (0), 2(1) -(0))=(0,2,2)\]
\[T(\vec{e_2})=T(0,1,0)=(2(1) - (0), 2(0) - (0), 2(0) -(1))=(2,0,-1)\]
\[T(\vec{e_3})=T(0,0,1)=(2(0) - (1), 2(0) - (1), 2(0) -(0))=(-1-1,0)\]
Entonces definimos a $A$ en el que cada columna es el vector resultante de la transfromaci\'on al vector dentro de la base:
\[B=\begin{pmatrix}T(\vec{e_1})&T(\vec{e_2})&T(\vec{e_3})\end{pmatrix}=\begin{pmatrix}0&2&-1\\2&0&-1\\2&-1&0\end{pmatrix}\]
Y como est\'a construida respecto de la base can\'onica, no es necesario calcular la combinaci\'on lineal respecto de la base can\'onica por cada vector (pues son los coeficientes del vector mismo).\\
Sabemos que para encontrar un valor propio $\lambda_n$ de una matriz $B\in\mathcal{M}_{n\times n }$ que satisfaga $T(\vec{v})=B\vec{v}=\lambda_n\vec{v}$, debemos tener que $\text{det}(B-\lambda_nI_{n\times n })=0$, de esta manera obtenemos su polinomio caracter\'istico, cuyas $n$ ra\'ices son sus valores propios.
Lo primero que se deber\'a hacer es encontrar su polinomio caracter\'istico, lo cual haremos por cofactores sobre la primera columna:
\[0=\text{det}(B-\lambda_nI_{n\times n })=\left|\begin{pmatrix}
0&2&-1\\
2&0&-1\\
2&-1&0
\end{pmatrix}-\lambda\begin{pmatrix}
1 &0&0\\0 &1&0\\0&0&1
\end{pmatrix}\right|=\begin{vmatrix}-\lambda&2&-1\\
2&-\lambda&-1\\
2&-1&-\lambda\end{vmatrix}=-\lambda\begin{vmatrix}
-\lambda&-1\\
-1&-\lambda\end{vmatrix}-2\begin{vmatrix}2&-1\\
-1&-\lambda\end{vmatrix}+2\begin{vmatrix}2&-1\\
-\lambda&-1\end{vmatrix}\]\[=-\lambda[(-\lambda)(-\lambda)-(-1)(-1)]-2[(2)(-\lambda)-(-1)(-1)]+2[(2)(-1)-(-\lambda)(-1)]=-\lambda[\lambda^2-1]-2[-2\lambda-1]+2[-2-\lambda]\]\[=-\lambda^3+\lambda+4\lambda+2-4-2\lambda=-\lambda^3+3\lambda-2\]
Por lo que su polinomio caracter\'istico es $\lambda^3-3\lambda+2=0$, del cual si jos damos cuenta 1 es ra\'iz, pues $1-3+2=0$, de modo que sus ra\'ices (valores propios) son:
\[\lambda^3-3\lambda+2=(\lambda-1)(\lambda^2+\lambda-2)=(\lambda-1)(\lambda-1)(\lambda+2)=0\]
$\lambda_1=\lambda_2=1$ (tiene multiplicidad 2) y $\lambda_3=-2$. Ahora lo que haremos ser\'a calcular los vectores propios de cada uno, para lo cual debemos resolver la ecuaci\'on $(B-\lambda_nI_{n\times n })\vec{v}=\vec{0}$ (con $\vec{v}=(x,y,z)$):
\begin{itemize}
    \item Para $\lambda_1=\lambda_2=1$, sustituyendo, tenemos que:
    \[\begin{pmatrix}0\\
0\\0\end{pmatrix}=\vec{0}=(B-\lambda_1I_{n\times n })\vec{v}=\begin{pmatrix}-1&2&-1\\
2&-1&-1\\
2&-1&-1\end{pmatrix}\begin{pmatrix}x\\
y\\z\end{pmatrix}=\begin{pmatrix}-x+2y-z\\
2x-y-z\\2x-y-z\end{pmatrix}\]
\[\therefore \begin{pmatrix}0\\
0\\0\end{pmatrix}=\begin{pmatrix}-x+2y-z\\
2x-y-z\\2x-y-z\end{pmatrix}\]
Por lo que tenemos el siguiente sistema de ecuaciones:
\begin{eqnarray*}
-x+2y-z&=&0\\
2x-y-z&=&0\\2x-y-z&=&0
\end{eqnarray*}
Pero si nos damos cuenta la segunda ecuaci\'on es igual a la tercera,
, por lo que si restamos la primera ecuaci\'on a la segunda, tenemos:
\[2x-y-z-(-x+2y-z)=0-0~~\Longrightarrow~3x-3y=0~~\Longrightarrow~x=y\]
De modo que al sustituir en la tercera:
\[2x-y-z=2y-y-z=y-z=0~~\Longrightarrow~y=z\]Por lo que si damos a $x\in\mathbb{R}$ como valor fijo tenemos un vector $(x,y,z)=(x,x,x)=x(1,1,1)$.\\
S.P.G., damos el valor de $x=1$, por lo que el valor propio $\lambda_1=\lambda_2=1$ tiene asociado el vector propio $\vec{v_1}=\vec{v_2}=(1,1,1)$.


\item Para $\lambda_2=-2$, sustituyendo, tenemos que:
    \[\begin{pmatrix}0\\
0\\0\end{pmatrix}=\vec{0}=(B-\lambda_1I_{n\times n })\vec{v}=\begin{pmatrix}2&2&-1\\
2&2&-1\\
2&-1&2\end{pmatrix}\begin{pmatrix}x\\
y\\z\end{pmatrix}=\begin{pmatrix}2x+2y-z\\
2x+2y-z\\2x-y+2z\end{pmatrix}\]
\[\therefore \begin{pmatrix}0\\
0\\0\\0\end{pmatrix}=\begin{pmatrix}2x+2y-z\\
2x+2y-z\\2x-y+2z\end{pmatrix}\]
Por lo que tenemos el siguiente sistema de ecuaciones:
\begin{eqnarray*}
2x+2y-z&=&0\\
2x+2y-z&=&0\\2x-y+2z&=&0
\end{eqnarray*}
Pero si nos damos cuenta la primer ecuaci\'on es igual a la segunda,
, por lo que si restamos la tercera ecuaci\'on a la primera, tenemos:
\[2x+2y-z-(2x-y+2z)=0-0~~\Longrightarrow~3y-3z=0~~\Longrightarrow~y=z\]
De modo que al sustituir en la tercera:
\[2x+2y-z=2x+2y-y=2x+y=0~~\Longrightarrow~y=-2x\]Por lo que si damos a $x\in\mathbb{R}$ como valor fijo tenemos un vector $(x,y,z)=(x,-2x,-2x)=x(1,-2,-2)$.\\
S.P.G., damos el valor de $x=1$, por lo que el valor propio $\lambda_3=-2$ tiene asociado el vector propio $\vec{v_3}=(1,-2,-2)$.


\end{itemize}
\end{itemize}
\section{Calcular la descomposición espectral de las siguientes matrices}
\begin{itemize}
    \item $A=\begin{pmatrix}1&2\\ \:2&-2\end{pmatrix}$\\\\
    \textbf{Soluci\'on 10.a:}\\
    Calculando los valores propios:
    $$\det\left(\begin{pmatrix}1&2\\ \:2&-2\end{pmatrix}-\lambda \begin{pmatrix}1-&0\\ 0&1\end{pmatrix}  \right)=\begin{vmatrix}1-\lambda&2\\ \:2&-2-\Lambda\end{vmatrix}$$
    de donde obtenemos:
    $$=(1-\lambda)(-2-\lambda)-2(2)=\lambda^2+\lambda-2-4=\lambda^2+\lambda-6\Rightarrow \lambda_1=2~~~;~~ \lambda_2=-3$$
    Calculando los subespacios propios:
    \begin{itemize}
        \item Para $\lambda_1=2$\\
        Sustituyendo en la matriz:
        \[A-2I=\begin{pmatrix}1&2\\ \:2&-2\end{pmatrix}+ \begin{pmatrix}-2&0\\ 0&-2\end{pmatrix}=\begin{pmatrix}-1&2\\ 2&-4\end{pmatrix}\]
        De este modo para encontrar $(A-2I)\Vec{v}=0$, damos $\vec{v}=(x,y)$ e igualamos al vector 0, es decir:
        \[\begin{pmatrix}0\\0\end{pmatrix}=\begin{pmatrix}-1&2\\ 2&-4\end{pmatrix}\begin{pmatrix}x\\y\end{pmatrix}=\begin{pmatrix}-x+2y\\2x-4y\end{pmatrix}\]
        De esta forma tenemos el sistema:
        \begin{eqnarray*}
        -x+2y&=&0\\
        2x-4y&=&0
        \end{eqnarray*}
        Vemos que ambas ecuaciones son multiplos uno de la otra, entonces si despejamos la primera llegamos a que $x=2y$, de modo que si $y$ es nuestra variable libre tenemos que $\vec{v}=(x,y)=(2y,y)=y(2,1)$. Por tanto:
        \[E(2)=<(2,1)>\]
        y $\text{dim}(E(2))=1=\text{ma}(2)$. 
        Normalizando el vector:
        \[v_1=\frac{1}{\Vert v_1\Vert}v'_1=\frac{1}{\sqrt{2^2+1^2}}(2,1)=\frac{1}{\sqrt{5}}(2,1)\]
        
        
        \item Para $\lambda_2=-3$\\
        Sustituyendo en la matriz:
        \[A-(-3)I=\begin{pmatrix}1&2\\ \:2&-2\end{pmatrix}+ \begin{pmatrix}3&0\\ 0&3\end{pmatrix}=\begin{pmatrix}4&2\\ 2&1\end{pmatrix}\]
        De este modo para encontrar $(A+3I)\Vec{v}=0$, damos $\vec{v}=(x,y)$ e igualamos al vector 0, es decir:
        \[\begin{pmatrix}0\\0\end{pmatrix}=\begin{pmatrix}4&2\\ 2&1\end{pmatrix}\begin{pmatrix}x\\y\end{pmatrix}=\begin{pmatrix}4x+2y\\22x+y\end{pmatrix}\]
        De esta forma tenemos el sistema:
        \begin{eqnarray*}
        4x+2y&=&0\\
        2x+y&=&0
        \end{eqnarray*}
        Vemos que ambas ecuaciones son multiplos uno de la otra, entonces si despejamos la segunda llegamos a que $y=-2x$, de modo que si $x$ es nuestra variable libre tenemos que $\vec{v}=(x,y)=(x,-2x)=x(1,-2)$. Por tanto:
        \[E(-3)=<(1,-2)>\]
        y $\text{dim}(E(-3))=1=\text{ma}(-3)$. 
        Normalizando el vector:
        \[v_2=\frac{1}{\Vert v_2\Vert}v'_2=\frac{1}{\sqrt{1^2+(-2)^2}}(1,-2)=\frac{1}{\sqrt{5}}(1,-2)\]
        
        
    \end{itemize}
    Como para todo valor propio $\text{dim}(E(\lambda))=\text{ma}(\lambda)$, llegamos a que la matriz es diagonalizable y por lo tanto tiene descomoposici\'on espectral, de modo que podamos escibir a $A$ como:
    \[A=\lambda_1A_1+\lambda_2A_2\]
    Donde $A_i=\vec{v_i}\vec{v_i}^T$, de este modo calculandolos:
    \begin{itemize}
        \item \[A_1=\left(\frac{1}{\sqrt{5}}\right)^2\begin{pmatrix}2\\1\end{pmatrix}\begin{pmatrix}2&1\end{pmatrix}=\frac{1}{5}\begin{pmatrix}4&2\\2&1\end{pmatrix}\]
        \item \[A_2=\left(\frac{1}{\sqrt{5}}\right)^2\begin{pmatrix}1\\-2\end{pmatrix}\begin{pmatrix}1&-2\end{pmatrix}=\frac{1}{5}\begin{pmatrix}1&-2\\-2&4\end{pmatrix}\]
    \end{itemize}
    Entonces la descomposición queda:
    \[A=\lambda_1A_1+\lambda_2A_2=2\left(\frac{1}{5}\right)\begin{pmatrix}4&2\\2&1\end{pmatrix}+(-3)\left(\frac{1}{5}\right)\begin{pmatrix}1&-2\\-2&4\end{pmatrix}=\frac{1}{5}\begin{pmatrix}8&4\\4&2\end{pmatrix}+\frac{1}{5}\begin{pmatrix}-3&6\\6&-12\end{pmatrix}=\frac{1}{5}\begin{pmatrix}5&10\\10&-10\end{pmatrix}=\begin{pmatrix}1&2\\ \:2&-2\end{pmatrix}\]

    
    \item $B=\begin{pmatrix}4&3i\\ \:\:\:3i&4\end{pmatrix}$  \\\\
    \textbf{Soluci\'on 10.b:}\\
    Calculando determinante:
    $$\det\left(\:\begin{pmatrix}4&3i\\ \:\:\:3i&4\end{pmatrix}-\lambda \:\begin{pmatrix}1&0\\0&1\end{pmatrix}\right)=\begin{pmatrix}4-\lambda&3i\\3i&4-\lambda\end{pmatrix}=(4-\lambda)(4-\lambda)-(3i)(3i)=\lambda^2-8\lambda+16+9=\lambda^2-8\lambda+25$$
    Resolviendo por formula general:
    \[\lambda = \frac{-(-8)\pm\sqrt{(-8)^2-4(1)(25)}}{2(1)}=\frac{8\pm\sqrt{64-100}}{2}=\frac{8\pm\sqrt{-36}}{6}=\frac{8\pm6i}{2}\]
    de donde obtenemos:
    $$\lambda_1=4+3i, \lambda_2= 4-3i$$
    Calculando los subespacios propios:
    \begin{itemize}
        \item Para $\lambda_1=4+3i$\\
        Sustituyendo en la matriz:
        \[B-(4+3i)I=\begin{pmatrix}4&3i\\ \:\:\:3i&4\end{pmatrix}-(4+3i) \:\begin{pmatrix}1&0\\0&1\end{pmatrix}=\begin{pmatrix}-3i&3i\\ \:\:\:3i&-3i\end{pmatrix}\]
        De este modo para encontrar $(B-(4+3i)I)\Vec{v}=0$, damos $\vec{v}=(x,y)$ e igualamos al vector 0, es decir:
        \[\begin{pmatrix}0\\0\end{pmatrix}=\begin{pmatrix}-3i&3i\\ \:\:\:3i&-3i\end{pmatrix}\begin{pmatrix}x\\y\end{pmatrix}=\begin{pmatrix}-3ix+3iy\\3ix-3iy\end{pmatrix}\]
        De esta forma tenemos el sistema:
        \begin{eqnarray*}
        -3ix+3iy&=&0\\
        3ix-3iy&=&0
        \end{eqnarray*}
        Vemos que ambas ecuaciones son multiplos uno de la otra, entonces si despejamos la primera llegamos a que $3ix=3iy~~\Longrightarrow~~x=y$, de modo que si $x$ es nuestra variable libre tenemos que $\vec{v}=(x,y)=(x,x)=x(1,1)$. Por tanto:
        \[E(4+3i)=<(1,1)>\]
        y $\text{dim}(E(4+3i))=1=\text{ma}(4+3i)$. 
        Normalizando el vector:
        \[v_1=\frac{1}{\Vert v_1\Vert}v'_1=\frac{1}{\sqrt{1^2+1^2}}(1,1)=\frac{1}{\sqrt{2}}(1,1)\]
        
        \item Para $\lambda_2=4-3i$\\
        Sustituyendo en la matriz:
        \[B-(4-3i)I=\begin{pmatrix}4&3i\\ \:3i&4\end{pmatrix}- \begin{pmatrix}4-3i&0\\ 0&4-3i\end{pmatrix}=\begin{pmatrix}3i&3i\\ 3i&3i\end{pmatrix}\]
        De este modo para encontrar $(B+(4-3i)I)\Vec{v}=0$, damos $\vec{v}=(x,y)$ e igualamos al vector 0, es decir:
        \[\begin{pmatrix}0\\0\end{pmatrix}=\begin{pmatrix}3i&3i\\ 3i&3i\end{pmatrix}\begin{pmatrix}x\\y\end{pmatrix}=\begin{pmatrix}3ix+3iy\\3ix+3iy\end{pmatrix}\]
        De esta forma tenemos el sistema:
        \begin{eqnarray*}
        3ix+3iy&=&0
        \end{eqnarray*}
        Vemos que ambas ecuaciones son multiplos uno de la otra, entonces si despejamos la segunda llegamos a que $3ix=-3iy~~\Longrightarrow~~-x=y$, de modo que si $x$ es nuestra variable libre tenemos que $\vec{v}=(x,y)=(x,-x)=x(1,-1)$. Por tanto:
        \[E(4-3i)=<(1,-1)>\]
        y $\text{dim}(E(4-3i))=1=\text{ma}(4-3i)$. 
        Normalizando el vector:
        \[v_2=\frac{1}{\Vert v_2\Vert}v'_2=\frac{1}{\sqrt{1^2+(-1)^2}}(1,-1)=\frac{1}{\sqrt{2}}(1,-1)\]
        
        
    \end{itemize}
    Como para todo valor propio $\text{dim}(E(\lambda))=\text{ma}(\lambda)$, llegamos a que la matriz es diagonalizable y por lo tanto tiene descomoposici\'on espectral, de modo que podamos escibir a $A$ como:
    \[B=\lambda_1B_1+\lambda_2B_2\]
    Donde $A_i=\vec{v_i}\vec{v_i}^T$, de este modo calculandolos:
    \begin{itemize}
        \item \[B_1=\left(\frac{1}{\sqrt{2}}\right)^2\begin{pmatrix}1\\1\end{pmatrix}\begin{pmatrix}1&1\end{pmatrix}=\frac{1}{2}\begin{pmatrix}1&1\\1&1\end{pmatrix}\]
        \item \[B_2=\left(\frac{1}{\sqrt{2}}\right)^2\begin{pmatrix}1\\-1\end{pmatrix}\begin{pmatrix}1&-1\end{pmatrix}=\frac{1}{2}\begin{pmatrix}1&-1\\-1&1\end{pmatrix}\]
    \end{itemize}
    
    De esta forma la descomposición espectral queda como:\
    \[B=\lambda_1B_1+\lambda_2B_2=(4+3i)\left(\frac{1}{2}\right)\begin{pmatrix}1&1\\1&1\end{pmatrix}+(4-3i)\left(\frac{1}{2}\right)\frac{1}{2}\begin{pmatrix}1&-1\\-1&1\end{pmatrix}=\frac{1}{2}\begin{pmatrix}4+3i&4+3i\\4+3i&4+3i\end{pmatrix}+\frac{1}{2}\begin{pmatrix}4-3i&-4+3i\\-4+3i&4-3i\end{pmatrix}\]\[=\frac{1}{2}\begin{pmatrix}8&6i\\6i&8\end{pmatrix}=\begin{pmatrix}4&3i\\ \:3i&4\end{pmatrix}\]
    
    
   
   
   
   
   
    
    \item $C=\begin{pmatrix}3&-7&-20\\ \:\:0&-5&-14\\ \:\:0&3&8\end{pmatrix} $
    
    Obteniendo el determinante de:
    $$\det \begin{pmatrix}3-\lambda&-7&-20\\ \:\:0&-5-\lambda&-14\\ \:\:0&3&8-\lambda\end{pmatrix}$$
    Realizando el procedimiento por cofactores sobre la primer columna
    $$=(3-\lambda)\begin{vmatrix}-5-\lambda&-14\\ 3&8-\lambda\end{vmatrix}=(3-\lambda)[(-5-\lambda)(8-\lambda)-3(-14)]=(3-\lambda)[\lambda^2-3\lambda-40+42]=(3-\lambda)(\lambda^2-3\lambda+2)$$$$=-(\lambda-1)(\lambda-2)(\lambda-3)=-\lambda^3+6\lambda^2-11\lambda+6=0$$
    donde obtenemos:
    $$\lambda_1=1~,~\lambda_2=2~,~\lambda_3=3$$
    Calculando los subespacios propios:
    \begin{itemize}
        \item Para $\lambda_1=1$\\
        Sustituyendo en la matriz:
        \[C-I=\begin{pmatrix}3-1&-7&-20\\ \:\:0&-5-1&-14\\ \:\:0&3&8-1\end{pmatrix}=\begin{pmatrix}2&-7&-20\\ \:\:0&-6&-14\\ \:\:0&3&7\end{pmatrix}\]
        De este modo para encontrar $(C-I)\Vec{v}=0$, damos $\vec{v}=(x,y,z)$ e igualamos al vector 0, es decir:
        \[\begin{pmatrix}0\\0\\0\end{pmatrix}=\begin{pmatrix}2&-7&-20\\ \:\:0&-6&-14\\ \:\:0&3&7\end{pmatrix}\begin{pmatrix}x\\y\\z\end{pmatrix}=\begin{pmatrix}2x-7y-20z\\-6y-14z\\3y+7z\end{pmatrix}\]
        De esta forma tenemos el sistema:
        \begin{eqnarray*}
        2x-7y-20z&=&0\\
        -6y-14z&=&0\\
        3y+7z&=&0
        \end{eqnarray*}
        Vemos que las ultimas dos ecuaciones son multiplos uno de la otra, entonces si despejamos la primera llegamos a que $\displaystyle 3y=-7z~~\Longrightarrow~~y=-\frac{7}{3}z$, entonces
        sustituyendo en la primera ecuaci\'on:
        \[0=2x-7\left(-\frac{7}{3}z\right)-20z~~\Longrightarrow~~0=2x+\frac{49-60}{3}z~~\Longrightarrow~~0=2x-\frac{11}{3}z~~\Longrightarrow~~2x=\frac{11}{3}z~~\Longrightarrow~~x=\frac{11}{6}z\], de modo que si $z$ es nuestra variable libre tenemos que $\displaystyle \vec{v}=(x,y,z)=\left(\frac{11}{6}z,-\frac{7}{3}z,z\right)=z\left(\frac{11}{6},-\frac{7}{3}z,z\right)=\frac{1}{6}z(11,-14,6)$. Por tanto:
        \[E(1)=<(11,-14,6)>\]
        y $\text{dim}(E(1))=1=\text{ma}(1)$. 
        Normalizando el vector:
        \[v_1=\frac{1}{\Vert v_1\Vert}v'_1=\frac{1}{\sqrt{11^2+(-14)^2+6^2}}(11,-14,6)=\frac{1}{\sqrt{121+196+36}}(11,-14,6)=\frac{1}{\sqrt{353}}(11,-14,6)\]
        
        \item Para $\lambda_2=2$\\
        Sustituyendo en la matriz:
        \[C-2I=\begin{pmatrix}3-2&-7&-20\\ \:\:0&-5-2&-14\\ \:\:0&3&8-2\end{pmatrix}=\begin{pmatrix}1&-7&-20\\ \:\:0&-7&-14\\ \:\:0&3&6\end{pmatrix}\]
        De este modo para encontrar $(C-I)\Vec{v}=0$, damos $\vec{v}=(x,y,z)$ e igualamos al vector 0, es decir:
        \[\begin{pmatrix}0\\0\\0\end{pmatrix}=\begin{pmatrix}1&-7&-20\\ \:\:0&-7&-14\\ \:\:0&3&6\end{pmatrix}\begin{pmatrix}x\\y\\z\end{pmatrix}=\begin{pmatrix}x-7y-20z\\-7y-14z\\3y+6z\end{pmatrix}\]
        De esta forma tenemos el sistema:
        \begin{eqnarray*}
        x-7y-20z&=&0\\
        -7y-14z&=&0\\
        3y+6z&=&0
        \end{eqnarray*}
        Vemos que las ultimas dos ecuaciones son multiplos uno de la otra, entonces si despejamos la primera llegamos a que $\displaystyle 7y=-14z~~\Longrightarrow~~y=-2z$, entonces
        sustituyendo en la primera ecuaci\'on:
        \[0=x-7\left(-2z\right)-20z~~\Longrightarrow~~0=x+14z-20z~~\Longrightarrow~~0=x-6z~~\Longrightarrow~~x=6z\], de modo que si $z$ es nuestra variable libre tenemos que $\displaystyle \vec{v}=(x,y,z)=\left(6z,-2z,z,z\right)=z(6,-2,1)$. Por tanto:
        \[E(2)=<(6,-2,1)>\]
        y $\text{dim}(E(2))=1=\text{ma}(2)$. 
        Normalizando el vector:
        \[v_2=\frac{1}{\Vert v_2\Vert}v'_2=\frac{1}{\sqrt{6^2+(-2)^2+1^2}}(6,-2,1)=\frac{1}{\sqrt{36+4+1}}(6,-2,1)=\frac{1}{\sqrt{41}}(6,-2,1)\]
        
        
        \item Para $\lambda_3=3$\\
        Sustituyendo en la matriz:
        \[C-3I=\begin{pmatrix}3-3&-7&-20\\ \:\:0&-5-3&-14\\ \:\:0&3&8-3\end{pmatrix}=\begin{pmatrix}0&-7&-20\\ \:\:0&-8&-14\\ \:\:0&3&5\end{pmatrix}\]
        De este modo para encontrar $(C-3I)\Vec{v}=0$, damos $\vec{v}=(x,y,z)$ e igualamos al vector 0, es decir:
        \[\begin{pmatrix}0\\0\\0\end{pmatrix}=\begin{pmatrix}0&-7&-20\\ \:\:0&-8&-14\\ \:\:0&3&5\end{pmatrix}\begin{pmatrix}x\\y\\z\end{pmatrix}=\begin{pmatrix}-7y-20z\\-8y-14z\\3y+5z\end{pmatrix}\]
        De esta forma tenemos el sistema:
        \begin{eqnarray*}
        -7y-20z&=&0\\
        -8y-14z&=&0\\
        3y+5z&=&0
        \end{eqnarray*}
        Vemos que si restamos la primera ecuaci\'on a la segunda obtenemos que $-y+6z=0$, entonces si la sumamos 3 veces con la tercera obtenemos $23z=0~~\Longrightarrow~~z=0$ y de ets aforma tambi\'en $y=0$, por lo que $x$ es la \'unica variable libre, por lo que tenemos que $\displaystyle \vec{v}=(x,y,z)=(x,0,0)=x(1,0,0)$. Por tanto:
        \[E(3)=<(1,0,0)>\]
        y $\text{dim}(E(3))=1=\text{ma}(3)$. 
        Vemos que el vector ya esta normalizado, por lo que :
        \[v_3=(1,0,0)\]
        
        
    \end{itemize}
    Como para todo valor propio $\text{dim}(E(\lambda))=\text{ma}(\lambda)$, llegamos a que la matriz es diagonalizable y por lo tanto tiene descomoposici\'on espectral, de modo que podamos escibir a $A$ como:
    \[C=\lambda_1C_1+\lambda_2C_2+\lambda_3C_3\]
    Donde $C_i=\vec{v_i}\vec{v_i}^T$, de este modo calculandolos:
    \begin{itemize}
        \item \[C_1=\left(\frac{1}{\sqrt{353}}\right)^2\begin{pmatrix}11\\-14\\6\end{pmatrix}\begin{pmatrix}11&-14&6\end{pmatrix}=\frac{1}{353}\begin{pmatrix}121&-154&66\\-154&196&-84\\66&-84&36\end{pmatrix}\]
        \item \[C_2=\left(\frac{1}{\sqrt{41}}\right)^2\begin{pmatrix}6\\-2\\1\end{pmatrix}\begin{pmatrix}6&-2&1\end{pmatrix}=\left(\frac{1}{{41}}\right)^2\begin{pmatrix}36&-12&6\\ -12&4&-2\\ 6&-2&1\end{pmatrix}\]
        \item \[C_3=\begin{pmatrix}1\\0\\0\end{pmatrix}\begin{pmatrix}1&0&0\end{pmatrix}=\begin{pmatrix}1&0&0\\0&0&0\\0&0&0\end{pmatrix}\]
        
    \end{itemize}
    
    De esta forma la descomposición espectral queda como:\
    \[C=\lambda_1C_1+\lambda_2C_2+\lambda_3C_3=1\left(\frac{1}{353}\right)\begin{pmatrix}121&-154&66\\-154&196&-84\\66&-84&36\end{pmatrix}+2\left(\frac{1}{41}\right)\begin{pmatrix}36&-12&6\\-12&4&-2\\6&-2&1\end{pmatrix}+3\begin{pmatrix}1&0&0\\0&0&0\\0&0&0\end{pmatrix}\]\[=\left(\frac{1}{353}\right)\begin{pmatrix}121&-154&66\\-154&196&-84\\66&-84&36\end{pmatrix}+\left(\frac{1}{41}\right)\begin{pmatrix}72&-24&12\\-24&8&-4\\12&-4&2\end{pmatrix}+\begin{pmatrix}3&0&0\\0&0&0\\0&0&0\end{pmatrix}\]
    \[=\begin{pmatrix}\frac{121}{353}&-\frac{154}{353}&\frac{66}{353}\\-\frac{154}{353}&\frac{196}{353}&-\frac{84}{353}\\\frac{66}{353}&-\frac{84}{353}&\frac{36}{353}\end{pmatrix}+\begin{pmatrix}\frac{72}{41}&-\frac{24}{41}&\frac{12}{41}\\-\frac{24}{41}&\frac{8}{41}&-\frac{4}{41}\\12&-\frac{4}{41}&-\frac{2}{41}\end{pmatrix}+\begin{pmatrix}3&0&0\\0&0&0\\0&0&0\end{pmatrix}=\begin{pmatrix}\frac{73796}{14473}&-\frac{14786}{14473}&\frac{6942}{14473}\\ -\frac{14786}{14473}&\frac{10860}{14473}&-\frac{4856}{14473}\\ \frac{4302}{353}&-\frac{4856}{14473}&\frac{770}{14473}\end{pmatrix}\]
    
    
    
   % por lo que la descomposición espectral queda:
    %$$\:\begin{pmatrix}3\\ \:0\\ \:0\end{pmatrix}\begin{pmatrix}3&0&0\end{pmatrix}+2\begin{pmatrix}-7\\ \:-5\\ \:3\end{pmatrix}\begin{pmatrix}-7&-5&3\end{pmatrix}+\begin{pmatrix}-20\\ \:-14\\ \:8\end{pmatrix}\begin{pmatrix}-20&-14&8\end{pmatrix}$$
    %$$ \begin{pmatrix}9&0&0\\ 0&0&0\\ 0&0&0\end{pmatrix}+2\begin{pmatrix}49&35&-21\\ 35&25&-15\\ -21&-15&9\end{pmatrix}+3\begin{pmatrix}400&280&-160\\ 280&196&-112\\ -160&-112&64\end{pmatrix} $$
    %$$\begin{pmatrix}9&0&0\\ 0&0&0\\ 0&0&0\end{pmatrix}+\begin{pmatrix}98&70&-42\\ 70&50&-30\\ -42&-30&18\end{pmatrix}+\begin{pmatrix}1200&840&-480\\ 840&588&-336\\ -480&-336&192\end{pmatrix}$$
    %$$=\begin{pmatrix}1307&910&-522\\ 910&638&-366\\ -522&-366&210\end{pmatrix}$$
    
    
    
    
    
    
    
    \item $D=\begin{pmatrix}1&1+i&0\\ \:\:1-i&2&0\\ \:\:0&0&1\end{pmatrix}$
    Calculando la determinante de:
    $$\det \begin{pmatrix}1-\lambda&1+i&0\\ \:\:1-i&2-\lambda&0\\ \:\:0&0&1-\lambda\end{pmatrix}$$
    $$(1-\lambda)(\lambda^2-3\lambda+2)-(1+i)(\lambda+1+i(-1+\lambda))+0\cdot0$$
    $$=-\lambda^3+4\lambda^2-3\lambda=-\lambda(\lambda-1)(\lambda-3)=0$$
    obtenemos:
    $$\lambda_1=0, \lambda_2=1, \lambda_3=3$$
    
    
    
    
        Calculando los subespacios propios:
\begin{itemize}
    \item Para $\lambda_1=0$\\
        Sustituyendo en la matriz:
        \[D-0I=D=\begin{pmatrix}1&1-i&0\\ \:\:1-i&2&0\\ \:\:0&0&1\end{pmatrix}\]
        De este modo para encontrar $(B-0I)\Vec{v}=0$, damos $\vec{v}=(x,y,z)$ e igualamos al vector 0, es decir:
        
        
        \[\begin{pmatrix}0\\0\\0\end{pmatrix}=\begin{pmatrix}1&1+i&0\\ \:\:1-i&2&0\\ \:\:0&0&1\end{pmatrix}\begin{pmatrix}x\\y\\z\end{pmatrix}=\begin{pmatrix}x+(1+i)y\\ (1-i)x+2y\\ z\end{pmatrix}\]
        De esta forma tenemos el sistema:
        \begin{eqnarray*}
       x+(1+i)y=0\\ 
       (1-i)x+2y=0\\
       1z=0
        \end{eqnarray*}
        
        Tenemos automaticamente que $z=0$ y para las primeras dos ecuaciones son multiplos una de otra despejamos a $x=-y(1+i)$ por lo tanto si tomamos a $y$ como variable libre, tenemos $\vec{v}=(x,y,z)=((-1-i)y,y,0)=y(-1-i,1,0)$
        Por lo que tenemos
        \[E(0)=<(-1-i,1,0)>\]
        y $\text{dim}(E(0))=1=\text{ma}(0)$. 
        Normalizando el vector:
        \[v_1=\frac{1}{\Vert v_1\Vert}v'_1=\frac{1}{\sqrt{1^2+1^2+1^2}}(-1-i,1,0)=\frac{1}{\sqrt{3}}(-1-i,1,0)\]
        
    
    
    
    \item Para $\lambda_2=1$\\
        Sustituyendo en la matriz:
        \[D-I=\begin{pmatrix}1&1+i&0\\ \:\:1-i&2&0\\ \:\:0&0&1\end{pmatrix}-\begin{pmatrix}1&0&0\\ \:\:0&1&0\\ \:\:0&0&1\end{pmatrix}=\begin{pmatrix}0&1+i&0\\ 1-i&1&0\\ 0&0&0\end{pmatrix}\]
        
        
        De este modo para encontrar $(D-I)\Vec{v}=0$, damos $\vec{v}=(x,y,z)$ e igualamos al vector 0, es decir:
        \[\begin{pmatrix}0\\0\\0\end{pmatrix}=\begin{pmatrix}0&1+i&0\\ 1-i&1&0\\ 0&0&0\end{pmatrix}\begin{pmatrix}x\\y\\z\end{pmatrix}=\begin{pmatrix}(1+i)y\\(1-i)x+y\\ 0\end{pmatrix}\]
        De esta forma tenemos el sistema:
        \begin{eqnarray*}
        (1+i)y&=&0\\
        (1-i)x+y&=&0
        \end{eqnarray*}
        Como podemos ver ambas ecuaciones nos indican que $x=y=0$, pero como no hay restricci\'on para $z$, tenemos que $\vec{v}=(x,y,z)=(0,0,z)=z(0,0,1)$
        
        
        \[E(1)=<(0,0,1)>\]
        y $\text{dim}(E(1))=1=\text{ma}(1)$. 
        Comoe le vector ya esta normalizado:
        \[v_2=(0,0,1)\]
        
    
    
    
    
    \item Para $\lambda_3=3$\\
        Sustituyendo en la matriz:
        \[D-3I=\begin{pmatrix}1&1+i&0\\ \:\:1-i&2&0\\ \:\:0&0&1\end{pmatrix}-\begin{pmatrix}3&0&0\\ \:\:0&3&0\\ \:\:0&0&3\end{pmatrix}=\begin{pmatrix}-2&1+i&0\\ 1-i&-1&0\\ 0&0&-2\end{pmatrix}\]
        
        
        De este modo para encontrar $(D-3I)\Vec{v}=0$, damos $\vec{v}=(x,y,z)$ e igualamos al vector 0, es decir:
        \[\begin{pmatrix}0\\0\\0\end{pmatrix}=\begin{pmatrix}-2&1+i&0\\ 1-i&-1&0\\ 0&0&-2\end{pmatrix}\begin{pmatrix}x\\y\\z\end{pmatrix}=\begin{pmatrix}-2x+(1+i)y\\(1-i)x-y\\ -2z\end{pmatrix}\]
        De esta forma tenemos el sistema:
        \begin{eqnarray*}
        -2x+(1+i)y&=&0\\
        (1-i)x-y&=&0        \\
        2z&=&0
        \end{eqnarray*}
        Tenemos directo que $z=0$, entonces observamos que la primer y segunda son multiplos d euna misma ecuaci\'on , por tanto despejando de la primer, tenemos $2x=(1+i)y$, por lo tanto tomamos a $y$ como variable libre, de modo que $\vec{v}=(x,y,z)=y(1+i,2,0)$.
        
        
        \[E(2)=<(1+i,2,0)>\]
        y $\text{dim}(E(1))=1=\text{ma}(1)$. 
        Normalizando el vector:
        \[v_3=\frac{1}{\Vert v_3\Vert}v'_3=\frac{1}{\sqrt{1^2+1^2+2^2}}(1+i,2,0)=\frac{1}{\sqrt{6}}(1+i,2,0)\]
    
        
    \end{itemize}
    Como para todo valor propio $\text{dim}(E(\lambda))=\text{ma}(\lambda)$, llegamos a que la matriz es diagonalizable y por lo tanto tiene descomoposici\'on espectral, de modo que podamos escibir a $A$ como:
    \[D=\lambda_1D_1+\lambda_2D_2+\lambda_3D_3\]
    Donde $D_i=\vec{v_i}\overline{\vec{v_i}}^T$, de este modo calculandolos:
    \begin{itemize}
        \item \[D_1=\left(\frac{1}{\sqrt{3}}\right)^2\begin{pmatrix}-1-i\\1\\0\end{pmatrix}\begin{pmatrix}-1+i&1&0\end{pmatrix}=\frac{1}{3}\begin{pmatrix}2i&-1-i&0\\-1-i&1&0\\0&0&0\end{pmatrix}\]
        \item \[D_2=\begin{pmatrix}0\\0\\1\end{pmatrix}\begin{pmatrix}0&0&1\end{pmatrix}=\begin{pmatrix}0&0&0\\0&0&0\\0&0&1\end{pmatrix}\]
        \item \[D_3=\left(\frac{1}{\sqrt{6}}\right)^2\begin{pmatrix}1+i\\2\\0\end{pmatrix}\begin{pmatrix}1-i&2&0\end{pmatrix}=\frac{1}{6}\begin{pmatrix}2i&2+2i&0\\2-2i&4&0\\0&0&0\end{pmatrix}\]
        
    \end{itemize}
    
    De esta forma la descomposición espectral queda como:\
    \[D=\lambda_1D_1+\lambda_2D_2+\lambda_3D_3=0\frac{1}{3}\begin{pmatrix}2i&-1-i&0\\-1-i&1&0\\0&0&0\end{pmatrix}+\begin{pmatrix}0&0&0\\0&0&0\\0&0&1\end{pmatrix}+3\frac{1}{6}\begin{pmatrix}2i&2+2i&0\\2-2i&4&0\\0&0&0\end{pmatrix}\]\[=\begin{pmatrix}0&0&0\\0&0&0\\0&0&1\end{pmatrix}+\frac{1}{2}\begin{pmatrix}2i&2+2i&0\\2-2i&4&0\\0&0&0\end{pmatrix}=\begin{pmatrix}1&1+i&0\\ \:\:1-i&2&0\\ \:\:0&0&1\end{pmatrix}\]
\end{itemize}
\section{Sean $\sigma_1, \sigma_2, \sigma_3$ las matrices de Pauli vistas en clase. Verificar que:}

Primeramente vamos a definir las matrices de Pauli y despu\'es sustituiremos. Entonces por lo visto en clase: 
\[ \sigma_1 = \begin{pmatrix} 0 & 1 \\ 1 & 0 \end{pmatrix}\hspace{1.5cm}\sigma_2 = \begin{pmatrix} 0 & -i \\ i & 0 \end{pmatrix} \hspace{1.5cm} \sigma_3 = \begin{pmatrix} 1 & 0 \\ 0 & -1 \end{pmatrix}\]
\begin{itemize}
    \item [$a)$] $\sigma_1^2 = \sigma_2^2 =\sigma_3^2 = -i \sigma_1 \sigma_2 \sigma_3 = I$. (La matriz identidad). \\\\
    \textbf{Soluci\'on 11.a:}\\
    Primero elevaremos al cuadrado cada matriz, de manera que tenemos que: 

\[\sigma_1^2 = \begin{pmatrix} 0 & 1 \\ 1 & 0 \end{pmatrix}\begin{pmatrix} 0 & 1 \\ 1 & 0 \end{pmatrix}=\begin{pmatrix} 0(0)+1(1) & 0(1)+1(0) \\ 1(0)+0(1) & 1(1)+0(0) \end{pmatrix}=\begin{pmatrix} 1 & 0 \\ 0 & 1 \end{pmatrix}\]
\[\sigma_2^2 = \begin{pmatrix} 0 & -i \\ i & 0 \end{pmatrix}\begin{pmatrix} 0 & -i \\ i & 0 \end{pmatrix}=\begin{pmatrix} 0(0)+-i(i) & 0(-i)+-i(0) \\ i(0)+0(i) & i(-i)+0(0) \end{pmatrix}=\begin{pmatrix} 1 & 0 \\ 0 & 1 \end{pmatrix}\]
\[\sigma_3^2 = \begin{pmatrix} 1 & 0 \\ 0 & -1 \end{pmatrix}\begin{pmatrix} 1 & 0 \\ 0 & -1 \end{pmatrix}=\begin{pmatrix} 1(1)+0(0) & 1(0)+0(-1) \\ 0(1)+-1(0) & 0(0)+-1(-1) \end{pmatrix}=\begin{pmatrix} 1 & 0 \\ 0 & 1 \end{pmatrix}\]
Para seguir con la soluci\'on de este ejercicio lo que haremos ser\'a realizar la multiplicaci\'on \textbf{$-i \sigma_1 \sigma_2 \sigma_3$} y usamos que la multiplicaci\'on de matrices es asociativa, al igual que el producto por escalar:\\
\[-i \sigma_1 \sigma_2 \sigma_3=-i[ (\sigma_1 \sigma_2) \sigma_3]=-i\left[\left(\begin{pmatrix} 0 & 1\\ 1 & 0\end{pmatrix} \begin{pmatrix} 0 & -i \\ i & 0\end{pmatrix}\right) \begin{pmatrix} 1 & 0 \\ 0 & -1 \end{pmatrix}\right]=-i\left[\begin{pmatrix} 0(0)+1(i) & 0(-i)+1(0)\\ 1(0)+0(i) & 1(-i)+0(0)\end{pmatrix}\begin{pmatrix} 1 & 0 \\ 0 & -1 \end{pmatrix}\right]\]\[=-i\begin{pmatrix} i & 0\\ 0 & -i\end{pmatrix} \begin{pmatrix} 1 & 0 \\ 0 & -1 \end{pmatrix}=-i(i)\begin{pmatrix} 1 & 0\\ 0 & -1\end{pmatrix} \begin{pmatrix} 1 & 0 \\ 0 & -1 \end{pmatrix} = -(-1)\sigma_3^2=I\]
Pero por lo anterior sabemos que $\sigma_3^2=I$, por tanto podemos asegurar que:
\[\sigma_1^2 = \sigma_2^2 =\sigma_3^2 = -i \sigma_1 \sigma_2 \sigma_3 = I\]\qed
    
    
\item [$b)$] $\sigma_1 \sigma_2 + \sigma_2\sigma_1 = 0.$\\\\
    \textbf{Soluci\'on 11.b:}\\
De igual manera definiremos las matrices, sustituiremos y despu\'es realizaremos las operaciones requeridas, ya sea multiplicando, sumando o factorizando escalares para comprobar si se cumple el resultado deseado.

\[\sigma_1 \sigma_2 + \sigma_2\sigma_1= \begin{pmatrix} 0 & 1\\ 1 & 0
\end{pmatrix} \begin{pmatrix} 0 & -i \\ i & 0\end{pmatrix} + \begin{pmatrix} 0 & -i \\ i & 0 \end{pmatrix} \begin{pmatrix} 0 & 1 \\ 1 & 0 \end{pmatrix}= \begin{pmatrix} 0(0)+1(i) & 0(-i)+1(0)\\ 1(0)+0(i) & 1(-i)+0(0)\end{pmatrix}+\begin{pmatrix} 0(0)+1(-i) & 0(1)+-i(0)\\ i(0)+0(1) & i(1)+0(0)\end{pmatrix}\]\[=\begin{pmatrix} i & 0\\ 0 & -i\end{pmatrix}+\begin{pmatrix} -i & 0\\ 0 & +i\end{pmatrix}=\begin{pmatrix} i & 0 \\ 0 & -i\end{pmatrix} -\begin{pmatrix} i & 0 \\ 0 & -i \end{pmatrix} = \begin{pmatrix} 0 & 0 \\ 0 & 0\end{pmatrix}\]
\[\therefore \sigma_1 \sigma_2 + \sigma_2\sigma_1 = 0\]\qed
    
\item [$c)$] $\sigma_x\sigma_y=-\sigma_y\sigma_x=i\sigma_z, x,y,z \in {1,2,3}.$\\\\
    \textbf{Soluci\'on 11.c:}\\
Los casos que debemos demostrar son $(\sigma_1,\sigma_2), (\sigma_3,\sigma_1)$ y $(\sigma_2,\sigma_3)$, por lo tanto:
\begin{itemize}
    \item $\sigma_1\sigma_2=-\sigma_2\sigma_1=i\sigma_3$
    \[\sigma_1\sigma_2= \begin{pmatrix} 0 & 1\\ 1 & 0
\end{pmatrix} \begin{pmatrix} 0 & -i \\ i & 0\end{pmatrix}= \begin{pmatrix} 0(0)+1(i) & 0(-i)+1(0)\\ 1(0)+0(i) & 1(-i)+0(0)\end{pmatrix}=\begin{pmatrix} i & 0 \\ 0 & -i\end{pmatrix}\]
\[-\sigma_2\sigma_1=-\begin{pmatrix} 0 & -i \\ i & 0 \end{pmatrix} \begin{pmatrix} 0 & 1 \\ 1 & 0 \end{pmatrix}= -\begin{pmatrix} 0(0)+1(-i) & 0(1)+-i(0)\\ i(0)+0(1) & i(1)+0(0)\end{pmatrix}=-\begin{pmatrix} -i & 0\\ 0 & +i\end{pmatrix}=\begin{pmatrix} i & 0 \\ 0 & -i \end{pmatrix}\]
\[i\sigma_3 =i \begin{pmatrix} 1 & 0 \\ 0 & -1 \end{pmatrix}=\begin{pmatrix} i & 0 \\ 0 & -i \end{pmatrix}\]
\[\therefore \sigma_1\sigma_2=-\sigma_2\sigma_1=i\sigma_3\]
    \item $\sigma_3\sigma_1=-\sigma_1\sigma_3=i\sigma_2$
\[\sigma_3\sigma_1=\begin{pmatrix} 1 & 0 \\ 0 & -1\end{pmatrix} \begin{pmatrix} 0 & 1 \\ 1 & 0 \end{pmatrix}= \begin{pmatrix} 1(0)+0(1) & 1(1)+0(0)\\ 0(0)+1(-1) & 0(1)+0(-1)\end{pmatrix}=\begin{pmatrix} 0 & 1\\ -1 & 0\end{pmatrix}\]
 \[-\sigma_1\sigma_3= -\begin{pmatrix} 0 & 1\\ 1 & 0
\end{pmatrix} \begin{pmatrix} 1 & 0 \\ 0 & -1\end{pmatrix}= -\begin{pmatrix} 0(1)+1(0) & 0(0)+1(-1)\\ 1(1)+0(0) & 1(0)+0(-1)\end{pmatrix}=-\begin{pmatrix} 0 & -1 \\ 1 & 0 \end{pmatrix}=\begin{pmatrix} 0 & 1 \\ -1 & 0 \end{pmatrix}\]
\[i\sigma_2 =i \begin{pmatrix} 0 & -i \\ i & 0 \end{pmatrix}=\begin{pmatrix} 0 & 1 \\ -1 & 0 \end{pmatrix}\]
\[\therefore \sigma_3\sigma_1=-\sigma_1\sigma_3=i\sigma_2\]
    \item $\sigma_2\sigma_3=-\sigma_3\sigma_2=i\sigma_1$
\[\sigma_2\sigma_3=\begin{pmatrix} 0 & -i \\ i & 0 \end{pmatrix}  \begin{pmatrix} 1 & 0\\ 0 & -1
\end{pmatrix}= \begin{pmatrix} 1(0)+0(-i) & 0(0)+-i(-1)\\ i(1)+0(0) & i(0)+0(-1)\end{pmatrix}=\begin{pmatrix} 0 & i \\ i & 0 \end{pmatrix}\]
\[-\sigma_3\sigma_2=- \begin{pmatrix} 1 & 0\\ 0 & -1
\end{pmatrix} \begin{pmatrix} 0 & -i \\ i & 0\end{pmatrix}= -\begin{pmatrix} 1(0)+0(i) & 1(-i)+0(0)\\ 0(0)+-1(i) & 0(-i)+-1(0)\end{pmatrix}=-\begin{pmatrix} 0 & -i \\ -i & 0 \end{pmatrix}=\begin{pmatrix} 0 & i \\ i & 0 \end{pmatrix}\]
\[i\sigma_1 =i \begin{pmatrix} 0 & 1 \\ 1 & 0 \end{pmatrix}=\begin{pmatrix} 0 & i \\ i & 0 \end{pmatrix}\]
\[\therefore \sigma_2\sigma_3=-\sigma_3\sigma_2=i\sigma_1\]
\end{itemize}
\end{itemize}


%\[\sigma_x = \begin{pmatrix} 0 & 1 \\ 1 & 0 \end{pmatrix} ,  \sigma_y = \begin{pmatrix} 0 & -i \\ i & 0 \end{pmatrix} , \sigma_z = \begin{pmatrix} 1 & 0 \\ 0 & -1 \end{pmatrix}\]\[\Longrightarrow \sigma_x\sigma_y = \begin{pmatrix} 0 & 1\\ 1 & 0 \end{pmatrix} \begin{pmatrix} 0 & -i \\ i & 0\end{pmatrix} = \begin{pmatrix} i & 0 \\ 0 & -i\end{pmatrix}\]\[\Longrightarrow -\sigma_y\sigma_x = - \begin{pmatrix} 0 & -i \\ i & 0\end{pmatrix} \begin{pmatrix} 0 & 1\\ 1 & 0 \end{pmatrix} = \begin{pmatrix} i & 0 \\ 0 & -i\end{pmatrix}\]\[\Longrightarrow i\sigma_z = \begin{pmatrix} 1 & 0 \\ 0 & -1 \end{pmatrix} = \begin{pmatrix} i & 0 \\ 0 & -i \end{pmatrix} \]Por lo tanto podemos concluir que sucede que \textbf{s\'i} se cumple la igualdad. \[\sigma_x = \begin{pmatrix} 0 & -i \\ i & 0 \end{pmatrix} ,  \sigma_y = \begin{pmatrix} 1 & 0 \\ 0 & -1 \end{pmatrix} , \sigma_z = \begin{pmatrix} 0 & 1 \\ 1 & 0 \end{pmatrix}\] \[\Longrightarrow \sigma_x\sigma_y=\begin{pmatrix} 0 & -i \\ i & 0 \end{pmatrix}\begin{pmatrix} 1 & 0 \\ 0 & -1 \end{pmatrix}= \begin{pmatrix} 0 & -i \\ i & 0 \end{pmatrix}\]\[\Longrightarrow -\sigma_y\sigma_x= -\begin{pmatrix} 1 & 0 \\ 0 & -1 \end{pmatrix}\begin{pmatrix} 0 & -i \\ i & 0 \end{pmatrix}= \begin{pmatrix}0 & -i \\ -i & 0\end{pmatrix}\]\[\Longrightarrow i\sigma_z = i\begin{pmatrix} 0 & 1 \\ 1 & 0 \end{pmatrix} = \begin{pmatrix} 0 & i \\ i & 0 \end{pmatrix}\]Por lo tanto podemos concluir que en este caso en particular \textbf{no} se cumple.\[\sigma_x =\begin{pmatrix} 1 & 0 \\ 0 & -1 \end{pmatrix} ,  \sigma_y = \begin{pmatrix} 0 & 1 \\ 1 & 0 \end{pmatrix} , \sigma_z = \begin{pmatrix} 0 & -i \\ i & 0 \end{pmatrix}\] \[\Longrightarrow \sigma_x\sigma_y= \begin{pmatrix} 1 & 0 \\ 0 & -1 \end{pmatrix}\begin{pmatrix} 0 & 1 \\ 1 & 0 \end{pmatrix} = \begin{pmatrix} 0 & 1 \\ -1 & 0 \end{pmatrix}\]\[\Longrightarrow -\sigma_y\sigma_x= -\begin{pmatrix} 0 & 1 \\ 1 & 0 \end{pmatrix}\begin{pmatrix} 1 & 0 \\ 0 & -1 \end{pmatrix} = \begin{pmatrix}0 & 1 \\ -1 & 0 \end{pmatrix}\]\[\Longrightarrow i\sigma_z =i\begin{pmatrix} 0 & -i \\ i & 0 \end{pmatrix} = \begin{pmatrix} 0 & 1 \\ -1 & 0 \end{pmatrix}\]Por lo tanto podemos concluir que en este caso en particular \textbf{s\'i} se cumple. \[\sigma_x =\begin{pmatrix} 0 & 1 \\ 1 & 0 \end{pmatrix} ,  \sigma_y = \begin{pmatrix} 1 & 0 \\ 0 & -1 \end{pmatrix} , \sigma_z = \begin{pmatrix} 0 & -i \\ i & 0 \end{pmatrix}\]\[\Longrightarrow \sigma_x\sigma_y= \begin{pmatrix} 0 & 1 \\ 1 & 0 \end{pmatrix}\begin{pmatrix} 1 & 0 \\ 0 & -1 \end{pmatrix} = \begin{pmatrix} 0 & -1 \\ 1 & 0 \end{pmatrix}\]\[\Longrightarrow -\sigma_y\sigma_x= -\begin{pmatrix} 1 & 0 \\ 0 & -1 \end{pmatrix}\begin{pmatrix} 0 & 1 \\ 1 & 0 \end{pmatrix} = \begin{pmatrix} 0 & -1 \\ 1 & 0 \end{pmatrix}\]\[\Longrightarrow i\sigma_z = i\begin{pmatrix} 0 & -i \\ i & 0 \end{pmatrix} = \begin{pmatrix} 0 & 1 \\ -1 & 0 \end{pmatrix}\]Por lo tanto podemos concluir que en este caso en particular \textbf{no} se cumple.\[\sigma_x = \begin{pmatrix} 1 & 0 \\ 0 & -1 \end{pmatrix},  \sigma_y = \begin{pmatrix} 0 & -i \\ i & 0 \end{pmatrix}, \sigma_z = \begin{pmatrix} 0 & 1 \\ 1 & 0 \end{pmatrix}\]\[\Longrightarrow \sigma_x\sigma_y= \begin{pmatrix} 1 & 0 \\ 0 & -1 \end{pmatrix}\begin{pmatrix} 0 & -i \\ i & 0 \end{pmatrix} = \begin{pmatrix}0 & -i \\ -i & 0 \end{pmatrix}\]\[\Longrightarrow -\sigma_y\sigma_x= -\begin{pmatrix} 0 & -i \\ i & 0 \end{pmatrix}\begin{pmatrix} 1 & 0 \\ 0 & -1 \end{pmatrix}=\begin{pmatrix}0 & -i \\ -i & 0 \end{pmatrix}\]\[\Longrightarrow i\sigma_z = i\begin{pmatrix} 0 & 1 \\ 1 & 0 \end{pmatrix}=\begin{pmatrix}0 & i \\ i & 0 \end{pmatrix}\]Por lo tanto podemos concluir que en este caso en particular \textbf{no} se cumple.\[\sigma_x = \begin{pmatrix} 0 & -i \\ i & 0 \end{pmatrix}  ,\sigma_y = \begin{pmatrix} 0 & 1 \\ 1 & 0 \end{pmatrix} ,\sigma_z = \begin{pmatrix} 1 & 0 \\ 0 & -1 \end{pmatrix}\] \[\Longrightarrow \sigma_x\sigma_y= \begin{pmatrix} 0 & -i \\ i & 0 \end{pmatrix}\begin{pmatrix} 0 & 1 \\ 1 & 0 \end{pmatrix} = \begin{pmatrix}-i & 0 \\ 0 & i \end{pmatrix}\]\[\Longrightarrow -\sigma_y\sigma_x= -\begin{pmatrix} 0 & 1 \\ 1 & 0 \end{pmatrix}\begin{pmatrix} 0 & -i \\ i & 0 \end{pmatrix} = \begin{pmatrix}-i & 0 \\ 0 & i \end{pmatrix}\]\[\Longrightarrow i\sigma_z= i\begin{pmatrix} 1 & 0 \\ 0 & -1 \end{pmatrix} = \begin{pmatrix}i & 0 \\ 0 & -i\end{pmatrix}\]Por lo tanto podemos concluir que en este caso en particular \textbf{no} se cumple.



\end{document}
