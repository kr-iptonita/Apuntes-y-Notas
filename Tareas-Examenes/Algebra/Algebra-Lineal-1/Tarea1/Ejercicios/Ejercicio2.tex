\section{Demuestre que $\mathbb{Q}(i) = {a + bi | a, b \in \mathbb{Q}}$ es un subcampo de $\mathbb{C}$.}

\begin{itemize}
	\item $\mathbb{Q}(i)$ es un subanillo de $\mathbb{C}$.
	\item $\mathbb{Q}(i)$ contiene la identidad aditiva de $\mathbb{C}$.
	\item Cada elemento no nulo en $\mathbb{Q}(i)$ tiene un inverso multiplicativo en $\mathbb{Q}(i)$.
\end{itemize}

\subsection{Demostrar que $\mathbb{Q}(i)$ es un subanillo de $\mathbb{C}$}

\begin{itemize}
	\item Suma: Sean $z_1 = a_1 + b_1i y z_2 = a_2 + b_2i$ dos elementos en $\mathbb{Q}(i)$, entonces su suma es $z_1 + z_2 = (a_1 + a_2) + (b_1 + b_2)i$, que es un número complejo en $\mathbb{Q}(i)$.
	\item Resta: De manera similar, si $z_1$ y $z_2$ son elementos en $\mathbb{Q}(i)$, entonces su diferencia es $z_1 - z_2 = (a_1 - a_2) + (b_1 - b_2)i$, que también es un número complejo en $\mathbb{Q}(i)$.
	\item Producto: Si $z_1$ y $z_2$ son elementos en $\mathbb{Q}(i)$, entonces su producto es $z_1z_2 = (a_1a_2 - b_1b_2) + (a_1b_2 + a_2b_1)i$, que es un número complejo en $\mathbb{Q}(i)$
\end{itemize}

$\therefore \mathbb{Q}(i)$ es un subanillo de $\mathbb{C}$

\subsection{$\mathbb{Q}(i)$ contiene la identidad aditiva de $\mathbb{C}$}

Es $0 + 0i$. Por lo tanto, esta condición se cumple.

\subsection{Cada elemento no nulo en $\mathbb{Q}(i)$ tiene un inverso multiplicativo en $\mathbb{Q}(i)$.}

Cada elemento no nulo en $\mathbb{Q}(i)$ tiene un inverso multiplicativo en $\mathbb{Q}(i)$. Si $z = a + bi$ es un elemento no nulo en $\mathbb{Q}(i)$, entonces su inverso multiplicativo es $z^{-1} = 1/(a+bi)$. Para encontrar el inverso, podemos multiplicar el numerador y el denominador por el conjugado de $z$, que es $a - bi$.

$$\Rightarrow z^{-1} = (1/(a+bi)) * ((a-bi)/(a-bi)) = (a - bi)/(a^{2} + b^{2})$$

Como $a$ y $b$ son racionales, $a^2 + b^2$ es un número racional positivo. Por lo tanto, $(a - bi)/(a^2 + b^2)$ es un número complejo en $\mathbb{Q}(i)$.

Por lo tanto, $\mathbb{Q}(i)$ cumple las tres condiciones para ser un subcampo de $\mathbb{C}$.