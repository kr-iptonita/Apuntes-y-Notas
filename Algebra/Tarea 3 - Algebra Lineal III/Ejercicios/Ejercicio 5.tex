\section{Sean $V = \mathcal{M}_2(\mathbb{R})$, $T:V\longrightarrow V$, dado por $T(X) = X^T$, su transpuesta.}
\begin{itemize}
    \item [$a)$] Prueba que $T$ es lineal.\\\\
    \textbf{Demostraci\'on 5.a:}\\
    Como $\mathcal{M}_2(\mathbb{R})$ ambos es un $\mathbb{R}$-espacio vectorial, para demostrar que $T$ es una transformaci\'on lineal, debemos probar que:
\begin{itemize}
    \item $T(A+B) = T(A) + T(B)$ para cada $A,B \in \mathcal{M}_2(\mathbb{R})$ . ($T$ es aditiva).\\
    Sean $A$ y $B$ matrices de $\mathcal{M}_2(\mathbb{R})$, primero calculamos \[T(A)=A^T~~~~\text{y}~~~~T(B)=B^T\]
    Por tanto:
    \[T(A)+T(B)=A^T+B^T\]
    Ahora calculamos:
    \[T(A+B)=(A+B)^T\]
    Por lo que por las mismas propiedades de la transpuesta, sabemos que se abre a sumas, por tanto:
    \[T(A)+T(B)=A^T+B^T=(A+B)^T=T(A+B)\]
    \[\therefore T(A+B) = T(A) + T(B)\]
    Por lo que $T$ es aditiva.
    
 \item  $T(\alpha A) = \alpha T(A)$ para cada $A \in \mathcal{M}_2(\mathbb{R})$ y $\alpha \in \mathbb{R}$. ($T$ es $\mathbb{R}$-lineal).\\
 Sean $A\in \mathcal{M}_2(\mathbb{R})$ una matriz arbitaria y $\alpha\in \mathbb{R}$ un escalar arbitrario, primero calculamos \[\alpha T(A)=\alpha (A)^T\]
    Ahora calculamos:
    \[T(\alpha A)=(\alpha A)^T\]
    Pero por la propiedad de la transpuesta, sabemos que podemos factorizar el escalar de una tranpuesta, de modo que:
    \[T(\alpha A) =(\alpha A)^T=\alpha(A)^T=\alpha T(A)\]
    \[\therefore T(\alpha A) =\alpha T(A)\]
    Por lo que $T$ es $\mathbb{R}-$lineal.

\end{itemize}
Como de ambos puntos, $T: \mathcal{M}_2(\mathbb{R}) \rightarrow \mathcal{M}_2(\mathbb{R})$ definida como $T(A)=A^T$ cumple ambos puntos, entonces $T$ es una transformaci\'on lineal, que adem\'as es un endomorfismo.\qed
    
    \item [$b)$] Calcula la matriz $A$ de $T$ relativa a la base canónica de $V$.\\\\
    \textbf{Soluci\'on 5.b:}\\
    Sabemos que para calcular la matriz $A$ asociada a $T$, lo primero que debemos hacer es plantear una base can\'onica para nuestro espacio vectorial $\mathcal{M}_2(\mathbb{R})$, la cual podemos definir de acuerdo a las matrices can\'onicas como:
    \[\mathcal{B}=\left\{E_{1\hspace{0.5mm}1}=e_1=\begin{pmatrix}
    1 & 0\\
    0 & 0
    \end{pmatrix}, E_{1\hspace{0.5mm}2}=e_2=\begin{pmatrix}
    0 & 1\\
    0 & 0
    \end{pmatrix}, E_{2\hspace{0.5mm}1}=e_3=\begin{pmatrix}
    0 & 0\\
    1 & 0
    \end{pmatrix}, E_{2\hspace{0.5mm}2}=e_4=\begin{pmatrix}
    0 & 0\\
    0 & 1
    \end{pmatrix}\right\}\]
    Ahora, sabemos que para determinar $A$ debemos definir cada columna como el vector coordenada de la matriz aplicada a cada elemento de la base (con la numeraci\'on correspodiente), respecto a la base $\mathcal{B}$, es decir:
    \[A=\begin{pmatrix} [T(e_1)]_{\mathcal{B}} & [T(e_2)]_{\mathcal{B}} &[T(e_3)]_{\mathcal{B}} &[T(e_4)]_{\mathcal{B}} \end{pmatrix}\]
    Por lo tanto, encontrando cada vector coordenada, tenemos:
    \begin{itemize}
        \item \[T(e_1)=\begin{pmatrix}
    1 & 0\\
    0 & 0
    \end{pmatrix}^T=\begin{pmatrix}
    1 & 0\\
    0 & 0
    \end{pmatrix}=e_1=1\cdot e_1+0\cdot e_2+0\cdot e_3+0\cdot e_4\]
    \[\therefore [T(e_1)]_{\mathcal{B}}=\begin{pmatrix} 1 & 0 & 0 & 0\end{pmatrix}\]
    
    \item \[T(e_2)=\begin{pmatrix}
    0 & 1\\
    0 & 0
    \end{pmatrix}^T=\begin{pmatrix}
    0 & 0\\
    1 & 0
    \end{pmatrix}=e_3=0\cdot e_1+0\cdot e_2+1\cdot e_3+0\cdot e_4\]
    \[\therefore [T(e_2)]_{\mathcal{B}}=\begin{pmatrix} 0 & 0 & 1 & 0\end{pmatrix}\]
    
    \item \[T(e_3)=\begin{pmatrix}
    0 & 0\\
    1 & 0
    \end{pmatrix}^T=\begin{pmatrix}
    0 & 1\\
    0 & 0
    \end{pmatrix}=e_2=0\cdot e_1+1\cdot e_2+0\cdot e_3+0\cdot e_4\]
    \[\therefore [T(e_3)]_{\mathcal{B}}=\begin{pmatrix} 0 & 1 & 0 & 0\end{pmatrix}\]
    
    \item \[T(e_4)=\begin{pmatrix}
    0 & 0\\
    0 & 1
    \end{pmatrix}^T=\begin{pmatrix}
    0 & 0\\
    0 & 1
    \end{pmatrix}=e_4=0\cdot e_1+0\cdot e_2+0\cdot e_3+1\cdot e_4\]
    \[\therefore [T(e_1)]_{\mathcal{B}}=\begin{pmatrix} 0 & 0 & 0 & 1\end{pmatrix}\]
    \end{itemize}
    De modo que finalmente coloc\'andolos como columna en la matriz obtenemos:
    \[A= \begin{pmatrix} [T(e_1)]_{\mathcal{B}} & [T(e_2)]_{\mathcal{B}} &[T(e_3)]_{\mathcal{B}} &[T(e_4)]_{\mathcal{B}} \end{pmatrix}=\begin{pmatrix}
    1 & 0 & 0 & 0\\
    0 & 0 & 1 & 0\\
    0 & 1 & 0 & 0\\
    0 & 0 & 0 & 1\end{pmatrix}\]
    
    \item [$c)$] Calcula los valores propios de $T$, los espacios propios, sus dimensiones y una base de cada uno.\\\\
    \textbf{Soluci\'on 5.c:}\\
    Lo primero que se deber\'a hacer es encontrar su polinomio caracter\'istico:
        \[0=\text{det}(A-\lambda_nI_{n\times n })=\left|\begin{pmatrix}
    1 & 0 & 0 & 0\\
    0 & 0 & 1 & 0\\
    0 & 1 & 0 & 0\\
    0 & 0 & 0 & 1\end{pmatrix}-\lambda\begin{pmatrix}
    1 & 0 & 0 & 0\\
    0 & 1 & 0 & 0\\
    0 & 0 & 1 & 0\\
    0 & 0 & 0 & 1\end{pmatrix}\right|=\begin{vmatrix}
    1-\lambda & 0 & 0 & 0\\
    0 & -\lambda & 1 & 0\\
    0 & 1 & -\lambda & 0\\
    0 & 0 & 0 & 1-\lambda\end{vmatrix}\] Como sabemos que una matriz equivalente por filas tiene el mismo determinante, si asumimos que $\lambda \neq 0$, podemos sumar $\displaystyle \frac{1}{\lambda}$ veces el segundo rengl\'on al tercero, de modo que tengamos una matriz escalonada y asi multiplicar la diagonal:
    \[=\begin{vmatrix}
    1-\lambda & 0 & 0 & 0\\
    0 & -\lambda & 1 & 0\\
    0 & 1+\frac{1}{\lambda}(-\lambda) & -\lambda+\frac{1}{\lambda}(1) & 0\\
    0 & 0 & 0 & 1-\lambda\end{vmatrix}=\begin{vmatrix}
    1-\lambda & 0 & 0 & 0\\
    0 & -\lambda & 1 & 0\\
    0 & 0 & -\lambda+\frac{1}{\lambda} & 0\\
    0 & 0 & 0 & 1-\lambda\end{vmatrix}=(1-\lambda)(-\lambda)\left(-\lambda+\frac{1}{\lambda}\right)(1-\lambda)\]\[=(1-\lambda)^2\left(+\lambda^2+\frac{-\lambda}{\lambda}\right)=(\lambda-1)^2(\lambda^2-1)=(\lambda-1)(\lambda-1)(\lambda-1)(\lambda+1)\]
    \[\therefore (\lambda-1)(\lambda-1)(\lambda-1)(\lambda+1)=0\]
    
    Por lo que tenemos que $\lambda_1=-1$ y $\lambda_2=\lambda_3=\lambda_4=1$ (multiplicidad de 3), por lo que tenemos que $\text{ma}(-1)=1$ y $\text{ma}(1)=3$.\\
    Ahora, usando la \textbf{Definici\'on 20} vista en clase, como $T$ es un endomorfismo y tenemos su matriz asociada $A$ respecto de alguna base can\'onica, calculamos cada uno de los subespacios propios correspondientes a $\lambda_i$ (subconjunto que consta de los vectores
propios), con $i=1,2,3,4$:
\begin{itemize}
    \item Para $\lambda_1=-1$, sabemos que $E(1)=\text{Nuc}(A-(-1)I)$, por lo que sustituyendo en la ecuaci\'on $(A-(-1)I)\vec{x}=(0)$, con $x=(a,b,c,d)$, tenemos:
    \[(A-(-1)I)\vec{x}=\begin{pmatrix}
    1-(-1) & 0 & 0 & 0\\
    0 & -(-1) & 1 & 0\\
    0 & 1 & -(-1) & 0\\
    0 & 0 & 0 & 1-(-1)\end{pmatrix}\begin{pmatrix}a\\b\\c\\d\end{pmatrix}=\begin{pmatrix}
    2 & 0 & 0 & 0\\
    0 & 1 & 1 & 0\\
    0 & 1 & 1 & 0\\
    0 & 0 & 0 & 2\end{pmatrix}\begin{pmatrix}a\\b\\c\\d\end{pmatrix}\]\[=\begin{pmatrix}2(a)+0(b)+0(c)+0(d)\\0(a)+1(b)+1(c)+0(d)\\0(a)+1(b)+1(c)+0(d)\\0(a)+0(b)+0(c)+2(d)\end{pmatrix}=\begin{pmatrix}2a\\b+c\\b+c\\2d\end{pmatrix}=\begin{pmatrix}0\\0\\0\\0\end{pmatrix}\]
    Por lo que tenemos el siguiente sistema de ecuaciones:
\begin{eqnarray*}
2a&=&0\\
b+c&=&0\\
b+c&=&0\\
2d&=&0
\end{eqnarray*}
De la primera y cuarta ecuaci\'on tenemos que $a=d=0$, mientras que despejando de la segunda y tercera, tenemos que $c=-b$, por lo que tenemos que $b$ es una variable libre, por lo que tenemos que $\begin{pmatrix}0&b&-b&0\end{pmatrix}\in E(-1)$, de tal forma que $E(-1)=\{(0,\alpha,-\alpha,0)~|~\alpha\in\mathbb{R}\}$.
Por tanto, si fijamos $(0,\alpha,-\alpha,0)=\alpha(0,1,-1,0)$, tenemos que $\langle(0,1,-1,0)\rangle=E(-1)$ y como es un \'unico vector es linealmente independiente, por lo que $\text{dim}(E(-1))=1$.


\item Para $\lambda_2=\lambda_3=\lambda_4=1$ (con multiplicidad algebraica 3), sabemos que $E(1)=\text{Nuc}(A-1I)$, por lo que sustituyendo en la ecuaci\'on $(A-1I)\vec{x}=(0)$, con $x=(a,b,c,d)$, tenemos:
    \[(A-1I)\vec{x}=\begin{pmatrix}
    1-1 & 0 & 0 & 0\\
    0 & -1 & 1 & 0\\
    0 & 1 & -1 & 0\\
    0 & 0 & 0 & 1-1\end{pmatrix}\begin{pmatrix}a\\b\\c\\d\end{pmatrix}=\begin{pmatrix}
    0 & 0 & 0 & 0\\
    0 & -1 & 1 & 0\\
    0 & 1 & -1 & 0\\
    0 & 0 & 0 & 0\end{pmatrix}\begin{pmatrix}a\\b\\c\\d\end{pmatrix}\]\[=\begin{pmatrix}0(a)+0(b)+0(c)+0(d)\\0(a)+(-1)(b)+1(c)+0(d)\\0(a)+1(b)+(-1)(c)+0(d)\\0(a)+0(b)+0(c)+0(d)\end{pmatrix}=\begin{pmatrix}0\\-b+c\\b-c\\0\end{pmatrix}=\begin{pmatrix}0\\0\\0\\0\end{pmatrix}\]
    Por lo que tenemos el siguiente sistema de ecuaciones:
\begin{eqnarray*}
0&=&0\\
-b+c&=&0\\
b-c&=&0\\
0&=&0
\end{eqnarray*}
Despejando de la segunda y tercera, tenemos que $c=b$, por lo que tenemos que $b$ es una variable libre, pero al tampoco tener restricciones para $a$ y $d$, tam bien las podemos tratar como variables libres, por lo que tenemos que $\begin{pmatrix}a&b&b&d\end{pmatrix}\in E(1)$, de tal forma que $E(1)=\{(\beta,\alpha,\alpha,\gamma)~|~\alpha,\beta,\gamma\in\mathbb{R}\}$.\\
Es f\'acil ver que los vectores $(1,0,0,0)$, $(0,1,1,0)$ y $(0,0,0,1)$ son base de $E(1)$, pues si tomamos un vector arbitrario $(\beta,\alpha,\alpha,\gamma)\in E(1)$, tenemos que:
\[(\beta,\alpha,\alpha,\gamma)=\beta(1,0,0,0)+\alpha(0,1,1,0)+\gamma(0,0,0,1)\]
y adem\'as son linealmente independientes, pues:
\[(\beta,\alpha,\alpha,\gamma)=\beta(1,0,0,0)+\alpha(0,1,1,0)+\gamma(0,0,0,1)=(0,0,0,0)\]
Esto pasa si y solo si $\alpha=\beta=\gamma=0$, entonces es obvio que los tres vectores son base de $E(1)$, por lo que $\text{dim}(E(1))=3$.
\end{itemize}
    
    \item [$d)$] ¿Es $T$ es diagonalizable?\\\\
    \textbf{Respuesta 5.d:}\\
    Usando el \textbf{Teorema 23} visto en clase, tenemos que comprobar los siguientes puntos:
    \begin{itemize}
        \item El polinomio característico de $T$ tiene todas sus raíces en $K=\mathbb{R}$.\\
        Esto es verdad, pues encontramos que $\lambda_1=-1$ y $\lambda_2=1$, por lo que ambas pertenecen a $\mathbb{R}$
\item Para cada valor propio $\lambda$ se verifica $\text{dim}(E(\lambda))=\text{ma}(\lambda)$.\\
Esto tambi\'en es verdad, pues encontramos que:
\[\text{dim}(E(-1))=1=\text{ma}(-1)\]Y\[\text{dim}(E(1))=3=\text{ma}(1)\]

    \end{itemize}
    Por lo que como ambos puntos se cumplen, tenemos que efectivamente $T$ es diagonizable.
    
\end{itemize}