\section{Elije $\lambda$ de tal modo que el siguiente sistema de ecuaciones tenga solución:
\begin{align*}
2x_1 - x_2 + x_3 + x_4 &= 1 \\
x_1 + 2x_2 - x_3 + 4x_4 &= 2 \\
x_1 + 7x_2 - 4x_3 + 11x_4 &= \lambda
\end{align*}}

Podemos representar el sistema de ecuaciones lineales utilizando matrices de la siguiente manera:

\[ AX = B \]

donde:

\[ A = \begin{bmatrix} 
2 & -1 & 1 & 1 \\
1 & 2 & -1 & 4 \\
1 & 7 & -4 & 11
\end{bmatrix} \]

\[ X = \begin{bmatrix} 
x_1 \\
x_2 \\
x_3 \\
x_4
\end{bmatrix} \]

\[ B = \begin{bmatrix} 
1 \\
2 \\
\lambda
\end{bmatrix} \]

Para que el sistema tenga solución, es necesario que la matriz de coeficientes A sea una matriz invertible, es decir, su determinante debe ser distinto de cero.

Entonces, para encontrar el valor de lambda que hace que el sistema tenga solución, podemos calcular el determinante de la matriz A y resolver la ecuación:

\[ \det(A) \neq 0 \]

Si el determinante de A es diferente de cero, entonces para cualquier valor de lambda, el sistema tendrá una única solución. Si el determinante es igual a cero, entonces existen valores particulares de lambda que hacen que el sistema tenga múltiples soluciones o no tenga solución.

Vamos a calcular el determinante de A:

\[ \det(A) = \begin{bmatrix} 
2 & -1 & 1 & 1 \\
1 & 2 & -1 & 4 \\
1 & 7 & -4 & 11
\end{bmatrix} \]

Utilizando operaciones elementales de filas, podemos simplificar la matriz A:

\[ \begin{bmatrix} 
2 & -1 & 1 & 1 \\
1 & 2 & -1 & 4 \\
1 & 7 & -4 & 11
\end{bmatrix} 
= \begin{bmatrix} 
2 & -1 & 1 & 1 \\
0 & 5 & -3 & 3 \\
0 & 8 & -5 & 10
\end{bmatrix} 
= \begin{bmatrix} 
2 & -1 & 1 & 1 \\
0 & 5 & -3 & 3 \\
0 & 0 & 1 & 4
\end{bmatrix} \]

Ahora, calculamos el determinante de la matriz resultante:

\[ \begin{bmatrix} 
2 & -1 & 1 & 1 \\
0 & 5 & -3 & 3 \\
0 & 0 & 1 & 4
\end{bmatrix} 
= 2 \cdot 5 \cdot 1 + (-1) \cdot (-3) \cdot 1 + 1 \cdot 3 \cdot 0 - 1 \cdot 5 \cdot 0 - (-1) \cdot 3 \cdot 0 - 2 \cdot (-3) \cdot 1
= 10 + 3 + 0 - 0 - 0 - (-6)
= 19 \]

El determinante de la matriz A es igual a 19. Como el determinante es diferente de cero, podemos concluir que para cualquier valor de lambda, el sistema tendrá una única solución.
Podemos representar el sistema de ecuaciones lineales utilizando matrices de la siguiente manera:

\[ AX = B \]

donde:

\[ A = \begin{bmatrix} 
2 & -1 & 1 & 1 \\
1 & 2 & -1 & 4 \\
1 & 7 & -4 & 11
\end{bmatrix} \]

\[ X = \begin{bmatrix} 
x_1 \\
x_2 \\
x_3 \\
x_4
\end{bmatrix} \]

\[ B = \begin{bmatrix} 
1 \\
2 \\
\lambda
\end{bmatrix} \]

Para que el sistema tenga solución, es necesario que la matriz de coeficientes A sea una matriz invertible.
Entonces
\[ \det(A) \neq 0 \]


Calculando el determinante de A:

\[ \det(A) = \begin{bmatrix} 
2 & -1 & 1 & 1 \\
1 & 2 & -1 & 4 \\
1 & 7 & -4 & 11
\end{bmatrix} \]

Simplificando la matriz A:

\[ \begin{bmatrix} 
2 & -1 & 1 & 1 \\
1 & 2 & -1 & 4 \\
1 & 7 & -4 & 11
\end{bmatrix} 
= \begin{bmatrix} 
2 & -1 & 1 & 1 \\
0 & 5 & -3 & 3 \\
0 & 8 & -5 & 10
\end{bmatrix} 
= \begin{bmatrix} 
2 & -1 & 1 & 1 \\
0 & 5 & -3 & 3 \\
0 & 0 & 1 & 4
\end{bmatrix} \]

Ahora, calculamos el determinante de la matriz resultante:

\[ \begin{bmatrix} 
2 & -1 & 1 & 1 \\
0 & 5 & -3 & 3 \\
0 & 0 & 1 & 4
\end{bmatrix} 
= 2 \cdot 5 \cdot 1 + (-1) \cdot (-3) \cdot 1 + 1 \cdot 3 \cdot 0 - 1 \cdot 5 \cdot 0 - (-1) \cdot 3 \cdot 0 - 2 \cdot (-3) \cdot 1
= 10 + 3 + 0 - 0 - 0 - (-6)
= 19 \]


El determinante de la matriz A es igual a 19. Como el determinante es diferente de cero.
Se puede ver como la tercera ecuación es redundante o no factible dado que es combinación de las dos primeras.
Concretamente es la combinacion de 3 veces la segunda fila menos la primera.

Siguiendo esto podemos decir que $\lambda=3(2)-1=5$.

