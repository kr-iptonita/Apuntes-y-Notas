\section{Sean A, B matrices de $n \times n$ con entradas en $\mathbb{R}$ demuestre que $AB-BA \neq I_n$}

Para demostrar que $AB - BA \neq I_n$ para matrices $A$ y $B$ de tamaño $n \times n$ con entradas en $\mathbb{R}$, podemos considerar un contraejemplo. 

Supongamos que $A$ y $B$ son matrices que satisfacen $AB - BA = I_n$, es decir, la diferencia entre $AB$ y $BA$ es la matriz identidad $I_n$. Vamos a demostrar que esta suposición lleva a una contradicción.

Multiplicando ambos lados de la ecuación por $A$, obtenemos:

$$A(AB - BA) = AI_n$$
$$AAB - ABA = A$$

Ahora, multiplicamos por $B$ en ambos lados de la ecuación:

$$(AAB - ABA)B = AB$$
$$AABB - ABAB = AB$$

En este punto, podemos reescribir la ecuación en términos de trazas:

$$\text{tr}(AABB) - \text{tr}(ABAB) = \text{tr}(AB)$$

Aplicando la propiedad cíclica de la traza, podemos reorganizar el lado izquierdo de la ecuación:

$$\text{tr}(AABB) - \text{tr}(ABAB) = \text{tr}(AB)$$
$$\text{tr}(ABAB) - \text{tr}(AABB) = -\text{tr}(AB)$$

Sin embargo, esto implica que $-\text{tr}(AB) = \text{tr}(AB)$, lo cual es una contradicción, ya que la traza de una matriz no nula no puede ser igual a su negativo.

Por lo tanto, hemos llegado a una contradicción a partir de nuestra suposición inicial. Por lo tanto, podemos concluir que $AB - BA \neq I_n$ para matrices $A$ y $B$ de tamaño $n \times n$ con entradas en $\mathbb{R}$.
