\section{Sean $\sigma_1, \sigma_2, \sigma_3$ las matrices de Pauli vistas en clase. Verificar que:}

Primeramente vamos a definir las matrices de Pauli y despu\'es sustituiremos. Entonces por lo visto en clase: 
\[ \sigma_1 = \begin{pmatrix} 0 & 1 \\ 1 & 0 \end{pmatrix}\hspace{1.5cm}\sigma_2 = \begin{pmatrix} 0 & -i \\ i & 0 \end{pmatrix} \hspace{1.5cm} \sigma_3 = \begin{pmatrix} 1 & 0 \\ 0 & -1 \end{pmatrix}\]
\begin{itemize}
    \item [$a)$] $\sigma_1^2 = \sigma_2^2 =\sigma_3^2 = -i \sigma_1 \sigma_2 \sigma_3 = I$. (La matriz identidad). \\\\
    \textbf{Soluci\'on 11.a:}\\
    Primero elevaremos al cuadrado cada matriz, de manera que tenemos que: 

\[\sigma_1^2 = \begin{pmatrix} 0 & 1 \\ 1 & 0 \end{pmatrix}\begin{pmatrix} 0 & 1 \\ 1 & 0 \end{pmatrix}=\begin{pmatrix} 0(0)+1(1) & 0(1)+1(0) \\ 1(0)+0(1) & 1(1)+0(0) \end{pmatrix}=\begin{pmatrix} 1 & 0 \\ 0 & 1 \end{pmatrix}\]
\[\sigma_2^2 = \begin{pmatrix} 0 & -i \\ i & 0 \end{pmatrix}\begin{pmatrix} 0 & -i \\ i & 0 \end{pmatrix}=\begin{pmatrix} 0(0)+-i(i) & 0(-i)+-i(0) \\ i(0)+0(i) & i(-i)+0(0) \end{pmatrix}=\begin{pmatrix} 1 & 0 \\ 0 & 1 \end{pmatrix}\]
\[\sigma_3^2 = \begin{pmatrix} 1 & 0 \\ 0 & -1 \end{pmatrix}\begin{pmatrix} 1 & 0 \\ 0 & -1 \end{pmatrix}=\begin{pmatrix} 1(1)+0(0) & 1(0)+0(-1) \\ 0(1)+-1(0) & 0(0)+-1(-1) \end{pmatrix}=\begin{pmatrix} 1 & 0 \\ 0 & 1 \end{pmatrix}\]
Para seguir con la soluci\'on de este ejercicio lo que haremos ser\'a realizar la multiplicaci\'on \textbf{$-i \sigma_1 \sigma_2 \sigma_3$} y usamos que la multiplicaci\'on de matrices es asociativa, al igual que el producto por escalar:\\
\[-i \sigma_1 \sigma_2 \sigma_3=-i[ (\sigma_1 \sigma_2) \sigma_3]=-i\left[\left(\begin{pmatrix} 0 & 1\\ 1 & 0\end{pmatrix} \begin{pmatrix} 0 & -i \\ i & 0\end{pmatrix}\right) \begin{pmatrix} 1 & 0 \\ 0 & -1 \end{pmatrix}\right]=-i\left[\begin{pmatrix} 0(0)+1(i) & 0(-i)+1(0)\\ 1(0)+0(i) & 1(-i)+0(0)\end{pmatrix}\begin{pmatrix} 1 & 0 \\ 0 & -1 \end{pmatrix}\right]\]\[=-i\begin{pmatrix} i & 0\\ 0 & -i\end{pmatrix} \begin{pmatrix} 1 & 0 \\ 0 & -1 \end{pmatrix}=-i(i)\begin{pmatrix} 1 & 0\\ 0 & -1\end{pmatrix} \begin{pmatrix} 1 & 0 \\ 0 & -1 \end{pmatrix} = -(-1)\sigma_3^2=I\]
Pero por lo anterior sabemos que $\sigma_3^2=I$, por tanto podemos asegurar que:
\[\sigma_1^2 = \sigma_2^2 =\sigma_3^2 = -i \sigma_1 \sigma_2 \sigma_3 = I\]\qed
    
    
\item [$b)$] $\sigma_1 \sigma_2 + \sigma_2\sigma_1 = 0.$\\\\
    \textbf{Soluci\'on 11.b:}\\
De igual manera definiremos las matrices, sustituiremos y despu\'es realizaremos las operaciones requeridas, ya sea multiplicando, sumando o factorizando escalares para comprobar si se cumple el resultado deseado.

\[\sigma_1 \sigma_2 + \sigma_2\sigma_1= \begin{pmatrix} 0 & 1\\ 1 & 0
\end{pmatrix} \begin{pmatrix} 0 & -i \\ i & 0\end{pmatrix} + \begin{pmatrix} 0 & -i \\ i & 0 \end{pmatrix} \begin{pmatrix} 0 & 1 \\ 1 & 0 \end{pmatrix}= \begin{pmatrix} 0(0)+1(i) & 0(-i)+1(0)\\ 1(0)+0(i) & 1(-i)+0(0)\end{pmatrix}+\begin{pmatrix} 0(0)+1(-i) & 0(1)+-i(0)\\ i(0)+0(1) & i(1)+0(0)\end{pmatrix}\]\[=\begin{pmatrix} i & 0\\ 0 & -i\end{pmatrix}+\begin{pmatrix} -i & 0\\ 0 & +i\end{pmatrix}=\begin{pmatrix} i & 0 \\ 0 & -i\end{pmatrix} -\begin{pmatrix} i & 0 \\ 0 & -i \end{pmatrix} = \begin{pmatrix} 0 & 0 \\ 0 & 0\end{pmatrix}\]
\[\therefore \sigma_1 \sigma_2 + \sigma_2\sigma_1 = 0\]\qed
    
\item [$c)$] $\sigma_x\sigma_y=-\sigma_y\sigma_x=i\sigma_z, x,y,z \in {1,2,3}.$\\\\
    \textbf{Soluci\'on 11.c:}\\
Los casos que debemos demostrar son $(\sigma_1,\sigma_2), (\sigma_3,\sigma_1)$ y $(\sigma_2,\sigma_3)$, por lo tanto:
\begin{itemize}
    \item $\sigma_1\sigma_2=-\sigma_2\sigma_1=i\sigma_3$
    \[\sigma_1\sigma_2= \begin{pmatrix} 0 & 1\\ 1 & 0
\end{pmatrix} \begin{pmatrix} 0 & -i \\ i & 0\end{pmatrix}= \begin{pmatrix} 0(0)+1(i) & 0(-i)+1(0)\\ 1(0)+0(i) & 1(-i)+0(0)\end{pmatrix}=\begin{pmatrix} i & 0 \\ 0 & -i\end{pmatrix}\]
\[-\sigma_2\sigma_1=-\begin{pmatrix} 0 & -i \\ i & 0 \end{pmatrix} \begin{pmatrix} 0 & 1 \\ 1 & 0 \end{pmatrix}= -\begin{pmatrix} 0(0)+1(-i) & 0(1)+-i(0)\\ i(0)+0(1) & i(1)+0(0)\end{pmatrix}=-\begin{pmatrix} -i & 0\\ 0 & +i\end{pmatrix}=\begin{pmatrix} i & 0 \\ 0 & -i \end{pmatrix}\]
\[i\sigma_3 =i \begin{pmatrix} 1 & 0 \\ 0 & -1 \end{pmatrix}=\begin{pmatrix} i & 0 \\ 0 & -i \end{pmatrix}\]
\[\therefore \sigma_1\sigma_2=-\sigma_2\sigma_1=i\sigma_3\]
    \item $\sigma_3\sigma_1=-\sigma_1\sigma_3=i\sigma_2$
\[\sigma_3\sigma_1=\begin{pmatrix} 1 & 0 \\ 0 & -1\end{pmatrix} \begin{pmatrix} 0 & 1 \\ 1 & 0 \end{pmatrix}= \begin{pmatrix} 1(0)+0(1) & 1(1)+0(0)\\ 0(0)+1(-1) & 0(1)+0(-1)\end{pmatrix}=\begin{pmatrix} 0 & 1\\ -1 & 0\end{pmatrix}\]
 \[-\sigma_1\sigma_3= -\begin{pmatrix} 0 & 1\\ 1 & 0
\end{pmatrix} \begin{pmatrix} 1 & 0 \\ 0 & -1\end{pmatrix}= -\begin{pmatrix} 0(1)+1(0) & 0(0)+1(-1)\\ 1(1)+0(0) & 1(0)+0(-1)\end{pmatrix}=-\begin{pmatrix} 0 & -1 \\ 1 & 0 \end{pmatrix}=\begin{pmatrix} 0 & 1 \\ -1 & 0 \end{pmatrix}\]
\[i\sigma_2 =i \begin{pmatrix} 0 & -i \\ i & 0 \end{pmatrix}=\begin{pmatrix} 0 & 1 \\ -1 & 0 \end{pmatrix}\]
\[\therefore \sigma_3\sigma_1=-\sigma_1\sigma_3=i\sigma_2\]
    \item $\sigma_2\sigma_3=-\sigma_3\sigma_2=i\sigma_1$
\[\sigma_2\sigma_3=\begin{pmatrix} 0 & -i \\ i & 0 \end{pmatrix}  \begin{pmatrix} 1 & 0\\ 0 & -1
\end{pmatrix}= \begin{pmatrix} 1(0)+0(-i) & 0(0)+-i(-1)\\ i(1)+0(0) & i(0)+0(-1)\end{pmatrix}=\begin{pmatrix} 0 & i \\ i & 0 \end{pmatrix}\]
\[-\sigma_3\sigma_2=- \begin{pmatrix} 1 & 0\\ 0 & -1
\end{pmatrix} \begin{pmatrix} 0 & -i \\ i & 0\end{pmatrix}= -\begin{pmatrix} 1(0)+0(i) & 1(-i)+0(0)\\ 0(0)+-1(i) & 0(-i)+-1(0)\end{pmatrix}=-\begin{pmatrix} 0 & -i \\ -i & 0 \end{pmatrix}=\begin{pmatrix} 0 & i \\ i & 0 \end{pmatrix}\]
\[i\sigma_1 =i \begin{pmatrix} 0 & 1 \\ 1 & 0 \end{pmatrix}=\begin{pmatrix} 0 & i \\ i & 0 \end{pmatrix}\]
\[\therefore \sigma_2\sigma_3=-\sigma_3\sigma_2=i\sigma_1\]
\end{itemize}
\end{itemize}


%\[\sigma_x = \begin{pmatrix} 0 & 1 \\ 1 & 0 \end{pmatrix} ,  \sigma_y = \begin{pmatrix} 0 & -i \\ i & 0 \end{pmatrix} , \sigma_z = \begin{pmatrix} 1 & 0 \\ 0 & -1 \end{pmatrix}\]\[\Longrightarrow \sigma_x\sigma_y = \begin{pmatrix} 0 & 1\\ 1 & 0 \end{pmatrix} \begin{pmatrix} 0 & -i \\ i & 0\end{pmatrix} = \begin{pmatrix} i & 0 \\ 0 & -i\end{pmatrix}\]\[\Longrightarrow -\sigma_y\sigma_x = - \begin{pmatrix} 0 & -i \\ i & 0\end{pmatrix} \begin{pmatrix} 0 & 1\\ 1 & 0 \end{pmatrix} = \begin{pmatrix} i & 0 \\ 0 & -i\end{pmatrix}\]\[\Longrightarrow i\sigma_z = \begin{pmatrix} 1 & 0 \\ 0 & -1 \end{pmatrix} = \begin{pmatrix} i & 0 \\ 0 & -i \end{pmatrix} \]Por lo tanto podemos concluir que sucede que \textbf{s\'i} se cumple la igualdad. \[\sigma_x = \begin{pmatrix} 0 & -i \\ i & 0 \end{pmatrix} ,  \sigma_y = \begin{pmatrix} 1 & 0 \\ 0 & -1 \end{pmatrix} , \sigma_z = \begin{pmatrix} 0 & 1 \\ 1 & 0 \end{pmatrix}\] \[\Longrightarrow \sigma_x\sigma_y=\begin{pmatrix} 0 & -i \\ i & 0 \end{pmatrix}\begin{pmatrix} 1 & 0 \\ 0 & -1 \end{pmatrix}= \begin{pmatrix} 0 & -i \\ i & 0 \end{pmatrix}\]\[\Longrightarrow -\sigma_y\sigma_x= -\begin{pmatrix} 1 & 0 \\ 0 & -1 \end{pmatrix}\begin{pmatrix} 0 & -i \\ i & 0 \end{pmatrix}= \begin{pmatrix}0 & -i \\ -i & 0\end{pmatrix}\]\[\Longrightarrow i\sigma_z = i\begin{pmatrix} 0 & 1 \\ 1 & 0 \end{pmatrix} = \begin{pmatrix} 0 & i \\ i & 0 \end{pmatrix}\]Por lo tanto podemos concluir que en este caso en particular \textbf{no} se cumple.\[\sigma_x =\begin{pmatrix} 1 & 0 \\ 0 & -1 \end{pmatrix} ,  \sigma_y = \begin{pmatrix} 0 & 1 \\ 1 & 0 \end{pmatrix} , \sigma_z = \begin{pmatrix} 0 & -i \\ i & 0 \end{pmatrix}\] \[\Longrightarrow \sigma_x\sigma_y= \begin{pmatrix} 1 & 0 \\ 0 & -1 \end{pmatrix}\begin{pmatrix} 0 & 1 \\ 1 & 0 \end{pmatrix} = \begin{pmatrix} 0 & 1 \\ -1 & 0 \end{pmatrix}\]\[\Longrightarrow -\sigma_y\sigma_x= -\begin{pmatrix} 0 & 1 \\ 1 & 0 \end{pmatrix}\begin{pmatrix} 1 & 0 \\ 0 & -1 \end{pmatrix} = \begin{pmatrix}0 & 1 \\ -1 & 0 \end{pmatrix}\]\[\Longrightarrow i\sigma_z =i\begin{pmatrix} 0 & -i \\ i & 0 \end{pmatrix} = \begin{pmatrix} 0 & 1 \\ -1 & 0 \end{pmatrix}\]Por lo tanto podemos concluir que en este caso en particular \textbf{s\'i} se cumple. \[\sigma_x =\begin{pmatrix} 0 & 1 \\ 1 & 0 \end{pmatrix} ,  \sigma_y = \begin{pmatrix} 1 & 0 \\ 0 & -1 \end{pmatrix} , \sigma_z = \begin{pmatrix} 0 & -i \\ i & 0 \end{pmatrix}\]\[\Longrightarrow \sigma_x\sigma_y= \begin{pmatrix} 0 & 1 \\ 1 & 0 \end{pmatrix}\begin{pmatrix} 1 & 0 \\ 0 & -1 \end{pmatrix} = \begin{pmatrix} 0 & -1 \\ 1 & 0 \end{pmatrix}\]\[\Longrightarrow -\sigma_y\sigma_x= -\begin{pmatrix} 1 & 0 \\ 0 & -1 \end{pmatrix}\begin{pmatrix} 0 & 1 \\ 1 & 0 \end{pmatrix} = \begin{pmatrix} 0 & -1 \\ 1 & 0 \end{pmatrix}\]\[\Longrightarrow i\sigma_z = i\begin{pmatrix} 0 & -i \\ i & 0 \end{pmatrix} = \begin{pmatrix} 0 & 1 \\ -1 & 0 \end{pmatrix}\]Por lo tanto podemos concluir que en este caso en particular \textbf{no} se cumple.\[\sigma_x = \begin{pmatrix} 1 & 0 \\ 0 & -1 \end{pmatrix},  \sigma_y = \begin{pmatrix} 0 & -i \\ i & 0 \end{pmatrix}, \sigma_z = \begin{pmatrix} 0 & 1 \\ 1 & 0 \end{pmatrix}\]\[\Longrightarrow \sigma_x\sigma_y= \begin{pmatrix} 1 & 0 \\ 0 & -1 \end{pmatrix}\begin{pmatrix} 0 & -i \\ i & 0 \end{pmatrix} = \begin{pmatrix}0 & -i \\ -i & 0 \end{pmatrix}\]\[\Longrightarrow -\sigma_y\sigma_x= -\begin{pmatrix} 0 & -i \\ i & 0 \end{pmatrix}\begin{pmatrix} 1 & 0 \\ 0 & -1 \end{pmatrix}=\begin{pmatrix}0 & -i \\ -i & 0 \end{pmatrix}\]\[\Longrightarrow i\sigma_z = i\begin{pmatrix} 0 & 1 \\ 1 & 0 \end{pmatrix}=\begin{pmatrix}0 & i \\ i & 0 \end{pmatrix}\]Por lo tanto podemos concluir que en este caso en particular \textbf{no} se cumple.\[\sigma_x = \begin{pmatrix} 0 & -i \\ i & 0 \end{pmatrix}  ,\sigma_y = \begin{pmatrix} 0 & 1 \\ 1 & 0 \end{pmatrix} ,\sigma_z = \begin{pmatrix} 1 & 0 \\ 0 & -1 \end{pmatrix}\] \[\Longrightarrow \sigma_x\sigma_y= \begin{pmatrix} 0 & -i \\ i & 0 \end{pmatrix}\begin{pmatrix} 0 & 1 \\ 1 & 0 \end{pmatrix} = \begin{pmatrix}-i & 0 \\ 0 & i \end{pmatrix}\]\[\Longrightarrow -\sigma_y\sigma_x= -\begin{pmatrix} 0 & 1 \\ 1 & 0 \end{pmatrix}\begin{pmatrix} 0 & -i \\ i & 0 \end{pmatrix} = \begin{pmatrix}-i & 0 \\ 0 & i \end{pmatrix}\]\[\Longrightarrow i\sigma_z= i\begin{pmatrix} 1 & 0 \\ 0 & -1 \end{pmatrix} = \begin{pmatrix}i & 0 \\ 0 & -i\end{pmatrix}\]Por lo tanto podemos concluir que en este caso en particular \textbf{no} se cumple.

