%%%%%%%%%%%%%%%%%%%%%%%%%%%%% Define Article %%%%%%%%%%%%%%%%%%%%%%%%%%%%%%%%%%
\documentclass{article}
%%%%%%%%%%%%%%%%%%%%%%%%%%%%%%%%%%%%%%%%%%%%%%%%%%%%%%%%%%%%%%%%%%%%%%%%%%%%%%%

%%%%%%%%%%%%%%%%%%%%%%%%%%%%% Using Packages %%%%%%%%%%%%%%%%%%%%%%%%%%%%%%%%%%
\usepackage{geometry}
\usepackage{graphicx}
\usepackage{amssymb}
\usepackage{amsmath}
\usepackage{amsthm}
\usepackage{empheq}
\usepackage{mdframed}
\usepackage{booktabs}
\usepackage{lipsum}
\usepackage{graphicx}
\usepackage{color}
\usepackage{psfrag}
\usepackage{pgfplots}
\usepackage{bm}
%%%%%%%%%%%%%%%%%%%%%%%%%%%%%%%%%%%%%%%%%%%%%%%%%%%%%%%%%%%%%%%%%%%%%%%%%%%%%%%

% Other Settings

%%%%%%%%%%%%%%%%%%%%%%%%%% Page Setting %%%%%%%%%%%%%%%%%%%%%%%%%%%%%%%%%%%%%%%
\geometry{a4paper}

%%%%%%%%%%%%%%%%%%%%%%%%%% Define some useful colors %%%%%%%%%%%%%%%%%%%%%%%%%%
\definecolor{ocre}{RGB}{243,102,25}
\definecolor{mygray}{RGB}{243,243,244}
\definecolor{deepGreen}{RGB}{26,111,0}
\definecolor{shallowGreen}{RGB}{235,255,255}
\definecolor{deepBlue}{RGB}{61,124,222}
\definecolor{shallowBlue}{RGB}{235,249,255}
%%%%%%%%%%%%%%%%%%%%%%%%%%%%%%%%%%%%%%%%%%%%%%%%%%%%%%%%%%%%%%%%%%%%%%%%%%%%%%%

%%%%%%%%%%%%%%%%%%%%%%%%%% Define an orangebox command %%%%%%%%%%%%%%%%%%%%%%%%
\newcommand\orangebox[1]{\fcolorbox{ocre}{mygray}{\hspace{1em}#1\hspace{1em}}}
%%%%%%%%%%%%%%%%%%%%%%%%%%%%%%%%%%%%%%%%%%%%%%%%%%%%%%%%%%%%%%%%%%%%%%%%%%%%%%%

%%%%%%%%%%%%%%%%%%%%%%%%%%%% English Environments %%%%%%%%%%%%%%%%%%%%%%%%%%%%%
\newtheoremstyle{mytheoremstyle}{3pt}{3pt}{\normalfont}{0cm}{\rmfamily\bfseries}{}{1em}{{\color{black}\thmname{#1}~\thmnumber{#2}}\thmnote{\,--\,#3}}
\newtheoremstyle{myproblemstyle}{3pt}{3pt}{\normalfont}{0cm}{\rmfamily\bfseries}{}{1em}{{\color{black}\thmname{#1}~\thmnumber{#2}}\thmnote{\,--\,#3}}
\theoremstyle{mytheoremstyle}
\newmdtheoremenv[linewidth=1pt,backgroundcolor=shallowGreen,linecolor=deepGreen,leftmargin=0pt,innerleftmargin=20pt,innerrightmargin=20pt,]{theorem}{Theorem}[section]
\theoremstyle{mytheoremstyle}
\newmdtheoremenv[linewidth=1pt,backgroundcolor=shallowBlue,linecolor=deepBlue,leftmargin=0pt,innerleftmargin=20pt,innerrightmargin=20pt,]{definition}{Definition}[section]
\theoremstyle{myproblemstyle}
\newmdtheoremenv[linecolor=black,leftmargin=0pt,innerleftmargin=10pt,innerrightmargin=10pt,]{problem}{Problem}[section]
%%%%%%%%%%%%%%%%%%%%%%%%%%%%%%%%%%%%%%%%%%%%%%%%%%%%%%%%%%%%%%%%%%%%%%%%%%%%%%%

%%%%%%%%%%%%%%%%%%%%%%%%%%%%%%% Plotting Settings %%%%%%%%%%%%%%%%%%%%%%%%%%%%%
\usepgfplotslibrary{colorbrewer}
\pgfplotsset{width=8cm,compat=1.9}
%%%%%%%%%%%%%%%%%%%%%%%%%%%%%%%%%%%%%%%%%%%%%%%%%%%%%%%%%%%%%%%%%%%%%%%%%%%%%%%

%%%%%%%%%%%%%%%%%%%%%%%%%%%%%%% Title & Author %%%%%%%%%%%%%%%%%%%%%%%%%%%%%%%%
\title{Tarea 1}
\author{Número de cuenta: 318013712\\
Nombre Social: Juárez Torres Karla Romina\\
Nombre Legal : Juárez Torres  \\
Profesor: Pascual Di Bella Nava}
%%%%%%%%%%%%%%%%%%%%%%%%%%%%%%%%%%%%%%%%%%%%%%%%%%%%%%%%%%%%%%%%%%%%%%%%%%%%%%%

\begin{document}
    \maketitle
Para los siguientes modelos definir el Lagrangiano y calcular las ecuaciones de Euler-Lagrange. Graficar la variación de la posición y velocidad con respecto del tiempo, así como el retrato de fase de ambos sistemas. Verificar si los modelos son análogos o no. Proponer una situación de uso para el modelo 1.

\section{Masa entre dos resortes sostenidos entre dos muros inertes.}
La energía cinética $ T $ de la masa es $ \frac{1}{2} m \dot{x}^2 $

Para un resorte, la energía potencial elástica $ U_i $ es $ \frac{1}{2} k_i x^2 $, donde $ k_i $ es la constante del resorte $ i $.

Por lo tanto, el Lagrangiano $ L $ está dado por:

\[ L = T - U = \frac{1}{2} m \dot{x}^2 - \left( \frac{1}{2} k_1 x^2 + \frac{1}{2} k_2 (-x)^2 \right) \]

\[ L = \frac{1}{2} m \dot{x}^2 - \frac{1}{2} (k_1 + k_2) x^2 \]

Euler-Lagrange:

\[ \frac{d}{dt} \left( \frac{\partial L}{\partial \dot{x}} \right) - \frac{\partial L}{\partial x} = 0 \]

\[ \frac{\partial L}{\partial \dot{x}} = m \dot{x} \]

\[ \frac{d}{dt} \left( \frac{\partial L}{\partial \dot{x}} \right) = m \ddot{x} \]

\[ \frac{\partial L}{\partial x} = -(k_1 + k_2) x \]

Entonces:

\[ m \ddot{x} + (k_1 + k_2) x = 0 \]

De este modo tenemos que:
\[ \Omega_0^2 = \dis(k_1 +playstyle\frac{k_1 + k_2}{m} = 0 \]


Usuaria este modelo para programar el autocentrado de mira de un videojuego de simulación de conducción.


\section{Masa con resortes en paralelo anclados a un muro inerte.}

La energía cinética $ T $ de la masa sigue siendo $ \frac{1}{2} m \dot{x}^2 $

Para este caso la energía potencial $ U $ del sistema resulta ser la suma de las energías potenciales elásticas de los dos resortes:

\[ U = \frac{1}{2} k_1 x_1^2 + \frac{1}{2} k_2 x_2^2 \]

Donde $ x_1 $ y $ x_2 $ son las extensiones de los resortes $ k_1 $ y $ k_2 $, respectivamente.

Por lo tanto, el Lagrangiano $ L $ está dado por:

\[ L = T - U = \frac{1}{2} m \dot{x}^2 - \left( \frac{1}{2} k_1 x_1^2 + \frac{1}{2} k_2 x_2^2 \right) \]

Dado que los resortes están en paralelo, la extensión total $ x $ de la masa es la suma de las extensiones de los dos resortes:

\[ x = x_1 + x_2 \]

La energía cinética $ T $ sigue siendo la misma, y el Lagrangiano se puede reescribir en términos de $ x $ y $ \dot{x} $.

Euler-Lagrange:


\[ \frac{\partial L}{\partial \dot{x}} = m \dot{x} \]

\[ \frac{\partial L}{\partial x_1} = -k_1 x_1 \]

\[ \frac{\partial L}{\partial x_2} = -k_2 x_2 \]

Derivada total con respecto al tiempo de $ \frac{\partial L}{\partial \dot{x}} $:

\[ \frac{d}{dt} \left( \frac{\partial L}{\partial \dot{x}} \right) = m \ddot{x} \]

Sustituyendo:

Para $ x_1 $:

\[ \frac{d}{dt} \left( \frac{\partial L}{\partial \dot{x}} \right) - \frac{\partial L}{\partial x_1} = 0 \]

\[ m \ddot{x} + k_1 x_1 = 0 \]

Para $ x_2 $:

\[ \frac{d}{dt} \left( \frac{\partial L}{\partial \dot{x}} \right) - \frac{\partial L}{\partial x_2} = 0 \]

\[ m \ddot{x} + k_2 x_2 = 0 \]

Sumando estas dos ecuaciones, obtenemos la ecuación de movimiento para la masa $ m $:

\[ m \ddot{x} + (k_1 + k_2) x = 0 \]

De este modo tenemos que:
\[ \Omega_0^2 = \displaystyle\frac{k_1 + k_2}{m} = 0 \]

Como podemos notar lo resultante es exactamente lo mismo al Modelo 1.


\section{Graficos}
\subsection*{Los modelos fueron análogos así que los graficos se aplican a ambos.}

Los graficos fueron realizados con \textsc{Python} en entorno local sin embargo se añade notebook de jupyter para visualizar las graficas resultantes. 




\end{document}
