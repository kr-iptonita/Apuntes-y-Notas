\section{Halle los valores de $p$ y $q$ de forma tal que $\mu (x,y) = x^p y^q$ sea un factor integrante de la ecuación diferencial y resuelva la ecuación. $$(\frac{x}{y^2}-\frac{1}{xy})dx-\frac{1}{y^2}dy=0$$}

Si $\mu (x,y) = x^p y^q$ es un factor integrante, entonces al multiplicar la ecuación diferencial obtenemos que $$(x^{p+1}y^{q-2}-x^{p-1}y^{q-1})dx-(x^py^{q-2})dy=0$$ entonces,
$$M(x,y)=x^{p+1}y^{q-2}-x^{p-1}y^{q-1}$$
$$\frac{\partial M}{\partial y}=(q-2)x^{p+1}y^{q-3}-(q-1)x^{p-1}y^{q-2}$$
$$N(x,y)=-x^py^{q-2}$$
$$\frac{\partial N}{\partial x}=-px^{p-1}y^{q-2}$$
para que sea una ecuación exacta, debe cumplir que
$$\frac{\partial M}{\partial y}=\frac{\partial N}{\partial x}$$ es decir, 
$$(q-2)x^{p+1}y^{q-3}-(q-1)x^{p-1}y^{q-2}=-px^{p-1}y^{q-2}$$
Igualando los coeficientes obtenemos lo siguiente
$$(q-2)=0 \Rightarrow q=2$$
$$-(q-1)=-p \Rightarrow -(2-1)=-p \Rightarrow -1=-p \Rightarrow p=1$$
Con lo anterior encontramos los valores para $p$ y $q$, los cuales son
$$p=1$$
$$q=2$$
Por lo tanto, el factor integrante esta dado por: $$\mu (x,y) = xy^2$$
Ahora resolvemos la ecuación con el factor integrante.\\
Multiplicando la ecuación por el factor integrante, obtenemos
$$(x^2-y)dx-xdy=0$$
Podemos ver que cumple con las condiciones de una ecuación exacta, ya que
$$\frac{\partial M}{\partial y}=(x^2-y)=-1$$
$$\frac{\partial N}{\partial x}=(-x)=-1$$
$$\frac{\partial F}{\partial x}=(x^2-y)=-1$$
entonces, tenemos que
$$F(x,y)=\frac{1}{3}x^3-xy-g(y)$$
$$\frac{\partial F}{\partial y}=-x+g'(y)=-x$$
Por lo cual 
$$g(y)=C_{1}$$
$$g'(y)=0$$
Por lo tanto, la solución de la EDO queda de la siguiente manera
$$\frac{1}{3} x^3-xy=C$$
