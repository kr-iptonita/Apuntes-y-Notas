\documentclass{homework}
\usepackage[T1]{fontenc}
\usepackage{array}
\usepackage{multirow}


\author{Frank Garcia Trillo}
\class{Teoría de Juegos}
\date{\today}
\title{Tarea 1}
\address{Facultad de Ciencias.}

\graphicspath{{./media/}}

\begin{document} \maketitle


\question EVALUE las siguientes afirmaciones (determine si es verdadera o falsa, bajo qué condiciones es verdadera o falsa, sea explícito y claro en su punto de vista; si lo cree necesario, incluya ejemplos)

\begin{enumerate}
    \item Todo juego en forma estratégica tiene al menos un equilibrio de Nash.
    \item Para motivar las cooperación de los rivales en el juego del \textit{Dilema del prisionero}, basta con que en los pagos se introduzcan premios y se eliminen los castigos.
    \item En el juego de de la batalla de los sexos, la cantidad de Equilibrio de Nash dependerá de la intensidad de la relación de los jugadores.
    \item Si cada jugador elige la estrategia que maximiza su pago, considerando que el resto también hace lo propio, entonces el equilibrio resultante es \textit{Pareto Eficiente}.
\end{enumerate}


1(A) .- \textbf{Contraejemplo: El juego del peso}:
Dos jugadores, cada uno elige sol o águila, la decision es simultanea. Si las decisiones difieren, el \textbf{jugador 1} pagaria 1 peso al \textbf{jugador 2}. En caso contrario, el \textbf{jugador 2} pagaria 1 peso al jugador 1.

El juego de forma estrategica podemos verlo como:

\begin{equation}
    N = \{ J1, J2 \},  \\
    D_i = D_j = \{ Aguila, Sol \},  \\
    D = \{ (A,A), (A,S), (S,A), (A,A) \} 
\end{equation}

Notemos como se observa esta matriz de pagos.
\centering
\begin{tabular}{ll}
\hline
\multicolumn{2}{|c|}{J1} \\ \hline
C & NC \\ \hline
\hline
    \multirow{2}{}{J2}& C\\
    & NC\\ \hline

(1,-1) & (-1,1) \\
(1,-1)  & (1,-1) 
\end{tabular}

De este juego podemos afirmar que no existe ningun equilibrio de Nash. Por lo tanto, no todo juego puede tener un equilibrio de Nash.






Consideremos la matriz de pagos del juego del \textit{Dilema del prisionero}, donde los pagos dependeran de la versión del juego que estemos jugando. Probemos varias combinaciones. 

Observemos primero la combinación que implica solo castigos, uno reducido para quien confiesa.

\centering
\begin{tabular}{ll}
\hline
\multicolumn{2}{|c|}{J1} \\ \hline
C & NC \\ \hline
\hline
    \multirow{2}{}{J2}& C\\
    & NC\\ \hline

(-1,-1) & (0,-3) \\
(-3,0)  & (0,0) 
\end{tabular}

En este juego, los equilibrios de Nash, $EN = \{ (C,C) (NC, NC) \}$. Esto es logico, pues si solo tratamos de no llevarnos el peor de los castigos, ambos confesarian tratando de no verse perjudicados y entonces nos llevariamos el equilibrio de $(-1, -1)$. 

Luego, en la combinación donde hay premios y castigos. Notemos como se observa esta matriz de pagos.
\centering
\begin{tabular}{ll}
\hline
\multicolumn{2}{|c|}{J1} \\ \hline
C & NC \\ \hline
\hline
    \multirow{2}{|r|}{J2}& C\\
    & NC\\ \hline

(1,1) & (2,-1) \\
(-1,2)  & (0,0) 
\end{tabular}

Veamos ahora dos combinaciones donde solo hay premios:
\centering
\begin{tabular}{ll}

\hline
\multicolumn{2}{|c|}{J1} \\ \hline
C & NC \\ \hline
\hline
    \multirow{2}{|r|}{J2}& C\\
    & NC\\ \hline

(1,1) & (0,3) \\
(3,0)  & (0,0) 
\end{tabular}

Y la segunda es \\

\centering
\begin{tabular}{ll}
\hline
\multicolumn{2}{|c|}{J1} \\ \hline
C & NC \\ \hline
\hline
    \multirow{2}{|r|}{J2}& C\\
    & NC\\ \hline

(0,0) & (2,0) \\
(0,2)  & (1,1) 
\end{tabular}

De la primera tabla, los equilibrios de Nash (EN), son $EN_{Castigo} = \{(C,C), (NC,NC)\}$. De las otras tablas, los EN son:


\begin{equation}
    EN_{Premios 1} = \{(C,NC), (NC, C)\} \\
    EN_{Premios 2} = \{(C,C), (NC, C), (C, NC)\} \\
    EN_{Premios y castigos} = \{(C,C)\}
\end{equation}

Podemos ver que:





\question Considere el popular juego de “piedra (Pi), papel (Pa) o tijeras (T)”: de forma simultánea dos
jugadores representan con su mano uno de estos tres objetos; \textit{si son diferentes los objetos
necesariamente un jugador gana, y el otro pierde; si son iguales, se empata el juego. La matriz de
pagos se representa en la siguiente tabla:}

Determine los elementos de forma estrategica y los equilibrios de Nash.

\emph{Conjunto de jugadores $N$}: $\{ J_1, J_2\}$

\emph{Conjunto de estrategias}: $D_{J_1} = D_{J_2} = \{ Ti, Pi, Pa\}$

$D$ = $\{ (Pi, Pi), (Pa, Pa), (Ti, Ti), (Pi, Pa), (Pi, Ti), (Pa, Pi), (Pa, Ti), (Ti, Pa), (Ti, Pi)\}$


\emph{Funciones de pago: $\phi_i $}: 
\begin{equation}
\begin{bmatrix}
(0,0) & (-1,1) & (1,-1)\\
(1,-1) & (0,0) & (-1,1)\\
(-1,1) & (1,-1) & (0,0)
\end{bmatrix}
\end{equation}


\emph{Equilibrios de Nash $(EN)$}: $\empty$

Si analizamos el metodo de eliminación de estrategias optimas, encontramos que el conjunto de equilibrios de Nash es vacio. Por como esta definido el juego de Piedra, Papel y Tijeras, podemos ver que no existe una estrategia pura optima, pues toda estrategia que deje fija una elección para un jugador, tiene una respuesta que puede hacerle empatar o perder.


\question Considere la situación planteada en clase acerca de la problematica del país occidental ($PO$), y los migrantes dentro del pais ($M$)

    $PO$ : País occidental que recibe migrantes y encuentra que éstos no se integran del todo en su cultura y sociedad.
    $M$ : Gente que ha migrado a $PO$ y que no estan decididos del todo a integrarse a la sociedad de $PO$.

Los conjuntos de estrategias de cada jugador serian: 

$D_{PO}: \{ tolera (T), acosa (A)\}$
$D_M : \{ \text{Se integra} I, \text{No se integra} NI \}$

Determine dos escenarios distintos para este juego (dos matrices de pagos distintas), en donde
prevalezca una perspectiva del conflicto: tolerante, xenofóbico, humanitario, con soluciones
aparentes para los que están fuera del conflicto, etc., o la perspectiva que usted quiera proponer.
Explique la perspectiva elegida y obtenga, para ambas versiones, los elementos del juego en forma
estratégica (conjunto de jugadores, conjuntos de estrategias para cada jugador, funciones de pago)
y los equilibrios de Nash.


    \textbf{Escenario 1:} \emph{Inclusión}

\begin{table}[]
\begin{tabular}{|ll|ll|}
\hline
\multicolumn{2}{|l|}{\multirow{2}{*}{}} & \multicolumn{2}{l|}{PO} \\ \cline{3-4} 
\multicolumn{2}{|l|}{} & \multicolumn{1}{l|}{Ti} & A \\ \hline
\multicolumn{1}{|c|}{\multirow{2}{*}{M}} & I & \multicolumn{1}{l|}{(2,2)} & (0,2) \\ \cline{2-4} 
\multicolumn{1}{|c|}{} & NI & \multicolumn{1}{l|}{(0,0)} & (-1,-1) \\ \hline
\end{tabular}
\end{table}


    \textbf{Escenario 2:} \emph{Hipocresía }

    
\begin{table}[]
\begin{tabular}{|ll|ll|}
\hline
\multicolumn{2}{|l|}{\multirow{2}{*}{}} & \multicolumn{2}{l|}{PO} \\ \cline{3-4} 
\multicolumn{2}{|l|}{} & \multicolumn{1}{l|}{Ti} & A \\ \hline
\multicolumn{1}{|c|}{\multirow{2}{*}{M}} & I & \multicolumn{1}{l|}{(2,1)} & (-1,0) \\ \cline{2-4} 
\multicolumn{1}{|c|}{} & NI & \multicolumn{1}{l|}{(1,0)} & (-2,1) \\ \hline
\end{tabular}
\end{table}



    \textbf{Escenario 3:} \emph{Xenofobia}
\begin{table}[]
\begin{tabular}{|ll|ll|}
\hline
\multicolumn{2}{|l|}{\multirow{2}{*}{}} & \multicolumn{2}{l|}{PO} \\ \cline{3-4} 
\multicolumn{2}{|l|}{} & \multicolumn{1}{l|}{Ti} & A \\ \hline
\multicolumn{1}{|c|}{\multirow{2}{*}{M}} & I & \multicolumn{1}{l|}{(0,-1)} & (-1,1) \\ \cline{2-4} 
\multicolumn{1}{|c|}{} & NI & \multicolumn{1}{l|}{(0,-2)} & (-3,3) \\ \hline
\end{tabular}
\end{table}


    \textbf{Escenario 4:} \emph{Etnia}
\begin{table}[]
\begin{tabular}{|ll|ll|}
\hline
\multicolumn{2}{|l|}{\multirow{2}{*}{}} & \multicolumn{2}{l|}{PO} \\ \cline{3-4} 
\multicolumn{2}{|l|}{} & \multicolumn{1}{l|}{Ti} & A \\ \hline
\multicolumn{1}{|c|}{\multirow{2}{*}{M}} & I & \multicolumn{1}{l|}{(1,1)} & (-1,-1) \\ \cline{2-4} 
\multicolumn{1}{|c|}{} & NI & \multicolumn{1}{l|}{(2,0)} & (-2,-2) \\ \hline
\end{tabular}
\end{table}



\emph{Conjunto de jugadores $N$}: $\{ M, PO\}$

\emph{Conjunto de estrategias}:
\begin{equation}
    D_M =  \{ I, NI\} \\
    D_{PO} = \{ T, A\}
\end{equation}

$D$ = $\{ (I, T), (I, A), (NI, T), (NI, A) \}$


\begin{equation}
    EN_{E_1} = \{ (I,T)\} \\
    EN_{E_2} = \{ (I,T)\}  \\
    EN_{E_3} = \{ (I,A)\} \\
    EN_{E_4} = \{ (NI,T)\} 
\end{equation}



% citations
\bibliographystyle{plain}
\bibliography{citations}

\end{document}