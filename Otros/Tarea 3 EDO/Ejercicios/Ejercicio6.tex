\textcolor{blue}{Ejercicio 6.} Considere la función
\begin{equation}
    f(t)=e^{-3t}\int_{0}^{t}u\sin(2u)du
\end{equation}
Encuentre $\mathscr{L}\{f(t)\}$.\\

Primero resolvemos la integral por partes donde $f=u \Rightarrow df=du$ \\
y $dg=\sin(2u)du \Rightarrow g=-\frac{1}{2}\cos(2u)$de forma que
\begin{equation*}
\begin{split}
   & \int_{0}^{t}u\sin(2u)du=  -u\frac{1}{2}\cos(2u) + \frac{1}{2} \int_{0}^{t}\cos(2u) du \\
    & = \left[-u\frac{1}{2}\cos(2u) + \frac{1}{4}\sin(2u)\right]\Big|_0^t \\
    &=-\frac{t\cos(2t)}{2}+ \frac{\sin(2x)}{4}
\end{split}
\end{equation*}
Entonces, utilizando propiedades de la Transformada de Laplace se obtiene
\begin{equation*}
\begin{split}
    \mathscr{L}\{f(t)\}=\mathscr{L}\{e^{-3t}(-\frac{t\cos(2t)}{2} + \frac{\sin(2x)}{4})\}\\
    =\mathscr{L}\{-e^{-3t}\frac{t\cos(2t)}{2}\} +  \mathscr{L}\{-e^{-3t}\frac{\sin(2t)}{4}\} \\
    =-\frac{1}{2}\mathscr{L}\{e^{-3t}t\cos(2t)\} - \frac{1}{4} \mathscr{L}\{e^{-3t}\sin(2t)\}
\end{split}
\end{equation*}
Sabemos que $\mathscr{L}\{e^{at}cos(wt)\}=\frac{s-a}{(s-a)^{2}+ w^{2}}$ y  $\mathscr{L}\{e^{at}sin(wt)\}=\frac{w}{(s-a)^{2}+ w^{2}}$ y $\mathscr{L}\{t^{n}f(t)\}=(-1)^{n}F^{n}(s)$, entonces 
\begin{equation*}
    \mathscr{L}\{e^{-3t}t\cos(2t)\}= -\frac{s+3}{(s+3)^{2}+ 4}
\end{equation*}
\begin{equation*}
    \mathscr{L}\{e^{-3t}sin(2t)\}=\frac{2}{(s+3)^{2}+ 4}
\end{equation*}

Por lo tanto, juntando 
\begin{equation}
    \mathscr{L}\{f(t)\}=\frac{s+3}{2(s+3)^{2}+ 8} - \frac{2}{4(s+3)^{2}+ 16}.
\end{equation}