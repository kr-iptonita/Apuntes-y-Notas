\textcolor{blue}{Ejercicio 5.}
Un núcleo radiactivo decae de acuerdo a la ley:
\begin{equation*}
    \frac{dx}{dt}=-\lambda N
\end{equation*}
donde $N$ es la concentración de un nucleón radiactivo y $\lambda$ la constante de decaimiento. En una serie de decaimientos de 2 diferentes nucleones comenzando por $N_{1}$.
\begin{equation*}
    \frac{dN_{1}}{dt}=-\lambda_{1}N_{1} \hspace{1cm} \frac{dN_{2}}{dt}=\lambda_{1}N_{1}-\lambda_{2}N_{2}
\end{equation*}
Encuentre $N_{2}$ con las condiciones $N_{1}(0)=N_{0}$ y $N_{2}(0)=0$. \\

Construimos la matriz $A$ de forma que $\dot N=(\frac{dN_{1}}{dt},\frac{dN{2}}{dt})$
\begin{equation*}
\dot N =
    \begin{pmatrix}
    \frac{dN_{1}}{dt} \\
    \frac{dN{2}}{dt}
    \end{pmatrix}
    =
    \begin{pmatrix}
    -\lambda_{1} & 0 \\
    \lambda_{1} & -\lambda_{2}
    \end{pmatrix}
    \begin{pmatrix}
    N_{1} \\
    N_{2}
    \end{pmatrix}
\end{equation*}
Determinamos los valores propios de A, entonces
\begin{equation*}
     |A - \alpha I| = 
     \left|\begin{pmatrix}
     -\lambda_{1} - \alpha & 0 \\
    \lambda_{1} & -\lambda_{2} - \alpha
     \end{pmatrix}\right |
     =(-\lambda_{1} - \alpha )(-\lambda_{2} - \alpha)-(\lambda_{1})(0)
\end{equation*}
\begin{equation*}
    (\alpha +\lambda_{1})(\alpha + \lambda_{2})=0
\end{equation*}
Entonces $\alpha_{1}=-\lambda_{1}$ y $\alpha_{2}=-\lambda_{2}$

Caso con $\alpha_{1}=-\lambda_{1}$
\begin{equation*}
     A - (-\lambda_{1}) I = A + \lambda_{1} I =
     \begin{pmatrix}
     -\lambda_{1}  +\lambda_{1} & 0 \\
    \lambda_{1} & -\lambda_{2} +\lambda_{1}
     \end{pmatrix}
     =
     \begin{pmatrix}
     0 & 0\\
      \lambda_{1} & -\lambda_{2} +\lambda_{1}
     \end{pmatrix}
\end{equation*}
Buscamos $ker(A + \lambda_{1} I)$
\begin{equation*}
\begin{pmatrix}
     0 & 0\\
      \lambda_{1} & -\lambda_{2} +\lambda_{1}
     \end{pmatrix}
\begin{pmatrix}
v_{1}   \\
v_{2}  
\end{pmatrix} =
\begin{pmatrix}
0   \\
0  
\end{pmatrix}
\end{equation*}
\begin{equation*}
    \Rightarrow 0v_{1}+ 0v_{2}=0 \atop
    \lambda_{1} v_{1} +(-\lambda_{2} +\lambda_{1}) v_{2} =0
\end{equation*}
\begin{equation*}
    \Rightarrow v_{1}=\frac{-(-\lambda_{2} +\lambda_{1}) v_{2}}{\lambda_{1}}
\end{equation*}
Tomamos $v_{2}=1$ entonces
\begin{equation*}
  v_{1}=\frac{\lambda_{2} -\lambda_{1}}{\lambda_{1}}= \frac{\lambda_{2}}{\lambda_{1}} - 1
\end{equation*}
Por lo tanto $u_{1}= 
\begin{pmatrix}
\frac{\lambda_{2}}{\lambda_{1}} - 1 \\
1
\end{pmatrix}$

Entonces
\begin{equation*}
  N_{1}=e^{-\lambda_{1}t}
\begin{pmatrix}
\frac{\lambda_{2}}{\lambda_{1}} - 1 \\
1
\end{pmatrix}
=
\begin{pmatrix}
\frac{\lambda_{2}}{\lambda_{1}}e^{-\lambda_{1}t} - e^{-\lambda_{1}t} \\
e^{-\lambda_{1}t}
\end{pmatrix}  
\end{equation*}

Caso con $\alpha_{1}=-\lambda_{2}$
\begin{equation*}
     A - (-\lambda_{2}) I = A + \lambda_{2} I =
     \begin{pmatrix}
     -\lambda_{1}  +\lambda_{2} & 0 \\
    \lambda_{1} & -\lambda_{2} +\lambda_{2}
     \end{pmatrix}
     =
     \begin{pmatrix}
 -\lambda_{1} +\lambda_{2} & 0 \\
 \lambda_{1} & 0
     \end{pmatrix}
\end{equation*}
Buscamos $ker(A + \lambda_{2} I)$
\begin{equation*}
    \begin{pmatrix}
 -\lambda_{1} +\lambda_{2} & 0 \\
 \lambda_{1} & 0
     \end{pmatrix}
     \begin{pmatrix}
v_{1}   \\
v_{2}  
\end{pmatrix} =
\begin{pmatrix}
0   \\
0  
\end{pmatrix}
\end{equation*}
\begin{equation*}
    \Rightarrow  (-\lambda_{1} +\lambda_{2})v_{1}+ 0v_{2}=0 \atop
    \lambda_{1} v_{1} +0 v_{2} =0
\end{equation*}
Por lo tanto $u_{2}= 
\begin{pmatrix}
0 \\
1
\end{pmatrix}$
Entonces
\begin{equation*}
  N_{2}=e^{-\lambda_{2}t}  
\begin{pmatrix}
0 \\
1
\end{pmatrix}
=
\begin{pmatrix}
0 \\
e^{-\lambda_{2}t}
\end{pmatrix}
\end{equation*}
Por lo que la matriz fundamental es
\begin{equation*}
    \Phi(t)= 
    \begin{pmatrix}
    \frac{\lambda_{2}}{\lambda_{1}}e^{-\lambda_{1}t} - e^{-\lambda_{1}t} & 0 \\
    e^{-\lambda_{1}t} & e^{-\lambda_{2}t}
    \end{pmatrix}
\end{equation*}
Y la solución es 
\begin{equation}
     N=c_{1}N_{1} + c_{2}N_{2} =c_{1}\begin{pmatrix}
\frac{\lambda_{2}}{\lambda_{1}}e^{-\lambda_{1}t} - e^{-\lambda_{1}t} \\
e^{-\lambda_{1}t}
\end{pmatrix}  
+ c_{2}\begin{pmatrix}
0 \\
e^{-\lambda_{2}t}
\end{pmatrix}
\label{eq:soluciongeneralp5}
\end{equation}

Ahora bien, tenemos que por condiciones iniciales,$N(0)=(N_{0},0)$,por lo que, sustituyendo en la ecuación \ref{eq:soluciongeneralp5}, se tiene que:
\begin{equation*}
    N=\begin{pmatrix}
    N_{0} \\
    0
    \end{pmatrix}
    =
    c_{1}\begin{pmatrix}
\frac{\lambda_{2}}{\lambda_{1}}e^{-\lambda_{1}(0)} - e^{-\lambda_{1}(0)} \\
e^{-\lambda_{1}(0)}
\end{pmatrix}  
+ c_{2}\begin{pmatrix}
0 \\
e^{-\lambda_{2}(0)}
\end{pmatrix}
\end{equation*}
\begin{equation*}
    =c_{1}\begin{pmatrix}
\frac{\lambda_{2}}{\lambda_{1}}e^{0} - e^{0} \\
e^{0}
\end{pmatrix}  
+ c_{2}\begin{pmatrix}
0 \\
e^{0}
\end{pmatrix}
 =c_{1}\begin{pmatrix}
\frac{\lambda_{2}}{\lambda_{1}} - 1 \\
1
\end{pmatrix}  
+ c_{2}\begin{pmatrix}
0 \\
1
\end{pmatrix}
\end{equation*}

\begin{equation*}
    =\begin{pmatrix}
c_{1}(\frac{\lambda_{2}}{\lambda_{1}} - 1) \\
c_{1}
\end{pmatrix}  
+ \begin{pmatrix}
0 \\
c_{2}
\end{pmatrix}
\end{equation*}
\begin{equation*}
    =\begin{pmatrix}
    c_{1}(\frac{\lambda_{2}}{\lambda_{1}} - 1) \\
    c_{1}+c_{2}
    \end{pmatrix}
\end{equation*}
Entonces
\begin{equation*}
    N_{0}= c_{1}(\frac{\lambda_{2}}{\lambda_{1}} - 1) \atop
    0 = c_{1}+c_{2}
\end{equation*}
\begin{equation*}
    \Rightarrow c_{1}= \frac{N_{0}}{\frac{\lambda_{2}}{\lambda_{1}} - 1} \atop
    c_{2}= -c_{1}=-\frac{N_{0}}{\frac{\lambda_{2}}{\lambda_{1}} - 1} =\frac{N_{0}}{1 -\frac{\lambda_{2}}{\lambda_{1}} } 
\end{equation*}
Por lo  que , la solución con el problema de valor inicial es:
\begin{equation}
     N=\frac{N_{0}}{\frac{\lambda_{2}}{\lambda_{1}} - 1}\begin{pmatrix}
\frac{\lambda_{2}}{\lambda_{1}}e^{-\lambda_{1}t} - e^{-\lambda_{1}t} \\
e^{-\lambda_{1}t}
\end{pmatrix}  
+ \frac{N_{0}}{1 -\frac{\lambda_{2}}{\lambda_{1}} } \begin{pmatrix}
0 \\
e^{-\lambda_{2}t}
\end{pmatrix}
\label{eq:solvalorinicialp5}
\end{equation}
Por lo tanto 


\begin{equation}
    N_{2}=\begin{pmatrix}
    0 \\
    \frac{N_{0}}{1 -\frac{\lambda_{2}}{\lambda_{1}} }e^{-\lambda_{2}t}
    \end{pmatrix}
\end{equation}   


