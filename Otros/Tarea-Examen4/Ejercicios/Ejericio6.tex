\textcolor{blue}{Ejercicio 6.} Dada 
$J = \begin{pmatrix}
\lambda & 1\\
0 & \lambda
\end{pmatrix}$, donde $\lambda$ es un número real arbitrario\\

a) Encuentre $J^2, J^3$ y $J^4$\\

b) Use un argumento inductivo para mostrar que 
$J^n = \begin{pmatrix}
\lambda^n & n \lambda^{n-1}\\
0 & \lambda^n
\end{pmatrix}$\\

c) Encuentre $e^{Jt}$\\

d) Use $e^{Jt}$ para resolver 

$$\dot{x}= Jx$$ 

$$x(0)= \begin{pmatrix} -1\\2 \end{pmatrix}$$

$\mathbf{Sol:}$\\

a) $$J^2= J*J= 
\begin{pmatrix}
\lambda & 1\\
0 & \lambda
\end{pmatrix}
\begin{pmatrix}
\lambda & 1\\
0 & \lambda
\end{pmatrix}=
\begin{pmatrix}
\lambda^2 & \lambda + \lambda\\
0 & \lambda^2
\end{pmatrix}=
\begin{pmatrix}
\lambda^2 & 2\lambda\\
0 & \lambda^2
\end{pmatrix}$$\\

$$J^3= J^2 * J=
\begin{pmatrix}
\lambda^2 & 2\lambda\\
0 & \lambda^2
\end{pmatrix}
\begin{pmatrix}
\lambda & 1\\
0 & \lambda
\end{pmatrix}=
\begin{pmatrix}
\lambda^3 & \lambda^2 + 2\lambda^2\\
0 & \lambda^3
\end{pmatrix}=
\begin{pmatrix}
\lambda^3 & 3\lambda^2\\
0 & \lambda^3
\end{pmatrix}$$\\

$$J^4= J^3 * J= 
\begin{pmatrix}
\lambda^3 & 3\lambda^2\\
0 & \lambda^3
\end{pmatrix}
\begin{pmatrix}
\lambda & 1\\
0 & \lambda
\end{pmatrix}=
\begin{pmatrix}
\lambda^4 & \lambda^3 + 3\lambda^3\\
0 & \lambda^4
\end{pmatrix}=
\begin{pmatrix}
\lambda^4 & 4\lambda^3\\
0 & \lambda^4
\end{pmatrix}$$

b) Por inducción:\\

La base inductiva fue realizada en el inciso a)\\

Ahora, para la hipótesis inductiva tenemos:

$$J^k= \begin{pmatrix}
\lambda^k & k \lambda^{k-1}\\
0 & \lambda^k
\end{pmatrix}$$

Finalmente en el paso inductivo tenemos: 

$$J^{k+1}= J^{k}* J= \begin{pmatrix}
\lambda^k & k \lambda^{k-1}\\
0 & \lambda^k
\end{pmatrix}
\begin{pmatrix}
\lambda & 1\\
0 & \lambda
\end{pmatrix}=
\begin{pmatrix}
\lambda^k +1 & \lambda^k + k \lambda^k\\
0 & \lambda^k
\end{pmatrix}=
\begin{pmatrix}
\lambda^k +1 & (k+1)\lambda^k\\
0 & \lambda^k
\end{pmatrix}$$

c) Determine $e^{Jt}$

$$e^{Jt}= \displaystyle \sum_{k=0}^{\infty} J^k \frac{t^k}{k!}= \frac{t^0}{0!}
\begin{pmatrix}
1 & 0\\
0 & 1
\end{pmatrix}+
\begin{pmatrix}
\displaystyle \sum_{k=1}^{\infty} \frac{\lambda^k t^k}{k!} & \displaystyle \sum_{k=1}^{\infty} \frac{k \lambda^{k-1} t^k}{k!}\\
0 & \displaystyle \sum_{k=1}^{\infty} \frac{\lambda^k t^k}{k!}
\end{pmatrix}$$

$$= \begin{pmatrix}
\frac{(\lambda t)^0}{0!} + \displaystyle \sum_{k=1}^{\infty} \frac{(\lambda t)^k}{k!} & t \displaystyle \sum_{k=1}^{\infty} \frac{(\lambda t)^{k-1}}{(k-1)!}\\
0 & \frac{(\lambda t)^0}{0!} + \displaystyle \sum_{k=1}^{\infty} \frac{(\lambda t)^k}{k!}
\end{pmatrix}$$

$$=\begin{pmatrix}
\displaystyle \sum_{k=0}^{\infty} \frac{(\lambda t)^k}{k!} & t \displaystyle \sum_{k=0}^{\infty} \frac{(\lambda t)^k}{k!} \\
0 & \displaystyle \sum_{k=0}^{\infty} \frac{(\lambda t)^k}{k!}
\end{pmatrix}$$

Y así tenemos: 

$$e^{Jt}= \begin{pmatrix}
e^{\lambda t} & t e^{\lambda t}\\
0 & e^{\lambda t}
\end{pmatrix}$$

d) Dado que $e^{Jt}$ es matriz fundamental de J, sean $x_1= \begin{pmatrix}
e^{\lambda t}\\
0
\end{pmatrix}$,
$x_2= \begin{pmatrix}
t e^{\lambda t}\\
e^{\lambda t}
\end{pmatrix}$
las columnas de $e^{Jt}$, se sigue que: 

$$J x_1= \begin{pmatrix}
\lambda & 1\\
0 & \lambda
\end{pmatrix}
\begin{pmatrix}
e^{\lambda t}\\
0
\end{pmatrix}=
\begin{pmatrix}
\lambda e^{\lambda t} +1(0)\\
0 e^{\lambda t} + \lambda(0)
\end{pmatrix}=
\begin{pmatrix}
\lambda e^{\lambda t}\\
0
\end{pmatrix}
\begin{pmatrix}
\frac{d(e^{\lambda t})}{dt}\\
\frac{d(0)}{dt}
\end{pmatrix}= x'_1$$

$$J x_2= \begin{pmatrix}
\lambda & 1\\
0 & \lambda
\end{pmatrix}
\begin{pmatrix}
t e^{\lambda t}\\
e^{\lambda t}
\end{pmatrix}= 
\begin{pmatrix}
\lambda(t e^{\lambda t}) + 1(e^{\lambda t})\\
0(t e^{\lambda t}) + \lambda(e^{\lambda t})
\end{pmatrix}=
\begin{pmatrix}
\lambda t e^{\lambda t} + e^{\lambda t}\\
\lambda e^{\lambda t}
\end{pmatrix}=
\begin{pmatrix}
\frac{d(t e^{\lambda t})}{dt}\\
\frac{d(e^{\lambda t})}{dt}
\end{pmatrix} = x'_2$$

Así: 

$$x(t)= C_1 X_1 + C_2 X_2= C_1\begin{pmatrix}
e^{\lambda t}\\
0
\end{pmatrix}
+ C_2 \begin{pmatrix}
t e^{\lambda t}\\
e^{\lambda t}
\end{pmatrix}= 
\begin{pmatrix}
C_1 e^{\lambda t} + C_2 t e^{\lambda t}\\
C_2 e^{\lambda t}
\end{pmatrix}$$

Luego 

$$x(0)= C_1 \begin{pmatrix}
e^{\lambda(1)}\\
0
\end{pmatrix}
+ C_2 \begin{pmatrix}
0 e^{\lambda(0)}\\
e^{\lambda(0)}
\end{pmatrix}
= C_1 \begin{pmatrix}
1\\
0
\end{pmatrix}
+ C_2 \begin{pmatrix}
0\\
1
\end{pmatrix}=
\begin{pmatrix}
C_1\\
0
\end{pmatrix}
+ \begin{pmatrix}
0\\
C_2
\end{pmatrix}
\begin{pmatrix}
C_1\\
C_2
\end{pmatrix}=
\begin{pmatrix}
-1\\
2
\end{pmatrix}$$

$\Rightarrow C_1= -1$ y $C_2= 2$\\

$\therefore x(t)= J x_0= \begin{pmatrix}
-e^{\lambda t} + 2 t e^{\lambda t}\\
2 e^{\lambda t}
\end{pmatrix}$