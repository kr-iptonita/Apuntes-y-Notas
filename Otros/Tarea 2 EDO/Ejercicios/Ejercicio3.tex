\section{Investigar sobre la cuantificación del error en los iterados de Picard.}
Es un método desarrollado por Charles Émile Picard, es un metodo de caracter iterativo que se emplea para obtener una solucion a una ecuación diferencial, esto significa que se emplea un valor/condición inicial $x_0$ mediante el cual desencadenara las aproximaciones para cada iteración con $x_i$ donde $i\geq0$, entonces para el metodo iterativo de Picard se considera generalmente una ecuación diferencial como:
$$\frac{dx}{dt}=f(x,t)$$

donde $f(x,t): \|x-x_0\| \leq a\cdot \| t-t_0 \|\leq \mathcal{T}\rightarrow \mathcal{R}$

De este modo la idea principal es optener un proceso que tiende a una solución del \texttt{}{PVI}.

de esta forma para una función solución tal que:
$$\int _{t_0}^t\frac{dx}{dt}=\int _{t_0}^tf\left(x\left(t'\right),t'\right)dt'$$
si se elabora un despeje de $x(t)$
tenemos:

$$\:x\left(t\right)=x_0+\int _{t_0}^tf\left(x\left(t'\right),t'\right)dt'$$
Aunque no se resuelve el problema se obtiene una sucesión tal que:

$$x_{l+1}(t)=x_0+\int_{t_0}^t f(x_l(t'),t')dt'$$