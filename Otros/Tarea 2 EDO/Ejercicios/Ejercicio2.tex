\section{Ejercicio 2: ¿Tiene solución el PVI $x’ = t^{2}x-x^{5}$, con $x(2)=-3$?}
Observemos que para $ f = t^{2}x-x^{5}$ es continua, además se tiene que \par
\begin{equation*}
  \frac{f}{dx}=t^{2}-5x^{4}  
\end{equation*}


también es continua, en particular si t=2. \par \noindent
Entonces podemos afirmar que existe un rectángulo $R=\{ (t,x) | a<t<b, c<x<d \} $ que contiene al punto (2,-3) en su interior dentro del cual $f$ y $f’$ son continuas, por lo tanto, por el Teorema de Existencia y Unicidad (Picard-Lindelöf) el PVI tiene una única solución.
