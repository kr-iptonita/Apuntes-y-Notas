\section{Si $a e \neq b d$ demuestre que pueden elegirse constantes $h, k$ de modo que las sustituciones $x=U h, y=V k$ reducen la ecuación:}
$$
\frac{d y}{d x}=f\left(\frac{a x+b y+c}{d x+e y+f}\right)
$$
a una ecuación homogénea.\\

Por la regla de la cadena:\\
$$
\frac{d y}{d x}=\frac{d y}{d V} \frac{d V}{d U} \frac{d U}{d x}
$$
Pero, como $d x=d U$ y $d y=d V$, entonces:
$$
\frac{d y}{d x}=1 \cdot \frac{d V}{d U} \cdot 1=\frac{d V}{d U}
$$
Sustituyendo en la función:
$$
\frac{d V}{d U}=f\left(\frac{a U-a h+b V-b k+c}{d U-d h+e V-e k+f}\right)
$$
Si tomamos tales contantes como aquellas que cumplen el sistema:
$$
\begin{array}{l}
a h+b k=c \\
d h+e k=f
\end{array}
$$
Que tiene solución, pues:
$$
\left|\begin{array}{ll}
a & b \\
d & e
\end{array}\right|=a e-b d \neq 0
$$
por la hipóteis.\\

Luego:\\

$$
\frac{d v}{d u}=f\left(\frac{a U+b V}{d U+e V}\right)=f\left(\frac{a+b \frac{V}{U}}{d+e^{\frac{V}{U}}}\right)
$$
que es una ecuación homogénea.\\

\textbf{Ejercicio $5 a$}\\

$$
\frac{d y}{d x}=\frac{-x+2 y+4}{x+y-1}
$$
Notemos que el sistema que estamos buscando es:
$$
\begin{array}{l}
-h+2 k=4 \\
h+k=-1
\end{array}
$$
La solución, por la regla de Cramer, es:
$$
h=\frac{\left|\begin{array}{cc}
4 & 2 \\
-1 & 1
\end{array}\right|}{\left|\begin{array}{cc}
-1 & 2 \\
1 & 1
\end{array}\right|}=-2 ; \quad k=\frac{\left|\begin{array}{cc}
-1 & 4 \\
1 & -1
\end{array}\right|}{\left|\begin{array}{cc}
-1 & 2 \\
1 & 1
\end{array}\right|}=1
$$
La sustituación adecuada es:
$$
x=U+2 ; y=V-1
$$
De aquí que, haciendo la substitución:
$$
\frac{d V}{d U}=\frac{-U+2 V}{U+V}=\frac{-1+2 \frac{V}{U}}{1+\frac{V}{U}}
$$
Sea $z=\frac{V}{U} \Rightarrow V=z U \Rightarrow V^{\prime}=z+z^{\prime} U$ y sustituimos:
$$
z+z^{\prime} U=\frac{2 z-1}{z+1} \Rightarrow z^{\prime} U=\frac{2 z-1}{z+1}-z \Rightarrow z^{\prime} \frac{z+1}{-z^{2}+z-1}=\frac{1}{U}
$$

Integrando de ambos lados:
$$
\int \frac{z+1}{-z^{2}+z-1} d z=\int \frac{d U}{U} \Rightarrow-\frac{1}{2} \int \frac{2 z-1}{z^{2}-z+1} d z-\frac{3}{2} \int \frac{1}{x^{2}-x+1}=\ln U
$$
De donde:
$$
-\frac{1}{2} \ln \left(z^{2}-z+1\right)-\sqrt{3} \arctan \left(\frac{2 z-1}{\sqrt{3}}\right)=\ln U
$$
$$
e^{-\sqrt{z^{2}-z+1}}-e^{\sqrt{3} \arctan \left(\frac{2 z-1}{\sqrt{3}}\right)}=U
$$
4
$$
e^{-\sqrt{\left(\frac{y+1}{x-2}\right)^{2}-\frac{y+1}{x-2}+1}}-e^{\sqrt{3} \arctan \left(\frac{2\left(\frac{y+1}{x-2}\right)-1}{\sqrt{3}}\right)}=x-2
$$
Ejercicio 5 b
$$
\frac{d y}{d x}=\frac{x+y+4}{x+y-6}
$$
Notemos que el $a e=1$ y $b d=1$ y por lo tanto, no se puede resolver por el método anterior.