\chapter{Complejos}
\section{Generalidades}
\begin{definition}[Campo de los complejos]
Al campo $\mathbb{C}$ lo llamaremos el plano complejo y a sus elementos números complejos.
\end{definition}

Además $\forall z =a+ib$ definimos:

\begin{itemize}
  \item La parte real de $z$ como $\text{Re} \in z=a$
  \item La parte imaginaria de $z$ como $\text{Im} z=b$
\end{itemize}

\textsc{Nota:} A los números complejos $z$ tales que $\text{Re} z = 0$, se les denominará imaginarios puros. 

\begin{definition}[Conjugado]
Para cada $z \in \mathbb{C}$ con $z=a+ib$, definimos el conjugado de $z$  como $\underline{z}= a-ib$
\end{definition}
\section{Operaciones aritméticas}

$\forall z, w \in \mathbb{C}$ definimos:

\begin{itemize}
  \item \textsc{Resta: } $z-w=z+(-w)$
  \item \textsc{División: }Si$w \neq 0$ entonces $\frac{z}{w} = zw^{-1}$
\end{itemize}

\subsection{Proposición}
\textbf{Sea $z,w \in \mathbb{C}$ entonces:}
\begin{enumerate}
  \item $\underline{z \pm w} = \underline{z} \pm \underline{w}$
  \item $\underline{zw}= \underline{z} \cdot \underline{w}$
  \item Si $z \neq 0$, $\underline{z^{-1}}=\underline{z}^{-1}$
  \item Si $w \neq 0$, $\underline{\left(\displaystyle\frac{z}{w}\right)} = \displaystyle\frac{\underline{z}}{\underline{w}}$
  \item $\underline{\underline{z}} = z$
  \item $z$ es un número real, si y solo si $\underline{z}=z$
\end{enumerate}

\section{Campo de los número complejos}
Sea $\mathbb{R^2}, +, \circ$, donde:\\
$+ : \mathbb{R}^2 \times \mathbb{R}^2 \rightarrow \mathbb{R}^2$
\[
  (a,b)+(c,d)=(a+c, b+d)
\]
$\cdot \mathbb{R}^2 \times \mathbb{R}^2 \rightarrow \mathbb{R}^2$
\[
  (a,b) (c,d) = (ac-bd, ad+bc)
\]

\subsection{Proposición}
$\mathbb{C}:= (\mathbb{R}^2,+,\cdot)$ es un campo donde:
\begin{itemize}
  \item \textsc{Neutro aditivo} $0 = (0,0)$
  \item \textsc{Neutro Multiplicativo} $1 = (1,0)$
  \item \textsc{Inverso aditivo} $\forall z = (a,b) \in \mathbb{C} \rightarrow -z = (-a, -b)$
  \item \textsc{Inverso Multiplicativo} $\forall z = (a,b)\neq 0 \rightarrow z^{-1} = \left( \displaystyle\frac{a}{a^2 +b^2}, \displaystyle\frac{-b}{a^2 +b^2} \right)$
\end{itemize}

\textbf{Observación:} $\forall a,b \in \mathbb{R}, (a,0)+(b,0)=(a+b,0) \approx a+b$,
además $(a,0)(b,0)=(ab,0) \approx ab$

Si consideramos $A=\{ (a,b)| a\in \mathbb{R} \}$ entonces $(A,+,\cdot)$ es un \textit{sub-conjunto} de $\mathbb{C}$. De esta manera la función:
\[
  f:\mathbb{R}\rightarrow A, f(a)= (a,0)
\]
es un isomorfismo de campos, es decir:
\begin{itemize}
  \item $f$ es biyectiva
  \item $f(a+b)=f(a)+f(b)$
  \item $f(ab)=f(a)+f(b)$
\end{itemize}

\textsc{Consecuencia:}
\begin{itemize}
  \item $f(0) = (0,0)$
  \item $f(-a) = (-a,0)$
  \item $f(a^{-1}) = f(a)^{-1}$ si $a \neq 0$
\end{itemize}

\begin{definition}[Unidad imaginaria]
  Se define como $i=(0,1)$
\end{definition}

Observación:
\[
  i^2 = (0,1)(0,1) = (-1,0) = -1
\]
Entonces:
\[
  i = \sqrt{-1}
\]

\subsection*{Notación}
\begin{itemize}
  \item $Edx$ real : $\mathbb{R}=\{(a,0) | a \in \mathbb{R}\}$
  \item $Edx$ imaginario: $i\mathbb{R} = \{ia | a \in \mathbb{R}\}$
\end{itemize}

\subsection*{Proposición}
\textsc{(Notación $a+ib$)} Cada $z \in \mathbb{C}$ se puede expresar de forma unica como: $z=a+ib$ donde $a,b \in \mathbb{R}$

