\subsection*{Ejercicios Extra:}
\begin{itemize}
  \item $|\text{Re}(z)|+|\text{Im}(z)|\leq \sqrt{2}|z|$
  \item $2|\text{Re}(z)||\text{Im}(z)|\leq |z|^2$
\end{itemize}

Observación: $\forall z \in \mathbb{C}$ 
\[
  z\underline{z}=\text{Re}(z)^2+\text{Im}(z)^2 \geq 0
\]
\begin{definition}[El módulo de $z$]
  Se define como $|z|=\sqrt{z\underline{z}}$
\end{definition}

\textsc{Nota: } Si $\text{Im}(z)=0$ entonces $|z|$ corresponde al valor absoluto.\\

Observación: $\forall z \in \mathbb{C}$
\begin{itemize}
  \item $|\text{Re}(z)|\leq |z|$ y $|\text{Im}(z)|\leq |z|$
  \item $|z|\leq |\text{Re}(z)|+|\text{Im}(z)|$
\end{itemize}

\[
  \therefore \max \{ |\text{Re}(z)|, |\text{Im}(z)| \}\leq |z| \leq |\text{Re}(z)|+|\text{Im}(z)|
\]

Sea $z=a+ib$:\\
$|z|^2\leq (|\text{Re}(z)|+|\text{Im}(z)|)^2 \Leftrightarrow a^2+b^2 \leq (|a|+|b|)^2 \Leftrightarrow a^2+b^2 \leq a^2+ 2|a||b|+b^2 \Leftrightarrow 0\leq |a||b|$

Observación: Como $z\underline{z} = |z|^2$ si $z\neq 0$
\[
  \Rightarrow z\displaystyle\frac{z}{ |z|^2 }=1 \therefore z^{-1}=\displaystyle\frac{\underline{z}}{|z|^2}
\]

\textit{Ejemplo:} $i^{-1}=\frac{1}{i}=-i$

De lo anterior: $\forall z,w \in \mathbb{C}$, $w \neq 0$\\
\[
  \displaystyle\frac{z}{w}=zw^{-1}=\displaystyle\frac{z\underline{w}}{|w|^2}
\]

\subsection{Propiedades de:}
$\operatorname{Re}(z), \operatorname{Im}(z), \bar{z},|z|$

Sea $z=a+i b$

\[
 \begin{aligned}
  z+\bar{z}=a+i b+a-i b=2 a \Rightarrow a=\frac{z+\bar{z}}{2} & \therefore \operatorname{Re}(z)=\frac{z+\bar{z}}{2} \\
  z-\bar{z}=a+i b-a+i b=i 2 b \Rightarrow b=\frac{z-\bar{z}}{2 i} & \therefore \operatorname{Im}(z)=-i \frac{z-\bar{z}}{2}
 \end{aligned}
\]

\subsection{Proposición}
$\forall z, w \in \mathbb{C}$
\begin{itemize}
  \item $\operatorname{Re}(z)=\operatorname{Re}(\bar{z})=\frac{z+\bar{z}}{2}$ 
  \item $\operatorname{Im}(z)=-\operatorname{Im}(\bar{z})=-i \frac{z-\bar{z}}{2}$ 
  \item $\bar{z}=\operatorname{Re}(z)-i \operatorname{Im}(z)$ 
  \item La parte real e imaginarias son $\mathbb{R}$-lineales es decir $\operatorname{Re}(z+tw)=\operatorname{Re}(z)+t \operatorname{Re}(w)$\\
  y $\operatorname{Im}(z+t w)=\operatorname{Im}(z+t w)=\operatorname{Im}(z)+t \operatorname{Im}(w)$, donde  $t \in \mathbb{R}$\\
  $\begin{aligned}
       \operatorname{Re}(z w)&=\operatorname{Re}(z) \operatorname{Re}(w)-\operatorname{Im}(z) \operatorname{Im}(w) \\
       \operatorname{Im}(z w)&=\operatorname{Im}(z)\operatorname{Im}(w)-\operatorname{Re}(z) \operatorname{Re}(w)
   \end{aligned}$
\end{itemize}

\subsection{Propopsición}
$\forall z,w \in \mathbb{C}$
\begin{itemize}
  \item $|z|=0 \Leftrightarrow z=0$
  \item $|z|=|\bar{z}|$
  \item $|z+w|^2 = |z|^2 +2\operatorname{Re}(z\bar{w})+|w|^2$
        \begin{proof}[Demostración: ]
          $|z+w|^2 = (z+w)(\bar{z+w}) = (z+w)(\bar{z}+\bar{w})=z\bar{z}+z\bar{w}+\bar{z}w+w\bar{w}$
          $=|z|^2+z\bar{w}+ \bar{z\bar{w}}+|w|^2 = |z|^2 + 2\operatorname{Re}(zw)+|w|^2$
        \end{proof}
        \begin{definition}[Ley de cosenos: Ecuación cuando el signo es menos]
          \[
            |z-w|^2 = |z|^2 - 2|z||w|\cos{\theta} + |w|^2 
          \]
          \textbf{Recordando que: }
          \[
            \cos{\theta} = \displaystyle\frac{u \cdot v}{\|u\| \|v\|}  
          \]
          \[
            \operatorname(z\bar{w})= \operatorname(Re)[(a+ib)(c-id)]=ac+bd
          \]
        \end{definition}
  \item \begin{definition}[Ley del paralelogramo]
         $|z+w|^2 + |z-w|^2 = 2(|z|^2 + |w|^2)$
        \end{definition}
  \item $|zw| = |z||w|$ además $|zw|^2 = (zw)(\bar{zw}) = (z\bar{z})(w\bar{w}) = |z|^2 |w|^2$
  \item Si $z \neq 0$, $|z^{-1}| = |z|^{-1}$
  \item Si $w \neq 0$, $|\displaystyle\frac{z}{w}|=\displaystyle\frac{|z|}{|w|} \Rightarrow |z| = |\frac{z}{w} w| = |\frac{z}{w}||w| \Rightarrow \frac{|z|}{|w|} = |\frac{z}{w}|$
  \item $|z^n| = |z|^n$ donde $n \in \mathbb{N}$
\end{itemize}

