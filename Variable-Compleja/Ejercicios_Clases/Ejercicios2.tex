\section{Ejercicios 2}
\textbf{\textit{Ejercicios elaborados por: Karla Romina Juárez Torres, Nº cuenta : 318013712}}

\subsection{Sea $z \in \mathbb{C}$ Demuestra las siguientes desigualdades.}
\textbf{$2|\operatorname{Re}(z)||\operatorname{Im}(z)|\leq |z|^2$}

Expresamos $ z $ en términos de su parte real y su parte imaginaria:

\[ z = x + yi \]

donde $ x = \operatorname{Re}(z) $ y $ y = \operatorname{Im}(z) $.

Entonces, la desigualdad se convierte en:

\[ 2|x| \cdot |y| \leq |x + yi|^2 \]

Usando la definición del módulo de un número complejo:

\[ |x + yi|^2 = (x + yi)(x - yi) = x^2 + y^2 \]

Entonces, la desigualdad se convierte en:

\[ 2|x| \cdot |y| \leq x^2 + y^2 \]

Aplicando la desigualdad de la media aritmética-geométrica (AM-GM), tenemos:

\[ \frac{x^2 + y^2}{2} \geq 2|x| \cdot |y| \]

Por lo tanto,

\[ x^2 + y^2 \geq 2|x| \cdot |y| \]

Entonces,

\[ |x + yi|^2 \geq 2|x| \cdot |y| \]

Por lo tanto $ 2|\operatorname{Re}(z)||\operatorname{Im}(z)|\leq |z|^2 $.

\textbf{$ |\operatorname{Re}(z)|+|\operatorname{Im}(z)| \leq \sqrt{2}|z|$}

Expresamos $ z $ en términos de su parte real y su parte imaginaria:

\[ z = x + yi \]

donde $ x = \operatorname{Re}(z) $ y $ y = \operatorname{Im}(z) $.

Entonces, la desigualdad se convierte en:

\[ |\operatorname{Re}(z)| + |\operatorname{Im}(z)| \leq \sqrt{2}|x + yi| \]

Usando la definición del módulo de un número complejo:

\[ |x + yi| = \sqrt{x^2 + y^2} \]

Entonces, la desigualdad se convierte en:

\[ |\operatorname{Re}(z)| + |\operatorname{Im}(z)| \leq \sqrt{2}\sqrt{x^2 + y^2} \]

Ahora, aplicamos la desigualdad triangular para módulos:

\[ |\operatorname{Re}(z)| + |\operatorname{Im}(z)| \leq \sqrt{2}(|\operatorname{Re}(z)| + |\operatorname{Im}(z)|) \]

Definimos $ a = |\operatorname{Re}(z)| $ y $ b = |\operatorname{Im}(z)| $, entonces:

\[ a + b \leq \sqrt{2}(a + b) \]

Dividimos ambos lados de la desigualdad por $ a + b $ (que es positivo porque es la suma de dos valores absolutos):

\[ 1 \leq \sqrt{2} \]

Como $ \sqrt{2} $ es mayor que 1, la desigualdad $ 1 \leq \sqrt{2} $ es verdadera.

Por lo tanto 
\[ |\operatorname{Re}(z)| + |\operatorname{Im}(z)| \leq \sqrt{2}|z| \].

\subsection{Sean $z_1,z_2,\dots, z_n \in \mathbb{C}$ y $t_1, t_2, \dots, t_n \in \mathbb{R}$ tales que $|z_i|\leq 1$, $t_i \geq 0$, para cada $i=1,2,\dots n$, y $t_1+t_2+,\dots t_n = 1$ Demuestra que: $|t_1 z_1+t_2 z_2+\dots+t_n z_n| < 1$}


Sea $ z_i $ en términos de sus partes real e imaginaria:

\[ z_i = x_i + y_i \]

donde $ x_i = \operatorname{Re}(z_i) $ y $ y_i = \operatorname{Im}(z_i) $.

Entonces, tenemos:

\[ |t_1 z_1 + t_2 z_2 + \dots + t_n z_n| = |t_1(x_1 + y_1) + t_2(x_2 + y_2) + \dots + t_n(x_n + y_n)| \]

Por la desigualdad triangular:

\[ |t_1(x_1 + y_1) + t_2(x_2 + y_2) + \dots + t_n(x_n + y_n)| \leq |t_1(x_1) + t_1(y_1)| + |t_2(x_2) + t_2(y_2)| + \dots + |t_n(x_n) + t_n(y_n)| \]

Dado que $ |t_i| = t_i $ para $ t_i \geq 0 $, podemos simplificar esto aún más:

\[ |t_1(x_1) + t_1(y_1)| + |t_2(x_2) + t_2(y_2)| + \dots + |t_n(x_n) + t_n(y_n)| = t_1|x_1 + y_1| + t_2|x_2 + y_2| + \dots + t_n|x_n + y_n| \]

Dado que $ |z_i| \leq 1 $, tenemos que $ |x_i + y_i| \leq 1 $, por lo tanto:

\[ t_1|x_1 + y_1| + t_2|x_2 + y_2| + \dots + t_n|x_n + y_n| \leq t_1 + t_2 + \dots + t_n \]

Dado que $ t_1 + t_2 + \dots + t_n = 1 $

Por lo tanto, $ |t_1 z_1 + t_2 z_2 + \dots + t_n z_n| < 1 $.

, tenemos que:

\[ t_1 + t_2 + \dots + t_n = 1 \]

\subsection{Sean $z,w \in \mathbb{C}$, con $w \neq 0$. Demuestra que  $|z+w|^2 =|z^2|+|w|^2$ si y sólo si, $\operatorname{Re}\frac{z}{w}=0$}

Expresamos $ z $ y $ w $ en términos de sus partes real e imaginaria.

Dado que $ z = a + bi $ y $ w = c + di $, donde $ a $, $ b $, $ c $, $ d $ son números reales, tenemos:

\[ |z+w|^2 = |(a+c) + (b+d)i|^2 \]

Usando la definición del módulo de un número complejo:

\[ |(a+c) + (b+d)i|^2 = (a+c)^2 + (b+d)^2 \]

Ahora, expresamos $ z^2 $ en términos de sus partes real e imaginaria:

\[ z^2 = (a^2 - b^2) + 2abi \]

Usando la definición del módulo de un número complejo:

\[ |z^2| = |(a^2 - b^2) + 2abi| = (a^2 - b^2)^2 + (2ab)^2 \]

Dado que la igualdad $ |z+w|^2 = |z^2|+|w|^2 $ se mantiene si y solo si:

\[ (a+c)^2 + (b+d)^2 = (a^2 - b^2)^2 + (2ab)^2 + c^2 + d^2 \]

Expandiendo esta ecuación y cancelando los términos comunes, obtenemos:

\[ 2ac + 2bd = 0 \]

Dividiendo ambos lados por 2, obtenemos:

\[ ac + bd = 0 \]

Ahora, la parte real de $ \frac{z}{w} $ es:

\[ \operatorname{Re}\left(\frac{z}{w}\right) = \frac{ac + bd}{c^2 + d^2} \]

Entonces, $ \operatorname{Re}\left(\frac{z}{w}\right) = 0 $ si y solo si $ ac + bd = 0 $.

Por lo tanto $ |z+w|^2 = |z^2|+|w|^2 $ si y solo si $ \operatorname{Re}\frac{z}{w} = 0 $.
