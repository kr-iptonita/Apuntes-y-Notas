\section*{Determinantes}



La inflación es un fenómeno económico que se refiere al aumento general de los precios de los bienes y servicios. En México, la inflación ha sido un problema recurrente que ha afectado la economía y la vida de los ciudadanos. Los determinantes de la inflación en México son múltiples y pueden ser clasificados en dos categorías: determinantes exógenos y determinantes endógenos.\\

Determinantes exógenos:\\

\begin{itemize}
    \item   Cambios en el tipo de cambio: una devaluación de la moneda puede aumentar los precios de los productos importados y, por lo tanto, aumentar la inflación.
        
    \item   Precios de los productos básicos: los precios de los productos básicos como el petróleo, los alimentos y los materiales, pueden afectar la inflación.
            
    \item   Shock externo: eventos como las crisis financieras o las guerras pueden aumentar los precios de los productos importados y aumentar la inflación.
\end{itemize}


Determinantes endógenos:
\begin{itemize}
    \item   Demanda agregada: un aumento en la demanda agregada puede generar un aumento en los precios de los productos y, por lo tanto, aumentar la inflación.
    \item   Oferta agregada: una disminución en la oferta agregada puede generar un aumento en los precios de los productos y, por lo tanto, aumentar la inflación.
    \item   Política fiscal: los gastos públicos y los impuestos pueden afectar la inflación.
\end{itemize}



En cuanto a la política monetaria, es importante considerar la estabilidad financiera para mantener una inflación baja y controlada. La política monetaria puede ser usada para controlar la inflación a través de medidas como la regulación de la oferta de dinero y la tasa de interés. La política monetaria puede ser efectiva en la lucha contra la inflación siempre y cuando sea coherente y consistente.\\

La política monetaria es un conjunto de acciones y medidas implementadas por un banco central para influir en la oferta de dinero y en las condiciones crediticias en una economía con el objetivo de alcanzar determinados objetivos económicos, como el control de la inflación, el crecimiento económico, la estabilidad financiera y la estabilidad de los precios.\\ 

Hay varias herramientas que un banco central puede utilizar para implementar su política monetaria:\\

\begin{enumerate}
    \item Tasa de interés
    \item Oferta de dinero
    \item Sistema de descuento
    \item Requisitos de reserva
\end{enumerate}

\textsc{Tasa de interés}

Tasa de interés: la tasa de interés es una de las herramientas más importantes para la política monetaria. Al aumentar las tasas de interés, el banco central puede disuadir el consumo y la inversión, reduciendo así la demanda agregada y la inflación. Por otro lado, al disminuir las tasas de interés, el banco central puede estimular el consumo y la inversión, aumentando la demanda agregada y el crecimiento económico.\\

\textsc{Oferta de dinero}

El banco central también puede controlar la cantidad de dinero en circulación en la economía, lo que afecta la demanda agregada y, por lo tanto, la inflación.\\

\textsc{Sistema de descuento}

El sistema de descuento es una herramienta utilizada por el banco central para proveer préstamos a los bancos comerciales, lo que les permite a ellos prestar más dinero a los consumidores y las empresas.\\

\textsc{Requisitos de la reserva}

Los requisitos de reserva son la cantidad mínima de fondos que los bancos comerciales deben mantener en su cuenta con el banco central. Al aumentar los requisitos de reserva, el banco central puede disminuir la cantidad de dinero disponible para prestar, reduciendo así la demanda agregada y la inflación.\\

En resumen, la política monetaria es una herramienta poderosa para el control de la inflación y el logro de otros objetivos económicos, pero debe ser utilizada con cuidado para evitar consecuencias no deseadas.\\

En conclusión, para combatir la inflación en México es necesario considerar tanto los determinantes exógenos como endógenos y utilizar herramientas como la política monetaria de manera efectiva. La estabilidad financiera debe ser una prioridad en la toma de decisiones y la política monetaria debe ser coherente y consistente para lograr una inflación baja y controlada.\\