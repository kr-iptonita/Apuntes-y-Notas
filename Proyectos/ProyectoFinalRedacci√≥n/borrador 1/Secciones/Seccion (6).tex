\section{Conclusiones}

La teoría de juegos ofrece herramientas y conceptos que son útiles para analizar situaciones de 
juegos del todo o nada, como el juego de Monty Hall y Deal or No Deal. 
Al utilizar la teoría de juegos, es posible identificar estrategias óptimas para maximizar 
las ganancias en estos juegos y evaluar las limitaciones de la teoría en su aplicación a juegos 
del todo o nada.\\

En el caso del juego de Monty Hall, la estrategia óptima es cambiar de elección después de que se
revele una de las puertas sin el premio. Esta estrategia se basa en el equilibrio de Nash, 
que indica que cambiar de elección tiene una mayor probabilidad de ganar que quedarse con la elección
original. En cuanto al juego de Deal or No Deal, es posible utilizar la teoría de juegos para 
evaluar las ofertas del banquero y determinar cuál es la mejor estrategia en términos de ganancias 
esperadas.\\

Es importante tener en cuenta que la teoría de juegos tiene sus limitaciones en la aplicación 
a situaciones de juegos del todo o nada. Por ejemplo, en ambos juegos se asume que los 
jugadores tienen información completa y actúan racionalmente, lo que puede no ser el caso en 
situaciones reales. Además, la teoría de juegos no puede tener en cuenta factores como la emoción, 
la intuición y otros aspectos subjetivos que pueden influir en la toma de decisiones de los jugadores.\\

En conclusión, la teoría de juegos es una herramienta útil para analizar situaciones de 
juegos del todo o nada y determinar estrategias óptimas. Sin embargo, es importante tener en 
cuenta sus limitaciones y considerar otros factores que pueden influir en la toma de decisiones 
de los jugadores.\\
