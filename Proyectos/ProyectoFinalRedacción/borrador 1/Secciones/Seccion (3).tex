\section{Juegos del todo o nada}

Los juegos del todo o nada tienen características específicas que los hacen diferentes a otros tipos 
de juegos. En estos juegos, los jugadores tienen opciones limitadas y el resultado depende de la 
elección realizada por ambos.\\

Además, los juegos del todo o nada son juegos de suma cero, lo que significa que el beneficio de un 
jugador es el perjuicio del otro. En este tipo de juegos, los jugadores tienen que tomar decisiones 
estratégicas para maximizar sus ganancias y minimizar sus pérdidas.\\

Un ejemplo de juego del todo o nada es el problema de Monty Hall, que se mencionó anteriormente 
en la introducción. Otro ejemplo es el juego de póker, en el que los jugadores tienen que tomar 
decisiones estratégicas para maximizar sus ganancias y minimizar sus pérdidas.\\

En el juego de póker, los jugadores tienen que tomar decisiones basadas en la información que tienen, 
pero también tienen que considerar la información que los otros jugadores están mostrando.
En este sentido, el póker es un juego de información imperfecta, lo que significa que los jugadores 
no tienen acceso a toda la información que necesitan para tomar una decisión óptima.\\

Otro ejemplo de juego del todo o nada es el programa de televisión "Deal or no Deal", en el que 
los concursantes tienen que tomar decisiones sobre cuándo aceptar una oferta para vender su caja y
cuándo seguir jugando para tener la oportunidad de ganar más dinero.\\

En este juego, los concursantes tienen que tomar decisiones basadas en su propia percepción del 
valor de su caja, pero también tienen que considerar las ofertas que les hace el presentador del
programa. Además, los concursantes tienen que considerar el riesgo de perder todo si continúan 
jugando y no ganan el premio mayor.
