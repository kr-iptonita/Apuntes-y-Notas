\section{Limitaciones de la teoría de juegos en juegos del todo o nada}

La teoría de juegos es una herramienta útil para analizar los juegos del todo o nada y puede 
proporcionar una comprensión valiosa de las estrategias óptimas. Sin embargo, 
también tiene algunas limitaciones que es importante tener en cuenta.\\

Información incompleta o incorrecta: En muchos juegos del todo o nada, los jugadores 
no tienen acceso a toda la información relevante. Por ejemplo, en el juego de Deal or No Deal, 
los jugadores no saben qué hay en cada maleta antes de abrirlas. Esto puede hacer que sea difícil 
aplicar la teoría de juegos de manera efectiva, ya que los jugadores no tienen una comprensión 
completa de la situación.\\

Asimetría de información: Además de la información incompleta o incorrecta, también puede haber 
asimetría de información entre los jugadores. En algunos juegos del todo o nada, uno de los 
jugadores puede tener más información que el otro. Por ejemplo, en el juego de Monty Hall, 
el presentador sabe qué hay detrás de cada puerta y puede influir en la elección del jugador. 
Esto puede hacer que sea difícil aplicar la teoría de juegos de manera efectiva, ya que los jugadores 
no tienen acceso a la misma información.\\

Emociones y preferencias subjetivas: La teoría de juegos se basa en la suposición de que los 
jugadores son racionales y toman decisiones basadas en el resultado esperado. 
Sin embargo, en la realidad, los jugadores pueden tener emociones y preferencias subjetivas 
que pueden influir en su toma de decisiones. Por ejemplo, un jugador en Deal or No Deal puede 
tener una conexión emocional con una maleta en particular, lo que puede hacer que tomen decisiones 
irracionales.\\

Competencia y habilidad: En muchos juegos del todo o nada, la competencia y la habilidad de los 
jugadores pueden influir en el resultado. Por ejemplo, en el póker, los jugadores pueden usar 
diferentes estrategias para engañar a sus oponentes y ganar la partida. Esto puede hacer que sea 
difícil aplicar la teoría de juegos de manera efectiva, ya que los jugadores no solo están tomando 
decisiones racionales, sino que también están compitiendo con otros jugadores que pueden estar 
utilizando estrategias impredecibles.\\

Ante estas limitaciones, es importante tener en cuenta que la teoría de juegos es una herramienta 
útil, pero no siempre es la única forma de analizar los juegos del todo o nada. 
Es importante considerar también otros factores, como las emociones, la competencia y la habilidad, 
para comprender completamente la dinámica de estos juegos y tomar decisiones informadas.\\

