\section{Considerar el siguiente problema}

\textbf{Cambio de Base}
\textsc{Dado un número en base 6 convertirlo a binario, la entrada es un arreglo de digitos en base 6 y la salida es un arreglo de bits}

\subsection{Diseñar usando Inducción Matematica un algoritmo eficiente que solucione el problema y que use el 
menor número de comparaciones}
\textbf{Caso Base}: Cuando $n=1$, el número en base 6 es simplemente un dígito, que se puede convertir a su 
representación binaria de 3 bits usando una tabla de búsqueda o una serie de comparaciones. El resultado es un arreglo de 3 bits.\\
\textbf{Hipotesis}: Supongamos que ya hemos convertido los primeros $k$ dígitos del número en base 6 a su 
representación binaria de longitud $k*3$ bits. \\
\textbf{Paso Inductivo}: Para convertir el dígito $k+1$, podemos utilizar una tabla de búsqueda o una serie 
de comparaciones, como se hizo en el caso base. Sin embargo, en lugar de comparar cada dígito con los 6 posibles valores en base 6, 
podemos aprovechar el hecho de que los dígitos en base 6 solo tienen 3 bits de información. Podemos comparar el 
valor del dígito con los valores 0, 1, 2, 4, 5 y 6 en su representación binaria de 3 bits. Si el valor del dígito 
es mayor que 6, podemos dividirlo en dos dígitos en base 6 y convertirlos por separado.\\
Una vez que tenemos la representación binaria del dígito $k+1$, podemos concatenarla con la representación 
binaria de los primeros $k$ dígitos para obtener la representación binaria de los primeros $k+1$ dígitos. 
En total, se necesitan $k*3+3$ comparaciones para convertir los primeros $k+1$ dígitos.\\
Finalmente, cuando hemos convertido todos los dígitos, obtenemos la representación binaria completa. El número total de comparaciones necesarias para convertir un número en base 6 de longitud $n$ a su representación binaria de longitud $m$ es:
    \[(n-1)*3 + \log_2(m)\]
donde $\log_2(m)$ es el número de bits necesarios para representar la longitud $m$ en binario. El término $(n-1)*3$ representa el número de comparaciones necesarias para convertir los $n-1$ primeros dígitos, y $\log_2(m)$ representa el número de bits necesarios para representar la longitud $m$ en binario, lo que se puede calcular en tiempo constante.\\

Finalmente obtenemos el siguiente algoritmo

\begin{lstlisting}[language=python]

\end{lstlisting}

\subsection{Determinar la complejidad del algoritmo obtenido}
