\section{Los dos mayores elementos de un Conjunto
Dada una secuencia $S= [x_1, x_2, . . . , x_n]$ encontrar a los dos mayores elementos de $S$ usando la menor cantidad posible de comparaciones.}

\subsection{Diseñar, usando Inducción matematica, un algoritmo que resuelva el problema}

\textbf{Caso Base}:  Si la secuencia S tiene longitud 1, el algoritmo devuelve el único elemento como 
el mayor 
y el segundo mayor, y no se realiza ninguna comparación. Por lo tanto, la afirmación es verdadera para 
 secuenciasde longitud 1.\\
\textbf{Hipótesis de Inducción} Supongamos que el algoritmo es correcto para todas las secuencias de 
longitud $k$, donde $k$ es un entero positivo menor que $n$.\\
\textbf{Paso Inductivo}\\
Consideremos una secuencia S de longitud $n$. Dividimos $S$ en dos mitades, $S_1$ y $S_2$, 
de longitud $n_1$ y $n_2$, respectivamente, donde $n_1$ y $n_2$ son enteros positivos que suman $n$. 
Por hipótesis de inducción, el algoritmo encuentra los dos mayores elementos de $S_1$ y $S_2$ utilizando 
la menor cantidad posible de comparaciones.\\

Encontrar los dos mayores elementos de $S$ utilizando el algoritmo implica encontrar los 
dos mayores elementos de $S_1$ y $S_2$ y compararlos para obtener los dos mayores elementos de $S$. 
El número de comparaciones necesarias para encontrar los dos mayores elementos de $S$ es el número de 
comparaciones necesarias para encontrar los dos mayores elementos de $S_1$ y $S_2$, más una comparación 
adicional para comparar los dos mayores elementos encontrados en $S_1$ y $S_2$.\\

Por lo tanto, el número de comparaciones necesarias para encontrar los dos mayores elementos de $S$ 
es la menor cantidad posible de comparaciones, ya que se utiliza la menor cantidad posible de comparaciones 
para encontrar los dos mayores elementos de $S_1$ y $S_2$.

El algoritmo se puede describir utilizando el siguiente pseudocódigo:

\begin{lstlisting}[language=python]

function Dosgrandes(S):
    if length(S) == 1:
        return S[1], None
    elif length(S) == 2:
        if S[1] >= S[2]:
            return S[1], S[2]
        else:
            return S[2], S[1]
    else:
        n = length(S)
        n1 = floor(n/2)
        n2 = n - n1
        S1 = S[1:n1]
        S2 = S[n1+1:n]
        max1, max2 = Dosgrandes(S1)
        max3, max4 = Dosgrandes(S2)
        if max1 >= max3:
            if max1 >= max4:
                if max2 >= max4:
                    return max1, max2
                else:
                    return max1, max4
            else:
                return max3, max4
        else:
            if max3 >= max2:
                if max4 >= max2:
                    return max3, max4
                else:
                    return max3, max2
            else:
                return max1, max2
\end{lstlisting}


\subsection{Determinar el número de comparaciones que realiza el algoritmo propuesto}

Depende del tamaño de la secuencia de entrada. Sea $n$ la longitud de la secuencia $S$.

\begin{itemize}
  \item Si $n = 1$, el algoritmo no realiza ninguna comparación.
  \item Si $n = 2$, el algoritmo realiza una comparación para determinar cuál de los dos elementos es mayor.
  \item Si $n > 2$, el algoritmo divide la secuencia en dos mitades de tamaño aproximadamente igual y 
  llama recursivamente al algoritmo para cada mitad. 
  Después, el algoritmo realiza tres comparaciones para determinar los dos mayores 
  elementos de la secuencia original a partir de los dos mayores elementos de cada mitad. 
  Por lo tanto, el número total de comparaciones es:
 \[ T(n) = T\left(\left\lfloor\frac{n}{2}\right\rfloor\right) + T\left(\left\lceil\frac{n}{2}\right\rceil\right) + 3 \]

  Donde $T(n)$ es el número total de comparaciones realizadas por el algoritmo para una 
  secuencia de longitud $n$. 
  Esta ecuación de recurrencia puede resolverse mediante el método de resolución de recurrencias.
  Podemos asumir que $n$ es 
  una potencia de 2 (es decir, $n = 2^k$ para algún entero $k$). En este caso, 
  la ecuación de recurrencia se puede escribir como:\\
 \[ T(n) = 2T\left(\frac{n}{2}\right) + 3 \]

Aplicando la regla de sustitución repetida, podemos encontrar una expresión para 
$T(n)$ en términos de $T(1)$

\begin{align*}
T(n) &= 2T\left(\frac{n}{2}\right) + 3 \\
&= 2\left(2T\left(\frac{n}{2^2}\right) + 3\right) + 3 \\
&= 2^2 T\left(\frac{n}{2^2}\right) + 2^1\cdot 3 + 2^0\cdot 3 \\
&= 2^2 T\left(\frac{n}{2^2}\right) + 2^1\cdot 3 + 2^0\cdot 3 \\
&\vdots \\
&= k\cdot 3 + 2^{k-1} T\left(\frac{n}{2^{k-1}}\right) \\
&= k\cdot 3 + 2^{k-1} T(2) \\
&= k\cdot 3 + 2^{k-1} \cdot 1 \\
&= 3n - 2
\end{align*}

Por lo tanto, el número de comparaciones que realiza el algoritmo para una secuencia de 
longitud $n$ es $3n - 2$. En otras palabras, el algoritmo es $O(n)$ lineal en el peor de los casos.

\end{itemize}
