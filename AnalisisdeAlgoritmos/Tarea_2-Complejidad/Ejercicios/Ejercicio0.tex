\section{Sea $P$ un problema. El desempeño computacional en el peor de los casos para $P$ es $O(n^2)$ y tambien es $\Omega (n \log_2 n)$ sea $A$
  un algoritmo que soluciona a $P$. ¿Cuáles de las siguientes afirmaciones resultan consistentes con la información sobre $P$?}

Las afirmaciones consistentes con la información sobre P son:\\

\begin{itemize}
	\item $\Theta (n^2)$ es consistente con la información sobre $P$, ya que se sabe que el desempeño computacional en el peor de los casos para $P$ es $O(n^2)$
	\item $\Theta (n \log n)$ es consistente con la información sobre $P$, ya que se sabe que el desempeño computacional en el peor de los casos para $P$ es
	      $\Omega (n \log n)$
	\item Sobre $O (n^{3/2})$, $O(n)$, $O(n^3)$ no son consistentes con la información de $P$, ya que ninguna de ellas está incluida en el rango de complejidad
	      de P en el peor de los casos para $P$
\end{itemize}

\section{Proporcionar un problema que satisfaga la Hipótesis del Ejercicio 1 y que cumpla al
	menos tres de los cinco incisos. Debe indicar el problema y enunciar los algoritmos.}

El problema es tan simple como el de la ordenación de números y los algoritmos que tomaremos en cuestión\\

\begin{itemize}
	\item Bubble Short. En el peor de los casos su desempeño computacional es de  $O(n^2)$
	\item Merge Short. En el peor de los casos su desempeño computacional es de $\Theta(n \log n)$
	\item Quick Short. En el peor de los casos su desempeño computacional es de $O(n^2)$, aunque en la practica generalmente se tiene el desempeño computacional de $\Theta (n \log n)$
\end{itemize}

\section{Usando la definición formal de $O$, $o$, $\omega$ y $\Omega$ justificar si los siguientes argumentos son correctos o dar un contra-ejemplo.}

\begin{itemize}
	\item $2n \in O(n^2)$ \&  $2n^2 \notin o(n^2)$:\\
	      El argumento es correcto. Para justificarlo,
	      podemos utilizar la definición formal de la notación o (pequeña o). \\

	      Decimos que $f(n) \in o(g(n))$ si existe una constante $c > 0$ tal que para todo $n$ suficientemente grande, se cumple que $f(n) \leq c*g(n)$.
	      En este caso, podemos tomar $c=1$ y vemos que $2n \leq n^2$ para $n \geq 2$, lo cual implica que $2n \in o(n^2)$.\\

	      Por otro lado, decimos que $f(n) \notin o(g(n))$ si para cualquier constante $c > 0$,
	      no existe un valor de n suficientemente grande tal que $f(n) \leq c * g(n)$.
	      En este caso, podemos tomar $c = 1$ y ver que $2n^2 > n^2$ para todo $n \geq 1$,
	      lo que implica que $2n^2$ no pertenece a $o(n^2)$. Por lo tanto, el argumento es correcto.
	\item $2n$ es $\Omega (5 \log n )$\\
	      Decimos que $f(n)$ es $\Omega(g(n))$ si existe una constante $c > 0$ y un valor $n_0$ tal que para todo $n \geq n_0$,
	      se cumple que $f(n) \geq c \cdot g(n)$. En este caso, podemos tomar $c = 1$ y ver que $2n \geq 5 \log n$ para todo
	      $n \geq 2$. Sin embargo, la definición de $\Omega$ también requiere que exista un valor $n_0$ tal que $f(n) \geq c \cdot g(n)$
	      para todo $n \geq n_0$, y esto no es cierto en este caso.

	      Para demostrar esto, podemos utilizar el límite de la razón. La razón entre $2n$ y $5 \log n$ tiende a infinito cuando $n$ tiende a infinito.
	      Esto significa que no hay un valor de $n_0$ tal que $2n \geq c \cdot 5 \log n$ para todo $n \geq n_0$ y cualquier constante $c > 0$.
	      Por lo tanto, el argumento es incorrecto.
	\item $\log 3^n$ es $\Omega (\log 4^n)$\\
	      El argumento es correcto. Para justificarlo, podemos utilizar la definición formal de la notación $\Omega$ (Omega).

	      Decimos que $f(n)$ es $\Omega(g(n))$ si existe una constante $c > 0$ y un valor $n_0$ tal que para todo $n \geq n_0$,
	      se cumple que $f(n) \geq c \cdot g(n)$. En este caso, podemos utilizar la propiedad de logaritmos que dice que $\log a^n = n \cdot \log a$
	      para cualquier base $a$. Entonces, $\log 3^n = n \cdot \log 3$ y $\log 4^n = n \cdot \log 4$.

	      Podemos tomar $c = 1/2$ y ver que $\log 3^n \geq (1/2) \cdot \log 4^n$ para todo $n \geq 1$.
	      Esto se puede demostrar tomando logaritmo natural en ambos lados de la desigualdad:

	      $$\log 3^n \geq (1/2) \cdot \log 4^n$$

	      $$n \cdot \log 3 \geq (1/2) \cdot n \cdot \log 4$$

	      $$\log 3 \geq (1/2) \cdot \log 4$$

	      $$\log 3^2 \geq \log 4$$

	      $$2 \cdot \log 3 \geq \log 4$$

	      Esta última desigualdad es cierta porque $\log 3$ es aproximadamente 1.585 y $\log 4$ es aproximadamente 1.386,
	      por lo que $2 \cdot \log 3 > \log 4$. Por lo tanto, el argumento es correcto.

	\item $5^n$ es $O (3^n)$\\
	      El argumento es incorrecto. Para justificarlo, podemos utilizar la definición formal de la notación $O$ (grande o).

	      Decimos que $f(n)$ es $O(g(n))$ si existe una constante $c > 0$ y un valor $n_0$ tal que para todo $n \geq n_0$,
	      se cumple que $f(n) \leq c \cdot g(n)$. En este caso, podemos tomar $c = 5$ y ver que $5^n \leq 5 \cdot 3^n$ para todo $n \geq 1$.

	      Sin embargo, la definición de $O$ también requiere que exista un valor $n_0$ tal que $f(n) \leq c \cdot g(n)$ para todo $n \geq n_0$,
	      y esto no es cierto en este caso. Podemos demostrar esto tomando el límite de la razón:

	      $$\lim_{n \rightarrow \infty} \frac{5^n}{3^n} = \lim_{n \rightarrow \infty} \left(\frac{5}{3}\right)^n = \infty$$

	      Esto significa que no hay un valor de $n_0$ tal que $5^n \leq c \cdot 3^n$ para todo $n \geq n_0$ y cualquier constante $c > 0$.
	      Por lo tanto, el argumento es incorrecto.

	\item Si $f(n) = 4 n^2 \log n + 7n^2 + 10n$\\
	      Podemos analizar la complejidad asintótica de la función $f(n)$ mediante la notación O (grande o).
	      Para esto, necesitamos encontrar una función $g(n)$ tal que $f(n) \leq c \cdot g(n)$ para una constante $c > 0$ y $n$ suficientemente grande.

	      Podemos observar que el término dominante en la función $f(n)$ es $4 n^2 \log n$.
	      Por lo tanto, podemos elegir $g(n) = n^2 \log n$ y $c = 5$, por ejemplo. Entonces, para $n \geq 1$, se cumple:

	      $$f(n) = 4 n^2 \log n + 7n^2 + 10n \leq 4n^2 \log n + 7n^2 \log n + 10n^2$$
	      $$= (4+7+10) n^2 \log n = 21 n^2 \log n \leq 5 n^2 \log n = c \cdot g(n)$$

	      Por lo tanto, concluimos que $f(n) \in O(n^2 \log n)$. Esto significa que la función $f(n)$
	      tiene una complejidad asintótica no peor que $n^2 \log n$, es decir, que $f(n)$ crece a una tasa no mayor que $n^2 \log n$
	      para valores suficientemente grandes de $n$.

	\item Si $f(n) = 4 n^2 \log n + 7n^2 + 10n$ entonces $f(n) \in \Theta (n^3)$
	      Para verificar si $f(n) \in \Theta(n^3)$, debemos demostrar que existen constantes $c_1$, $c_2$ y $n_0$ tales que se cumpla:

	      $$c_1 n^3 \leq f(n) \leq c_2 n^3 \quad \forall n \geq n_0.$$

	      Empecemos por la primera desigualdad:

	      $$4n^2\log n + 7n^2 + 10n \geq c_1 n^3$$

	      Dividimos ambos lados por $n^3$:

	      $$\frac{4}{\log n} + \frac{7}{n} + \frac{10}{n^2} \geq c_1$$

	      El lado izquierdo de la desigualdad anterior tiende a 0 cuando $n$ se acerca a infinito.
	      Por lo tanto, no podemos encontrar una constante $c_1$ que satisfaga la primera desigualdad para todo $n \geq n_0$.

	      Por lo tanto, la afirmación "$f(n) \in \Theta(n^3)$" es falsa. En otras palabras, $f(n)$ no está acotada superior e inferiormente por una función de la forma
	      $c n^3$ para todo $n$ suficientemente grande.

\end{itemize}
\newpage
\section{Resuelve con sumo detalle uno de los siguiente ejercicios}

\subsection{Ejercicio elegido: El Algoritmo $A$ realiza $30n^2$ y el algoritmo $B$ ejecuta $500 \ln n$ operaciones elementales.
	¿Para que valor de $n$ el Algoritmo $B$ empieza a mostrar un mejor desempeño?}

Primero, establecemos una ecuación que iguale el tiempo de ejecución de ambos algoritmos:

\begin{equation}
	30n^2 = 500 \ln n
\end{equation}

Podemos resolver esta ecuación utilizando el método de Newton-Raphson o cualquier otro método numérico, 
pero en este caso podemos resolverla analíticamente para obtener una solución exacta.

Primero, dividimos ambos lados de la ecuación por $n^2$ y luego aplicamos la función exponencial a ambos lados:

\begin{equation}
	e^{30/n^2} = e^{500 \ln n/n^2}
\end{equation}

A continuación, tomamos la raíz cuadrada de ambos lados y simplificamos:

\begin{equation}
	e^{15/n} = n^5
\end{equation}

Tomamos logaritmo natural en ambos lados:

\begin{equation}
	\frac{15}{n} = 5 \ln n
\end{equation}

Despejamos $n$:

\begin{equation}
	n = \frac{e^{15/5}}{\ln(e^{15/5})} \approx 164.72
\end{equation}

Por lo tanto, para valores de $n$ mayores a 164.72,
el Algoritmo B tendrá un mejor desempeño que el Algoritmo A.

\section{Clasificar las siguientes funciones de complejidad:}

\begin{enumerate}
	\item $6n^5 + 27$
	      Pertenece a $O(n^5)$, $\Omega(1)$, $\Theta(n^5)$ y $\omega(1)$, pero no pertenece a $o(n^2)$ ni a $\omega(n^2)$.

	\item $\frac{2n}{7} + \log n$

	      Pertenece a $O(n)$, $\Omega(\log n)$, $\Theta(n)$ y $\omega(\log n)$, pero no pertenece a $o(n^2)$ ni a $\omega(n^2)$.

	\item $2n + \ln n$

	      Pertenece a $O(n)$, $\Omega(\ln n)$, $\Theta(n)$ y $\omega(\ln n)$, pero no pertenece a $o(n^2)$ ni a $\omega(n^2)$.

	\item $3n \log \log n$

	      Pertenece a $O(n \log \log n)$, $\Omega(n)$, $\Theta(n \log \log n)$ y $\omega(1)$, pero no pertenece a $o(n^2)$ ni a $\omega(n^2)$.

	\item $5 \ln n + 8$

	      Pertenece a $O(\ln n)$, $\Omega(1)$, $\Theta(\ln n)$ y $\omega(1)$, pero no pertenece a $o(n^2)$ ni a $\omega(n^2)$.

	\item $5n^2 + n \log n$

	      Pertenece a $O(n^2)$, $\Omega(n \log n)$, $\Theta(n^2)$ y $\omega(\log n)$, pero no pertenece a $o(n^2)$ ni a $\omega(n^2)$.

	\item $7n^3 + 3n^2 + 5n$

	      Pertenece a $O(n^3)$, $\Omega(n^3)$, $\Theta(n^3)$ y $\omega(1)$, pero no pertenece a $o(n^2)$ ni a $\omega(n^2)$.

	\item $2^n + 8n^2 + 5n + \log \log n$

	      Pertenece a $O(2^n)$, $\Omega(n^2)$, $\Theta(2^n)$ y $\omega(\log n)$, pero no pertenece a $o(n^2)$ ni a $\omega(n^2)$.
\end{enumerate}

\begin{table}[htbp]
	\centering
	\caption{Tabla de pertenencia de funciones}
	\begin{tabular}{|c|c|}
		\hline
        Conjunto de funciones        & Funciones \\ \hline
        \rowcolor[HTML]{FFECEC}
	    O($n^2$)        & 6$n^5$ + 27, 5$n^2$ + n log $n$, 7$n^3$ + 3$n^2$ + 5n \\ \hline
		\rowcolor[HTML]{FFF2CC}
	    $\Omega$($n^2$) & 2$^n$ + 8$n^2$ + 5n + log log n                       \\ \hline
		\rowcolor[HTML]{D0ECE7}
	    $\Theta$($n^2$) & Ninguna de las funciones                              \\ \hline
		\rowcolor[HTML]{D0E8F9}
	    o($n^2$)        & Ninguna de las funciones                              \\ \hline
		\rowcolor[HTML]{F9D0E8}
	    $\omega$($n^2$) & 3n log log n                                          \\ \hline
	\end{tabular}
\end{table}

\section{Ejercicio Opcional}

\textbf{Supongamos que tenemos una computadora que requiere un minuto para
	resolver ejemplares de tamaño $n=1000$. ¿Qué ejemplares pueden ser ejecutados en un
	minuto si compramos una nueva computadora mil veces más rápida que la anterior,
	suponiendo las siguientes complejidades $T(n)$?}

\begin{enumerate}
	\item Si $T(n)$ es $\Theta(n)$, entonces la computadora tarda $n$ segundos en resolver un problema de tamaño $n$.
	      Si la nueva computadora es mil veces más rápida, entonces tardará $n/1000$ segundos en resolver el mismo problema.
	      Para resolver el problema en un minuto (60 segundos), podemos resolver problemas de tamaño $n \leq 60,000$.
	      En otras palabras, la nueva computadora puede resolver problemas 60 veces más grandes que la computadora original.
	\item Si $T(n)$ es $\Theta(n^3)$, entonces la computadora tarda $n^3$ segundos en resolver un problema de tamaño $n$.
	      Si la nueva computadora es mil veces más rápida, entonces tardará $(n/10)^3$ segundos en resolver el mismo problema.
	      Para resolver el problema en un minuto (60 segundos), podemos resolver problemas de tamaño $n \leq 215$.
	      En otras palabras, la nueva computadora puede resolver problemas aproximadamente 5.5 veces más grandes que la computadora original.
	\item Si $T(n)$ es $\Theta(10^n)$, entonces la computadora tarda $10^n$ segundos en resolver un problema de tamaño $n$.
	      Si la nueva computadora es mil veces más rápida, entonces tardará $10^{n-3}$ segundos en resolver el mismo problema.
	      Para resolver el problema en un minuto (60 segundos), podemos resolver problemas de tamaño $n \leq 3$.
	      En otras palabras, la nueva computadora puede resolver problemas de tamaño pequeño, con $n$ no mayor a 3.
\end{enumerate}